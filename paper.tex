\documentclass[a4paper,11pt]{article}
\pdfoutput=1 % if your are submitting a pdflatex (i.e. if you have
             % images in pdf, png or jpg format)

\usepackage{jheppub} % for details on the use of the package, please
                     % see the JHEP-author-manual

\usepackage[T1]{fontenc} % if needed
%\usepackage[utf8]{inputenc}
\usepackage{graphicx}
\usepackage{caption}
\usepackage{subcaption}

\usepackage{hepparticles}
\usepackage{hepnames}
\usepackage{hepnicenames}

\usepackage[usenames,dvipsnames]{xcolor}
\usepackage{tikz}
\usepackage{hyperref}

\usepackage{soul}% For strikeout, use \st{text} . Can be removed from submitted manuscript

\usepackage{lineno}
\linenumbers
\usepackage{cancel}

%%%Physics, mostly
\newcommand{\overbar}[1]{\mkern 1.5mu\overline{\mkern-1.5mu#1\mkern-1.5mu}\mkern 1.5mu}
\DeclareMathOperator{\Tr}{Tr}
\DeclareMathOperator{\gev}{GeV}
\DeclareMathOperator{\mev}{MeV}

\newcommand{\fou}[1]{\ensuremath{\mathscr{F}\left[#1\right]}}
\newcommand{\infou}[1]{\ensuremath{\mathscr{F}^{-1}\left[#1\right]}}
\newcommand{\psibar}{\bar{\psi}}
\newcommand{\pbar}{\overbar{p}}
\newcommand{\ubar}{\overbar{u}}
\newcommand{\vbar}{\overbar{v}}
\newcommand{\kbar}{\overbar{k}}
\newcommand{\chibar}{\overbar{\chi}}
\newcommand{\MG}{M{\footnotesize AD}G{\footnotesize RAPH}5}
\newcommand{\FNMG}{{\footnotesize M}{\tiny AD}{\footnotesize G}{\tiny RAPH} }
\newcommand{\PYTHIA}{P{\footnotesize YTHIA} }

\newcommand{\tr}[1]{\Tr\left[#1\right]}

\makeatother % End of region containing @ commands
\renewcommand{\labelenumi}{(\alph{enumi})} % Use letters for enumerate
 \DeclareMathOperator{\Sample}{Sample}
\let\vaccent=\v % rename builtin command \v{} to \vaccent{}
\renewcommand{\v}[1]{\ensuremath{\mathbf{#1}}} % for vectors
\newcommand{\gv}[1]{\ensuremath{\mbox{\boldmath$ #1 $}}} 
% for vectors of Greek letters
\newcommand{\e}[1]{\ensuremath{\times 10^{#1}}}
\newcommand{\uv}[1]{\ensuremath{\mathbf{\hat{#1}}}} % for unit vector
\newcommand{\abs}[1]{\left| #1 \right|} % for absolute value
\newcommand{\avg}[1]{\left< #1 \right>} % for average
\let\underdot=\d % rename builtin command \d{} to \underdot{}
\renewcommand{\d}[2]{\frac{d #1}{d #2}} % for derivatives
\newcommand{\dd}[2]{\frac{d^2 #1}{d #2^2}} % for double derivatives
\newcommand{\pd}[2]{\frac{\partial #1}{\partial #2}} 
% for partial derivatives
\newcommand{\pdd}[2]{\frac{\partial^2 #1}{\partial #2^2}} 
% for double partial derivatives
\newcommand{\pdn}[3]{\frac{\partial^#1 #2}{\partial #3^#1}}
\newcommand{\pdh}[1]{\frac{\partial}{\partial #1}}   %things like d/dt, 
\newcommand{\eval}[1]{\left< #1 \right>}
\newcommand{\pdc}[3]{\left( \frac{\partial #1}{\partial #2}
 \right)_{#3}} % for thermodynamic partial derivatives
\newcommand{\ket}[1]{\left| #1 \right>} % for Dirac bras
\newcommand{\bra}[1]{\left< #1 \right|} % for Dirac kets
\newcommand{\braket}[2]{\left< #1 \vphantom{#2} \right|
 \left. #2 \vphantom{#1} \right>} % for Dirac brackets
\newcommand{\matrixel}[3]{\left< #1 \vphantom{#2#3} \right|
 #2 \left| #3 \vphantom{#1#2} \right>} % for Dirac matrix elements
\newcommand{\grad}[1]{\gv{\nabla} #1} % for gradient
\let\divsymb=\div % rename builtin command \div to \divsymb
\renewcommand{\div}[1]{\gv{\nabla} \cdot #1} % for divergence
\newcommand{\curl}[1]{\gv{\nabla} \times #1} % for curl
\let\baraccent=\= % rename builtin command \= to \baraccent
\renewcommand{\=}[1]{\stackrel{#1}{=}} % for putting numbers above =
\newtheorem{prop}{Proposition}
\newtheorem{thm}{Theorem}[section]
\newtheorem{lem}[thm]{Lemma}

\newcommand{\question}[1]{\textcolor{red}{#1}}%For comments
\newcommand{\comm}[1]{\textcolor{magenta}{#1}}%For comments
\newcommand{\draft}[1]{\textcolor{blue}{#1}}%For comments

\newcommand{\fg}[1]{\textcolor{ForestGreen}{#1}}   % to avoid writing ForestGreen all the freaking time

\tikzstyle arrowformat=[scale=2.5]
\tikzstyle scalar =[dashed]
\tikzstyle mom=[draw=blue]
\tikzset{
gluon/.style={decorate, draw=black, decoration={coil, amplitude = 3pt, segment length=5pt}},
photon/.style={decorate, draw=black, decoration={snake=coil}},
zbo/.style={decorate, draw=black, decoration={snake=coil}},
fermion/.style={draw=black, postaction={decorate}, decoration={markings, mark = at position 0.55 with {\arrow[arrowformat]{>}}}},
antifermion/.style={draw=black, postaction={decorate}, decoration={markings, mark = at position 0.55 with {\arrow[arrowformat]{<}}}}
}
\usetikzlibrary{patterns}
\usetikzlibrary{arrows}
\usetikzlibrary{decorations.pathmorphing,decorations.markings, decorations.pathreplacing, trees}

\title{\boldmath Draft: Constraints on Simplified Dark Matter Models using Mono-X Collider Searches}


%% %simple case: 2 authors, same institution
%% \author{A. Uthor}
%% \author{and A. Nother Author}
%% \affiliation{Institution,\\Address, Country}

% more complex case: 4 authors, 3 institutions, 2 footnotes
\author[a,1]{Amelia J. Brennan,\note{Corresponding author.}}
\author[b]{Johanna Gramling,}
\author[b]{Thomas Jacques,}
\author[a]{and Millie F. McDonald}

% The "\note" macro will give a warning: "Ignoring empty anchor..."
% you can safely ignore it.

\affiliation[a]{The University of Melbourne, Parkville 3010, Australia}
\affiliation[b]{Universit\'{e} de Gen\`{e}ve, Quai E. Ansermet 24, 1211 Gen\`{e}ve 4, Switzerland}

% e-mail addresses: one for each author, in the same order as the authors
\emailAdd{a.brennan@student.unimelb.edu.au}
\emailAdd{johanna.gramling@cern.ch}
\emailAdd{thomas.jacques@unige.ch}
\emailAdd{milliem@student.unimelb.edu.au}

\abstract{Abstract...}

\begin{document} 
\maketitle
\flushbottom

% Keep this section clean - create a new input tex file in sections/ if a new section is required, and add it here.

%\section{Checking some latex commands}
%\label{sec:intro}
%% !TEX root = ../paper.tex

The first simplified model considered here comprises a Dirac fermion DM particle (labeled $\chi$), coupled to SM quarks via an $s$-channel vector mediator (labeled $\xi$). The couplings of the mediator to the quark pair and DM pair are denoted $g_q$ and $g_{\chi}$ respectively, and are assumed to be equal, that is, $g_q = g_{\chi}$. The width of the vector mediator is variable by hand, and in the following study three indicative widths will be used.


\begin{flushleft}
Amelia, Johanna and Thomas: Anything in \textcolor{red}{red} (\textcolor{magenta}{magenta}) is a question (comment) regarding content and anything in \textcolor{blue}{blue} is incomplete/not ideally worded.
\end{flushleft}

\begin{flushleft}
\comm{Thanks, this looks great so far! I've gone through and made a few changes to the theory section, but ran out of time, there's still quite a bit more for me to add. Could people please sign their initials on their comments so we can keep track of who's saying what?  -Tom}
\end{flushleft}

\begin{flushleft}
\textcolor{ForestGreen}{FYI - bits in green are what I've added, I'll drop the colour once others have had a chance to see (and because some of my sentences could maybe use rewriting). - Amelia}
\end{flushleft}

\section{Introduction} 
\label{sec:sec1}
\begin{flushleft}
Simplified models of dark matter have emerged as a powerful tool for the interpretation of collider, direct and indirect detection signals of dark matter. Previous searches for dark matter were conducted within the context of both Effective Field Theories (EFTs) \cite{Aad:1363019, ATLAS-CONF-2012-147, CMS-PAS-EXO-12-048, Buckley:2013jwa, Abdallah:1472683, MonoZ, MonoX} and full UV-complete theories like Supersymmetry \cite{ComppMSSM, Aad:2012ms, Aad:2012fqa, Aad:2014wea, SUSY_official_paper} and extra dimensions \cite{}.


The latter approach, though well-motivated, is typified by a broad parameter space and generally yields results which are insensitive to the wider class of dark matter models. EFT constraints, in comparison, are applicable to a broad range of models and rely on the specification of only a small set of parameters, namely the suppression scale, $\Lambda = M/g_{q}g_{DM}$ (where $M$ is the mass of the mediator and $g_{q}$ and $g_{DM}$ are the Standard Model-mediator and dark matter-mediator coupling strengths respectively) and the dark matter mass, $m_{DM}$.

\hspace{1cm} In the EFT framework, interactions between the dark and Standard Model (SM) sector are parametrised by a set of higher-dimensional effective operators \cite{}. These operators arise when the mass of the mediating particle is assumed to be significantly larger than the momentum transferred in a given interaction \cite{}. Where this is not the case, the EFT prescription can produce constraints which do not align with those of the associated UV-complete model \cite{}. This is not so important in direct detection experiments where the momentum transferred in the scattering of dark matter particles with heavy nuclei is on the order of tens of keV \cite{EFTDM, DMCons3}. Similarly, in indirect searches the annihilations of non-relativistic dark matter particles in the galactic halo occur with momentum transfers on the order of $m_{DM}$ \cite{}. However, for collider searches, where the accessible center of mass energy of two colliding baryons may be sufficient to produce the mediator on-shell, the range of validity of the EFT approach is rapidly diminishing \cite{}. To accurately probe outside this range it is now a popular practise to use simplified Models \cite{Buchmueller:2014yoa}.

\hspace{1cm} In a nutshell, simplified models arise when the heavy mediator which was integrated out in the EFT framework is reintroduced. Like EFTs, simplified models are defined by a relatively small set of parameters ($M$, $m_{DM}$, $g_{q}$ and $g_{DM}$) and admit the comparison of results obtained in the different dark matter detection avenues \cite{}. Unlike EFTs, constraints calculated within the context of a simplified model are accurate across a broad energy range ($\mathcal{O}$(GeV-TeV)), removing the question of validity \cite{}. 

\hspace{1cm} The goal of this paper is not to conduct a full survey of all possible simplified models but rather to investigate a small handful of phenomenologically distinct cases. In particular, constraints are placed on the simplified models corresponding to the most straightforward UV-completions of the D1, D5 and D8 effective operators\footnote{The D1, D5 and D8 operators form a nice starting point in the analysis of simplified models as they have been studied exhaustively in the past (see Ref. \cite{}). This attention is motivated by the fact that collider limits for the D1 and D5 (D8) operators can be readily transformed into limits on spin-independent (spin-dependent) dark matter-nucleon scattering and vice versa. With the exception of D9 and D11 (which have no simple simplified model counterparts \cite{}), the remaining effective operators induce elastic scattering which is suppressed by powers of the DM velocity or the momentum transferred \cite{Kumar}. Hence, these operators are largely ignored in the literature.}. A t-channel resolution of the D1 operator is also included for completeness. These models are constrained using publically available and model-independent limits on the visible cross-section for new physics processes originating from mono-X\footnote{Where X is a W/Z boson or a parton.} + missing transverse energy ($E_{T}^{miss}$) searches conducted by the ATLAS Collaboration.  \textcolor{blue}{The re-interpretation of theses results is a relatively simple task, requiring only the simulation of the signal associated with the new model for each point in parameter space (see Section \ref{sec:sec3}).}

\hspace{1cm} \textcolor{red}{Should I say something here about the specifics of our limits? E.g. Something like "Limits are applied to the cross-section for the process $pp \rightarrow X\chi\bar{\chi}$ and to $g_{q} = g_{DM} = f$"?}

\hspace{1cm} The remainder of the paper is organised as follows. Section \ref{sec:sec2} contains a compendium of the simplified models chosen for analysis. Section \ref{sec:sec3} outlines the technique used to convert mono-X + $E_{T}^{miss}$ limits on the visible cross-section for any new physics process into constrains on simplified models. \textcolor{blue}{Lastly, in Section \ref{sec:sec4} we present and discuss our results.}
\bigskip

\textcolor{magenta}{Include:}
\begin{enumerate}
\item \textcolor{magenta}{Motivation for SiMs: limited validity of EFTs \& uneconomically broad parameter space of full UV-complete models/insensitivity of model-dependent results to the wider class of DM models $\rightarrow$ advantage of simplified models (relatively small parameter space, predictions accurate and complementary across a broad energy range $\mathcal{O}$(GeV-TeV)).\footnote{\textcolor{red}{Are we assuming that the reader understands the basics of a "simplified model"? Or should we have one or two sentences that summarise the differences between, say, an EFT and a SiM?}}}
\item \textcolor{magenta}{Brief description of the aim of the analysis (e.g. something like "to apply constraints to $m_{\chi}$, $m_{M}$ and $g_{q} = g_{\chi} = f$ for four phenomenologically distinct simplified models, etc) along with a brief motivation for the choice of simplified models\footnote{\textcolor{red}{Or should this part be in section \ref{sec:sec2}?}}.}
\item \textcolor{magenta}{Summary of content of paper (eg. "This paper is organised as follows: section 2 is a compendium of the simplified models chosen for analysis..." etc)}
\end{enumerate}
\end{flushleft}

\section{Models} 
\label{sec:sec2}
\begin{flushleft}
\fg{We begin with a short set of assumptions: that the DM particle, $\chi$, is a Dirac fermion, that it is a singlet under the SM, and that it is the lightest stable new particle. We also require minimal flavour violation (MVF) to hold wherever relevant. Each model is built around a scenario whereby $\chi$ and SM quarks are coupled via a mediator. Coupling to SM leptons \cite{} or gluons \cite{} is beyond the scope of this paper, but these cases have been studied elsewhere.} Resolving the contact interaction of an EFT \fg{at tree-level} leads to two possibilities: the mediating particle is exchanged in the $s$-channel, in which case it \st{may be colour neutral} \fg{is also a SM singlet}, or it is exchanged in the $t$-channel, in which case it is necessarily \fg{charged and} coloured \st{to conserve colour flow between the quarks}. With comprehensiveness in mind, three \comm{(Should this be four? What about pseudo-scalar? - Amelia)} $s$-channel models and one $t$-channel model were chosen for analysis. The $s$-channel models are characterised by vector (sV), axial-vector (sA) and scalar (sS) couplings to both the dark and SM sectors. In the notation of Ref.~\cite{DMCons2}, these correspond to the D5, D8 and D1 operators\footnote{\comm{Moved this footnote to earlier section - first reference to DX operators - Amelia.}} respectively in the EFT regime ($\sqrt{\hat{s}} \ll M$). \comm{I've changed  AV to A, to avoid confusion with $\bar\chi \gamma^\mu (1-\gamma^5)\chi$ type models. Is that ok? - Tom.} These models are described by the following interaction Lagrangians \comm{(Two suggestions here - either just a sum over i (remove the i = 1 and 6, these should be evident), or sum over q - Amelia)}:
\begin{equation}
\label{L_int_sV}
\mathcal{L}_{sV} = - \xi_{\mu}\left[ \sum\limits_{i=1}^{6} g_{q_i}\bar{q_i}\gamma^{\mu}q_i - g_{\chi}\bar{\chi}\gamma^{\mu}\chi\right],
\end{equation}
\begin{equation}
\label{L_int_sA}
\mathcal{L}_{sA} =  \xi_{\mu}\left[\sum\limits_{i=1}^{6} g_{q_i}\bar{q_i}\gamma^{\mu}\gamma_{5}q_i - g_{\chi}\bar{\chi}\gamma^{\mu}\gamma_{5}\chi\right],
\end{equation}
\begin{equation}
\label{L_int_sS}
\mathcal{L}_{sS} = \sum\limits_{i=1}^{6} g_{q_i}\bar{q_i}q_i\eta + g_{\chi}\bar{\chi}\chi\eta,
\end{equation}
where $i$ represents an index running over quark flavour, $\xi$ is the (axial-)vector mediator and $\eta$ is the scalar mediator, and $\chi$ is the dark matter particle - a weakly interacting SM singlet Dirac fermion with mass $m_{\chi}$. \question{The scalar model should really have a Yukawa coupling (ie $g_q \rightarrow g_q m_q / v$) if we want to be consistent with D1 and the ATLAS/CMS run2 simplified models. Is it too late to implement this? - Tom.} \comm{I agree, and this should be easy enough to implement I think. - Amelia} \st{For simplicity, it is assumed that both $\xi$ and $\eta$ are SM singlet particles and couple only to pair-antipair combinations of $\chi$ and the SM quarks, with coupling strengths $g_{\chi}$ and $g_q$ respectively.} For the couplings \fg{$g_q$ and $g_{\chi}$} to remain within the perturbative regime, \fg{they must satisy $g_{q},g_{\chi} \leq 4\pi$}, \draft{though stronger perturbativity requirements can exist} \cite{ValidEFT}. \textcolor{blue}{\st{Finally, for convenience, the mediator is assumed to couple to }$\cancel{\chi}$\st{ with the same strength as to each of the six quarks, that is, }$\cancel{g_{q} = g_{\chi} \equiv f}$.} \comm{Removed, since we want to include cases where this isn't true! - Amelia} Note that the vector and axial-vector models have already been studied extensively \cite{Buchmueller:2014yoa, Chatrchyan:2013qha, Aad:2012hf, Harris:2014hga} and so serve as a good basis for the comparison of constraints. \comm{I need to add some citations for the scalar model here - Tom}

\hspace{1cm}The last model considered in this paper - \fg{with a scalar mediator exchanged in} the $t$-channel \st{UV-completion of the D1 operator} (denoted tS) - juxtaposes nicely with the $s$-channel models and is motivated separately by analogy with a common aspect of Supersymmetric models: neutralino dark matter interacting with the SM sector via $t$-channel exchange of a squark\footnote{Note that in the Supersymmetric scenario the dark matter particle is a Majorana fermion. Simplified models in which the dark matter particle is a Majorana fermion are not covered here as they are \st{not expected to produce significantly different results} \fg{kinematically identical to the corresponding Dirac case, and only require multiplication of the cross section by a simple factor to calculate limits. The exception is} the case of the $s$-channel vector mediator model, which vanishes if $\chi$ is a Majorana fermion \cite{METSig}.} \cite{SUSYDM}. \fg{In the heavy-mediator limit, under a Fierz transform it approaches a combination of the D5 and D8 EFT operators.}

In this model, the mediator which we call $\eta'$ \question{(Is there a better symbol we can use? What about $\phi$? - Amelia)} must necessarily have colour charge, and can couple to the either the left or right-handed quarks as a SU(2) doublet or singlet respectively. Since the LHC is insensitive to the chirality of the quarks, for simplicity we assume that the mediator couples to the left-handed quarks only, that the masses and couplings of $\eta'$ are equal across the three generations, and that the masses of the two components $\eta'$ are equal. %
%
The interaction Lagrangian for this model is then 
\begin{equation}
\label{L_int_tS}
\mathcal{L}_{int} = \sum_{i=1}^3 g_{Q_i} \bar{Q_i} P_R \eta' \chi + {\rm h.c.},
\end{equation}
\fg{where $i$ runs over quark families,} the quark doublets are represented by $Q = \{ (u,d), \, (c, s), \, (t, b) \}$, and $P_R$ is the usual chiral projection operator. \question{Have you included the projection operator in the code? - Tom}

\comm{If we used $i$ before, it might be better to use $j$ here now, since they are sums over different things. Is it clearer to use $Q_L$ here, or is that implied enough? Also, introducing $g_{Q_i}$ is a bit awkward, though I'm not sure what is best to use - what about $g_{q\chi}$? - Amelia}

\hspace{1cm}\textcolor{blue}{\st{When constraining the above models using the collider production cross-section for the process }$\cancel{pp \rightarrow X + \bar{\chi}\chi}$\st{ it is also important to consider the impact of the mediator decay width, }$\cancel{\Gamma}$ \cite{}\st{. The reason being that }$\cancel{\sigma\left(pp \rightarrow X + \bar{\chi}\chi\right)}$\st{ scales as:}}
\begin{equation}
\label{cross_section}
\cancel{\frac{f^{4}E^{2}}{(q^{2} - M^{2})^{2} + M^2 \Gamma^{2}}}
\end{equation}
\st{where }$E$\st{ is the centre-of-mass energy of two colliding partons and }$q$\st{ is the four-momentum transferred by the mediator.} \comm{Careful with formulae from other papers, it's caused me enormous headaches in the past when they were wrong :) The formulae we quoted here from \cite{METSig} didn't even have the consistent dimensions! Somehow they dropped the wrong term when squaring the propagator! - Tom.} 

\fg{An important factor when considering simplified models is to ensure the mediator width is treated appropriately, as it impacts the cross section calculation and, in some cases, the kinematic behaviour of the model.} In previous analyses it has been customary to consider mediators of fixed width ranging from $\Gamma = M/8\pi$ to $\Gamma = M/3$ \cite{METSig, Fox:2012ee} \footnote{\textcolor{magenta}{Also add a reference for the recent monojet paper.}}. This approach is motivated by the observation that, where the mediator is exchanged in the $s$-channel and produced on-shell, $\sigma\left(pp \rightarrow X + \bar{\chi}\chi\right)$ is maximally enhanced when $\Gamma$ is small, as expected from equation \ref{cross_section}\footnote{For a more in depth discussion see Ref. \cite{METSig}.}. The smallest width, $\Gamma = M/8\pi$, corresponds to a mediator which couples  to only one helicity and flavour of quark with $g_{q} = 1$ \cite{METSig}. In this paper, the mediator widths are expanded to include coupling to all kinematically accessible quark flavours \st{(in addition to the single flavour case). This move is based on the \emph{a priori} assumption that the mediators couple equally to each quark and so should be allowed to decay into quark pairs of any flavour. For the above models, the full decay widths are} \fg{In this work, we allow flavour-blind coupling (when not restricted by MFV), so our definition of the minimum width must change to reflect this. Following \cite{} \comm{(other minimum width papers)}, the minimum width for each model is given by}:
\begin{equation}
\label{gamma_sV}
\Gamma_{sV} = \frac{\fg{g_{\chi}^2} M}{12\pi}\left(1 + \frac{2m_{\chi}^{2}}{M^{2}}\right)\left(1 - \frac{4m_{\chi}^{2}}{M^{2}}\right)^{1/2} + \sum_{\substack{q}}\frac{\fg{g_q^2}M}{4\pi}\left(1 + \frac{2m_{q}^{2}}{M^{2}}\right)\left(1 - \frac{4m_{q}^{2}}{M^{2}}\right)^{1/2}
\end{equation}
\begin{equation}
\label{gamma_sA}
\Gamma_{sA} = \frac{\fg{g_{\chi}^2} M}{12\pi}\left(1 - \frac{4m_{\chi}^{2}}{M^{2}}\right)^{3/2} + \sum_{\substack{q}}\frac{\fg{g_q^2} M}{4\pi}\left(1 - \frac{4m_{q}^{2}}{M^{2}}\right)^{3/2}
\end{equation}
\begin{equation}
\label{gamma_sS}
\Gamma_{sS} = \frac{\fg{g_{\chi}^2} M}{8\pi}\left(1 - \frac{4m_{\chi}^{2}}{M^{2}}\right)^{3/2} + \sum_{\substack{q}}\frac{3\fg{g_q^2} M}{8\pi}\left(1 - \frac{4m_{q}^{2}}{M^{2}}\right)^{3/2}
\end{equation}
\begin{equation}
\label{gamma_tS}
\Gamma_{tS} = \sum_{\substack{q}} \frac{\fg{g_{q\chi}^2}M}{16\pi}\left(1 - \frac{m_{q}^{2}}{M^{2}} - \frac{m_{\chi}^{2}}{M^{2}}\right)\sqrt{\left(1 - \frac{m_{q}^{2}}{M^{2}} + \frac{m_{\chi}^{2}}{M^{2}}\right)^{2} - 4\frac{m_{\chi}^{2}}{M^{2}}}
\end{equation}
where $m_{\chi} \leq M/2$ is required for the $s$-channel models and $M \geq m_{\chi} + m_{q}$ is required for the $t$-channel model. Note also that the summation over $q$ is restricted to quarks with $m_{q} \leq M/2$ for all models. \comm{We need to extend this discussion; See e.g. my paper with Karl, 1502.05721; The width expression above is only valid for $\Gamma < M$, which doesn't always hold true for Run 1 data - Tom.}

\comm{I agree - I might try to implement something where we don't test a point if the minimum width is larger than the mediator mass, or at least reduce it by scaling down the coupling, so that we always avoid this issue. - Amelia}

%\textcolor{magenta}{This section should include:}
%\begin{enumerate}
%\item \textcolor{magenta}{Brief motivation for choice of simplified models? (eg. something like "we consider the most straightforward UV-completions of the D1, D5 and D8 effective operators, corresponding to the s-channel scalar, vector and axial-vector models respectively.}
%\item \textcolor{magenta}{The interaction Lagrangians for our four SiMs along with an explanation for why we only assume coupling to SM quarks.}
%\item \textcolor{magenta}{The assumptions and the decay widths associated with our models (?).}
%\item \textcolor{magenta}{Comments on the requirement that $\sqrt{g_{q}g_{\chi}} \leq 4\pi$ in order for the theory to remain perturbative? $\rightarrow$ comments on the choice of mass and coupling points used? Or does this belong in section \ref{sec:sec3}?}
%\end{enumerate}
\end{flushleft}

%Resolving the mediator leads to two possibilities: the mediating particle is exchanged in the s-channel, in which case it may be colour neutral, or it is exchanged in the t-channel in which case it is necessarily coloured \cite{}. 

%\cite{ValidEFT, BeyondEFT, CSUSY} t-channel \cite{Buchmueller:2014yoa, SiM}.


\section{(Something like) Reinterpretation of Mono-X Contraints} 
\label{sec:sec3}
\begin{flushleft}
The mono-X + $E_{T}^{miss}$ signal is a popular signal in the search for new physics. It is predicted by an assorted collection of new physics models, including unparticle physics models \cite{CMS-PAS-EXO-12-048} and the Arkani-Hamed, Dimopoulos, and Dvali (ADD) model of large extra spatial dimensions \cite{ATLAS-CONF-2012-147}. For the $s$-channel simplified models discussed in Section \ref{sec:sec2} the SM particle, X, originates from one of a pair of intial-state quarks (shown in Figure \ref{Signal_phen_a} and \ref{Signal_phen_b}). The case where X is radiated from the mediator - a process known as virtual internal Bremsstrahlung - is only possible if the SM-dark matter interaction occurs via the $t$-channel (as shown in Figure \ref{Signal_phen_c}). In both cases, the parton, W and Z bosons are the SM particles most likely to be emitted \cite{Kumar:2013iva}. Hence, this paper will focus on three detection channels: monojet, mono-Z and mono-W.

\begin{figure}[ht!]
\begin{subfigure}[b]{0.4\textwidth}
\centering
\begin{tikzpicture}
\draw[fermion] (-1.5,1.5)node[left]{$q$} --(-0.75,0.75);
\draw[dashed] (-0.75,0.75) -- (0,1.5)node[right]{X};
\draw[fermion] (-0.75,0.75) -- (0,0);
\draw[fermion] (-1.5,-1.5)node[left]{$\bar{q}$} --(0,0);
\draw[fill] (0,0) circle [radius=0.05]node[left]{$g_{q}\mbox{ }$};
\draw[photon] (0,0) --node[above]{$\xi$} (2,0);
\draw[fermion] (2,0) -- (3.5,1.5)node[right]{$\chi$};
\draw[fermion] (2,0) --(3.5,-1.5)node[right]{$\bar{\chi}$};
\draw[fill] (2,0) circle [radius=0.05]node[right]{$\mbox{ }g_{\chi}$};
\end{tikzpicture}
\caption{}
\label{Signal_phen_a}
\end{subfigure} \hspace{0.2cm}
\begin{subfigure}[b]{0.5\textwidth}
\centering
\begin{tikzpicture}
\draw[fermion] (-1.5,1.5)node[left]{$q$} --(-0.75,0.75);
\draw[dashed] (-0.75,0.75) -- (0,1.5)node[right]{X};
\draw[fermion] (-0.75,0.75) -- (0,0);
\draw[fermion] (-1.5,-1.5)node[left]{$\bar{q}$} --(0,0);
\draw[fill] (0,0) circle [radius=0.05]node[left]{$g_{q}\mbox{ }$};
\draw[dashed] (0,0) --node[above]{$\eta$} (2,0);
\draw[fermion] (2,0) -- (3.5,1.5)node[right]{$\chi$};
\draw[fermion] (2,0) --(3.5,-1.5)node[right]{$\bar{\chi}$};
\draw[fill] (2,0) circle [radius=0.05]node[right]{$\mbox{ }g_{\chi}$};
\end{tikzpicture}
\caption{}
\label{Signal_phen_b}
\end{subfigure}\vspace{0.2cm}
\centering
\begin{subfigure}[b]{0.5\textwidth}
\centering
\begin{tikzpicture}
\draw[fermion] (-2,1.5)node[left]{$q$} --(0,1.5);
\draw[fermion] (-2,-1)node[left]{$\bar{q}$} --(0,-1);
\draw[fill] (0,1.5) circle [radius=0.05]node[above]{$g_{q}$};
\draw[photon] (0,1.5) -- node[left]{$\eta'$}(0,0.25);
\draw[dashed] (0,0.25) -- (2, 0.25)node[right]{X};
\draw[photon] (0,0.25) -- node[left]{$\eta'$}(0,-1);
\draw[fermion] (0,1.5) -- (2,1.5)node[right]{$\chi$};
\draw[fermion] (0,-1) --(2,-1)node[right]{$\bar{\chi}$};
\draw[fill] (0,-1) circle [radius=0.05]node[below]{$g_{\chi}$};
\end{tikzpicture}
\caption{}
\label{Signal_phen_c}
\end{subfigure}
\caption{Dark matter pair-production processes with a SM particle, X, in the final state for the $s$-channel (a) vector/axial-vector mediator and (b) scalar mediator simplified models and for (c) the $t$-channel scalar mediator model. \textcolor{red}{Do we need to include this?}}
\end{figure}

\hspace{1cm}The monojet contraints are derived from a Supersymmetry search conducted by the ATLAS Collaboration (see Section \ref{monojet_constraints}). Similarly, the mono-Z constraints are derived from a dark matter search originally optimised for the D1, D5 and D9 effective operators (see Section \ref{monoZ_constraints}). \textcolor{magenta}{Lastly, the mono-W constraints are derived from...}
\bigskip

\textcolor{magenta}{Do we need something here like "The model-independent results for each of the above analyses are reintepreted as simplified model constraints following a simple methodology. Firstly..."? Or is it ok to assume that the reader understands the process employed in converting model-independent limits into limits on a specific model? Do we need to introduce them to $\sigma = N/\mathcal{L}\times\epsilon\times\mathcal{A}$?}
\end{flushleft}

\subsection{Signal Generation}
\begin{flushleft}
For each channel, signal production was modelled with \MG5$\_$aMC 2.2.2 (with 0, 1 and 2 jets included in the matrix element calculation for the monojet channel). Showering and hadronisation was then performed by \PYTHIA  8.201 using the \textcolor{magenta}{ATLAS UE Tune AU2-MSTW2008LO}. Again, for the monojet channel, matching between \MG and \PYTHIA was done using the MLM scheme with a matching scale of $m_{\chi}/4$ (mirroring the Supersymmetry search value of $m_{\tilde{t}}/4$). Using the MSTW2008lo68 PDF set\footnote{This PDF set is currently the benchmark for ATLAS analyses \cite{}.} and the default \MG factorization and renormalization scales\footnote{In \FNMG5 2.2.2 the default scale, $\mu$ is defined as the sum of $\sqrt{m_{\chi}^{2} + p_{T}^{2}}$ for all dark matter particles in the final state.}, samples were produced for a representative set of of dark matter and mediator masses, shown in Table \ref{Mass_coup_points}. \textcolor{magenta}{Something about the coupling strength, $f$, being set to a value of 1?} Note that rows one to three of Table \ref{Mass_coup_points} cover the case where the dark matter mass is small and the mediator has either a mass that is degenerate, low or sufficiently large so as to be in the EFT regime of validity\footnote{A recent study by Alves et al. found that EFT results do not apply to mediators with a mass less than 2.5 TeV at the LHC during Run I \cite{Alves:2011wf}.}. Rows four and five cover the case where the dark matter particle has a medium mass and the mediator is either near-degenerate or larger. Rows six and seven (eight and nine) cover the case where the dark matter particle is a few GeV (TeV) and the mediator is either near-degenerate or again sufficiently large so as to be in the EFT regime of validity.

\begin{table}[!htbp]
\centering
\begin{tabular}{l|l|l}
\hline
\hline
 \rule{0pt}{2.2ex}Column & $m_{\chi}$ [GeV] & $M$ [GeV]\\
\hline
 \rule{0pt}{2.2ex}1 & 10 & 10\\
 2 & 10 & 200\\
 3 & 10 & 25000\\
 4 & 100 & 200\\
 5 & 100 & 2000\\
 6 & 500 & 600\\
 7 & 500 & 25000\\
 8 & 1000 & 2000\\
 9 & 1000 & 25000\\
 \hline
 \hline
\end{tabular}
\caption{Mass points chosen for the analysis of simplified dark matter models. The masspoints are primarily representative of three regimes: (near-)degenerate ($M\approx m_{\chi}$), kinematically allowed ($M \geq 2m_{\chi}$), and EFT-like ($\sqrt{\hat{s}} << M$).}
\label{Mass_coup_points}
\end{table}
\end{flushleft}

\subsection{Signal Systematics}
\begin{flushleft}
For each channel, there are three key sources of systematic uncertainty: the factoriation and renormalisation scale, the strong coupling constant ($\alpha_{s}$) and the PDF. The last two sources are difficult to separate, however, so they are examined in tandem. \textcolor{magenta}{Should we say something here about using only leading order predictions? For example, should we comment on the uncertainty associated with not including NLO corrections to the cross-section?} %It is important to note that a complete examination of the uncertainties associated
%with each simplified model and each mass and coupling combination was not necessary.

\hspace{1cm}The uncertainties associated with the aforementioned parameters were estimated in the usual way. Firstly, the factorisation and renormalisation scales were varied simultaneously by a factor of 0.5 and a factor of 2 times the default scale. The uncertainty on the final acceptance, $\mathcal{A}$ was then taken to be the average change in $\mathcal{A}$ resulting from these up and down variations. \textcolor{magenta}{Similarly, the uncertainty on $\mathcal{A}$ associated with $\alpha_{s}$ and the PDF was estimated by...}

\hspace{1cm}\textcolor{magenta}{For the monojet channel the matching scale introduces an additional source of uncertainty, which is estimated by...}
\bigskip

THIS IS WHERE I'M UP TO!
\end{flushleft}

\subsection{Monojet Constraints}
\label{monojet_constraints}
\begin{flushleft}
A monojet-like signal is predicted by R-parity conserving Supersymmetry models for the process $pp\rightarrow j + \tilde{t}(\rightarrow c+\tilde{\chi}^{0})\tilde{t}(\rightarrow c+\tilde{\chi}^{0})$ where $\tilde{t}$ is the top squark (stop), $\tilde{\chi}^{0}$ is the lightest neutralino and $j$ is a final-state jet. In this scenario, $\tilde{\chi}^{0}$ is assumed to be the lightest supersymmetric particle. The recent search for this process conducted by the ATLAS Supersymmetry group \cite{SUSY_official_paper} assumes that the stop and LSP masses are nearly degenerate. Jets from the charm-quark fragmentation are then too soft to be detected or identified. In this case, the stops decay essentially invisibly and the final state is trivially comparable to that of the dark matter pair production process $pp\rightarrow j + \chi\bar{\chi}$. The two processes are of course topologically dissimilar but since the Supersymmetry analysis simply looks for an initial-state jet recoiling against missing energy it seemed possible that the results of the Supersymmetry analysis optimised for a $\tilde{t}$ (and $\tilde{\chi}^{0}$) search might be directly transferable to $\chi$. This was confirmed via a two-fold check of the distributions of key kinematic variables - namely $E_{T}^{miss}$, the jet multiplicity and the $p_{T}$ and $\eta$ of the leading jet - associated with both the dark matter and Supersymmetry signal processes.
\bigskip

\textcolor{magenta}{Discussion of the cutflow and background estimation/uncertainties.}

THIS IS WHERE I'M UP TO!

\end{flushleft}
%\subsection*{MC for monojet}
%\begin{flushleft}
%For the SUSY monojet-like analysis events are selected with the goal to gain sensitivity in the region of parameter space where $\tilde{t}$ and $\tilde{\chi}^{0}$ are nearly degenerate in mass. With this in mind, event selection is divided into two categories: preselection and signal selection. For the first catgeory, events are required to have a reconstructed primary vertex (interaction point) with at least five tracks attached to it. This vertex must align with the nominal (designed) interaction point and should there be more than one primary vertex, the vertex with the highest summed $p_{T}^{2}$ of the attached tracks is selected. This is the normal procedure for selecting hard-scatter interactions.
%\end{flushleft}
%\begin{flushleft}
%Next events are required to have $E_{T}^{miss}>150\,\mbox{GeV}$ in accordance with the performance of the $E_{T}^{miss}$ trigger, which is close to 100\% efficient for $E_{T}^{miss}>150\,\mbox{GeV}$ \cite{Aad:1363019}. In addition to this, events must have a minimum of one jet with $p_{T}>150\,\mbox{GeV}$ and $|\eta|<2.8$. This ensures that candidate jets originate from a hard-scattering process which has the correct jet + $E_{T}^{miss}$ topology. These requirements also omit much of the QCD multijet background.
%\end{flushleft}
%\begin{flushleft}
%In order to distinguish hard-scatter jets from fake jets and non-collision jets, 'jet quality' requirements are imposed. First an event is vetoed if it contains any jet with $p_{T}>20\,\mbox{GeV}$ and $|\eta|<4.5$ that presents an anomalous charged fraction defined by:
%\begin{equation}
%f_{ch} = \frac{\Sigma p_{T}^{track, jet}}{p_{T}^{jet}}
%\end{equation}
%where $\Sigma p_{T}^{track, jet}$ is the scalar sum of the transverse momenta
%of tracks associated with the primary vertex within a cone of radius $\Delta R = 0.4$ around the jet axis. Additionally, if any jet has an electromagnetic fraction or timing inconsistent with a proton-proton origin then the event is vetoed. These latter requirements ensure that noise produced by the calorimeters, which might potentially fake a jet, is omitted.
%\end{flushleft}
%\begin{flushleft}
%Finally, events are vetoed if they contain at least one identified electron with $p_{T}>20\,\mbox{GeV}$ or one muon with $p_{T}>10\,\mbox{GeV}$. These cuts remove non-signal $E_{T}^{miss}$ events with leptons in the signal region, for example W ($\rightarrow\Pe/\Pmu\Pneutrino$) + jets events and semileptonic top decays. 
%\end{flushleft}
%\begin{flushleft}
%For the signal selection, a maximum of three \textit{good} jets with $p_{T}>30\,\mbox{GeV}$ and $|\eta|<2.8$ are allowed\footnote{It is this particular cut that permits us to ignore the differences between the $\eta$ distributions of the SUSY and simplified models. Requiring a jet with $p_{T}>30\,\mbox{GeV}$ to fall within the range -2.8 $<$ $\eta$ $<$ 2.8 is designed to omit background jets whilst preserving a maximal number of signal jets. Looking at the top left plot in Figure \ref{SUSY_DM_kinematics} we see that, regardless of the difference in shape, a maximal number of signal jets sit within $|\eta|<2.8$ for both classes of models. Specifically, 93.73\% of signal jets are within $|\eta|<2.8$ for the SUSY model compared with $\sim$98\% for the simplified models.}. This cut comes from examination of the jet mulitplicity associated with the radiated parton in the stop pair-production process (see Figure \ref{SUSY_kinematics}), which peaks at about 1. This cut is also designed to preserve any potentially detectable jets resulting from the decay of the charm quark. To reduce the multijet background when $E_{T}^{miss}$ is constructed from jet(s) with mis-measured $p_{T}$, each good jet must be azimuthally separated from $E_{T}^{miss}$ by no less than 0.4. Note that for the stop pair-production process, the parton and the two stop-quarks are emitted back-to-back. When the parton fragments into two or more jets however, the jet with the higher $p_{T}$ is not expected to satisfy $\Delta\phi(jet,\,E_{T}^{miss})=\pi$. Since the parton is likely to be boosted, requiring the lead jet to satisfy $\Delta\phi(jet,\,E_{T}^{miss})>0.4$ is reasonable. Lastly, three separate signal regions are defined with increasing lower thresholds on the leading jet $p_{T}$ and $E_{T}^{miss}$. These regions - denoted M1, M2 and M3 - result from an optimization performed across the stop-neutralino mass plane with increasing $m_{\tilde{t}}$ and $m_{\tilde{\chi}^{0}}$ as discussed in Ref. \cite{Abdallah:1636856}. For convenience, the event selection criteria are summarised in Table \ref{Cutflow}.
%
%\begin{table}[!htbp]
%\centering
%\begin{tabular}{l|c|c|c}
% \hline
% \hline
% \multicolumn{4}{c}{Preselection}\\
% \hline
% \multicolumn{4}{l}{Primary vertex}\\
% \multicolumn{1}{l}{$E_{T}^{miss} > 150\,\mbox{GeV}$}\\
% \multicolumn{1}{l}{At least one jet with $p_{T}>150\,\mbox{GeV}$ and $|\eta|<2.8$}\\
% \multicolumn{1}{l}{Jet quality requirements}\\
% \multicolumn{1}{l}{Lepton veto}\\
% \hline
% \multicolumn{4}{c}{Monojet-like selection}\\
% \hline
% \multicolumn{1}{l}{At most three jets wth $p_{T}>30\,\mbox{GeV}$ and $|\eta|<2.8$}\\
% \multicolumn{1}{l}{$\Delta\phi(jet\,E_{T}^{miss})>0.4$}\\
% \hline
% Signal Region: & M1 & M2 & M3\\
% \hline
% Minimum leading jet $p_{T}$ [GeV] & 280 & 340 & 450\\
% Minimum $E_{T}^{miss}$ [GeV] & 220 & 340 & 450\\
% \hline
% \hline
%\end{tabular}
%\caption{Event selection criteria for the monojet-like analysis. Adapted from Ref. \cite{SUSY_official_paper}.}
%\label{Cutflow}
%\end{table}
%
%It is important to note that the control regions used in the SUSY monojet-like analysis also abide the preselection criteria discussed above but sans the lepton veto. The idea is that leptons are now used to identify and select background events in order to assess background contributions in the signal regions. Specifically, muon selection is used to constrain events from background processes involving a muon (W$\rightarrow \Pmu\Pneutrino$, Z$\rightarrow \Pmuon\APmuon$) or neutrino (Z$\rightarrow \Pneutrino\APneutrino$). In the same way, electrons are used to constrain background contributions from processes involving electrons or hadronic taus.
%\end{flushleft}

%Note that we are interested in the behaviour of only high pT jets so we
%remove jets with pT < 30 GeV. 

\subsection{Mono-Z Constraints}
\label{monoZ_constraints}
\begin{flushleft}
Note to Amelia: Here you should discuss the original intention of the mono-Z analysis. Also comment on any validation you did in order to confirm that you could use the results of the analysis. Lastly, discuss or list the cuts used in the analysis and the uncertainties associated with the results. This will inevitably include details of the background estimation and the detector performance.
\end{flushleft}

\subsection{Mono-W Constraints}
\begin{flushleft}
Note to Johanna: Here you should discuss the original intention of the mono-W analysis. Also comment on any validation you did in order to confirm that you could use the results of the analysis. Lastly, discuss or list the cuts used in the analysis and the uncertainties associated with the results. This will inevitably include details of the background estimation and the detector performance.
\end{flushleft}

\begin{flushleft}
\textcolor{magenta}{This section should include:}
\begin{enumerate}
\item \textcolor{magenta}{A description of the general process employed to reinterpret monojet, mono-W and mono-Z limits. This should cover MC signal generation, the cuts used in each mono-X channel and validation of the procedure(s) used to generate the results.}
\item \textcolor{magenta}{Monojet specifics: motivation for why the SUSY results were used instead of the monojet+MET results.}
\item \textcolor{magenta}{A description of the assessment of the systematics.}
\item \textcolor{magenta}{The limit setting strategy (?).}
\end{enumerate}
\end{flushleft}

\section{Results} 
\label{sec:sec4}
\subsection{Limits on the coupling $\sqrt{g_{q}g_{\chi}}$}
The 95\% confidence level upper limits on the $sV$ and $sA$ model coupling combination $\sqrtgqgX$, and the $tS$ model coupling $\gqX$, obtained from each of the \monoX channels, are presented in figs.~\ref{fig:results_sVsA_rat05}-\ref{fig:results_tS}. These quantities were evaluated as described in appendix \ref{Appendix_limitsetting}, including statistical and systematic uncertainties, and correspond to the best limits of each signal region tested.

In each plot, limits are shown randing from $<$0.01 to the upper perturbative limit\footnote{We note that perturbativity is questionable for couplings greater than $\sim 2-3$, however $4\pi$ is an oft-quoted hard upper limit.} for each coupling, $4\pi$; where a limit was calculated to be larger than this, that limit is considered meaningless and the region is coloured grey. The white (hatched) regions coincide with those mass points which yield an initial (final) value of $\sqrtgqgX$ or $\gqX$ which fails to satisfy our requirement that $\Gamma < \Mmed / 2$. (We observe that values for which the width is just within our upper validity bound of $\Mmed/2$ may be pushed over into the invalid range with the addition of new particles, not considered here, which would serve to increase the mediator width.) When $\gX / \gq$ = 0.2, only the \monojet channel produces limits which survive this requirement, and so these are shown separately in fig. \ref{fig:results_sVsA_rat02}.

Detailed comments specific to each channel are below, however some trends are channel-independent. For the $sV$ model, strong limits exist when $\Mmed > 2 \mX$ as the mediator can go on-shell, thereby enhancing the cross-section. The $sA$ model limits show similar behaviour to the $sV$ model except in the region where $\mX \gtrsim 10 \times \Mmed$, where the cross-section is greatly enhanced \comm{WHY?}. The upper limit on $\sqrtgqgX$ is relatively constant across values of $\gX / \gq$, as is expected when the coupling (and hence the width) has been demonstrated to have little effect on kinematic behaviour, and using the assumptions of eq.~\ref{eq:sigma_propto_couplings_schan}. As the ratio increases, points in the region $\Mmed > \mX$ disappear as the initial condition, $\gq = 1$, leads to failure of the width condition. However, one could easily have chosen a smaller initial value of $\gq$ to recover these points, and we suggest that the limits in this region would be quite similar to those seen in the $\gX / \gq$ = 0.2 and 0.5 cases.

\comm{The results are limited when $\mX$ or $\Mmed$ is large due to the suppressed cross-sections in these cases, and are expected to improve at higher centre-of-mass energies. At small DM masses with an off-shell mediator, the $\met$ distribution is softer, so results in this region of phase space are limited by statistical uncertainties from being in the tail of the distribution; this region would benefit from further optimisation of event selection in analyses aimed at SiM study, as we expect to see from the upcoming Run II results.}

The results are discussed according to channel below.

\subsubsection{Mono-jet channel}

The upper limits on the coupling combination $\sqrtgqgX$ of the $sV$ and $sA$ models, obtained in the \monojet channel, are displayed in the left-hand column of figs.~\ref{fig:results_sVsA_rat05}-\ref{fig:results_sVsA_rat5}, for $\gX/\gq =$ 0.5, 1, 2 and 5 respectively. Additional results for the $\gX/\gq =$ 0.2 case are also shown separately in fig.~\ref{fig:results_sVsA_rat02}, as these limits are only meaningful within this channel.

As expected, the \monojet channel produces the strongest coupling limits in both $s$-channel models, which are better than those from the next-best \monoZ channel by a factor of a few. For the $sV$ model, the weakest limits result for large $\mX$ or large $\Mmed$, and in fact are so weak that they are pushed into the region of invalidity where $\Gamma > \Mmed/2$; this is because although the acceptance is considerably higher in these regions compared to low masses, the cross section is sufficiently small that this effect prevails. Within the valid region ($\mX \in [1, 100]$ GeV and $\Mmed \in [1, 200]$ GeV), the limit on $\sqrtgqgX$ generally ranges from 0.1 to 0.7, with a handful of on-shell masses reaching a limit of $\sim$0.05 in the large $\gX / \gq$ case. In the large $\gX/\gq$ scenario, limits for $\mX = 1000$ GeV start to become valid. This is because if $\sqrtgqgX$ remains constant but the $\gX / \gq$ increases then the value of $\gq$ is pushed downward and so the width, which is dominated by decays to SM particles, decreases.

The $sA$ model displays similar behaviour to the $sV$ model except for the $\mX \gtrsim 10 \times \Mmed$ region, as described above.

The uncertainties on the limits in the $sV$ model generally range from \comm{??}\% to \comm{??}\%, and are dominated by \comm{X}. The uncertainties for the $sA$ model are dominated by \comm{X} and are within the range $X \sim Y$\%.

\subsubsection{Mono-$Z$(lep) channel}

The simplicity of the \monoZ channel relative to the \monojet channel, and the speed of its production within \MG, allowed us to study a finer granularity of points in the mass phase space. The resulting limits on the sV and sA models are shown in the central column of figs.~\ref{fig:results_sVsA_rat05}-\ref{fig:results_sVsA_rat5}. The behaviour of the limits as $\gX / \gq$ varies is similar to that within the \monojet channel, and overall the limits are weaker compared to that channel by a factor of a few.

The total relative uncertainties on $\sqrtgqgX$ in the $s$-channel models are generally within 10\%, but can range up to 80\% in a few cases; they are dominated by \comm{X}.

The advantage of the mono-boson channels is in the study of the $tS$ model; since this was not included in the \monojet channel the strongest limits are obtained with the \monoZ analysis, and are shown in the left-hand side of fig.~\ref{fig:results_tS}. Note that the scale here is increased by a factor of 10 from the $s$-channel model limits, in general the $tS$ model limits are weaker than the corresponding $s$-channel points, thanks to an orders-of-magnitude weaker cross-section and the inability of the mediator to go on-shell in this channel. We find stronger limits for smaller $\mX$ and $\Mmed$ masses, where larger cross sections compensate for lower acceptances at these points.

\subsubsection{Mono-$W/Z$(had) channel}

The limits on the couplings of the $sV$, $sA$ and $tS$ models, obtained within the \monoWZ channel, are shown in the right-hand column of figs.~\ref{fig:results_sVsA_rat05}-\ref{fig:results_tS}. This channel was studied to compare with the leptonic \monoZ channel in particular, but a coarser selection of masses was chosen as the limits were initially found to be somewhat weaker. Additionally, further estimates were made: a) as the kinematic behaviour is reasonable independent of the couplings, a single acceptance was found for each ($\mX$, $\Mmed$) combination and applied to each value of $\gX / \gq$, and b) complete systematic uncertainties were generated for a subset of masses and compared to those from the \monoZ channel; from this comparison the \monoZ systematic uncertainties were multiplied by 3 and then applied to the \monoWZ limits. As a result, the limits obtained in this channel are not intended to be rigorously quantitative; rather, they are used more to indicate qualitatively how the channel compares.

The ATLAS \monoWZ analysis (and in particular the higher $\met$ signal region) was not optimised for a SiM interpretation, and much of the phase space produced insignificant numbers of events passing the event selection, with up to 200 thousand events generated. Generally, the limits are a factor of a few weaker again than those from the \monoZ channel, which is both consistent with the limits on the EFT models studied in the ATLAS analyses, and expected following our use of a cut-and-count interpretation, rather than a shape analysis, of the \monoWZ public results.

In some cases - most notably the $\mX \gtrsim 10 \times \Mmed$ region - the limits become comparable with the \monoZ channel, suggesting that more statistics and an improved treatment of systematic uncertainties would bring these closer in line with that channel.

Overall, the uncertainties from this channel lie within the range \comm{XXX} and are dominated by \comm{Y}.

\afterpage{\clearpage}

%\begin{figure}
%  \centering
%  \begin{subfigure}[t]{0.32\textwidth}
%    \centering
%    \includegraphics[width=1.\textwidth]{figures/grid_basepoints_SVD_rat05_monojet.pdf}
%    \caption{sV model, $g_q/g_{\chi} = 0.5$, \monojet channel.}
%  \end{subfigure}
%  \caption{}
%\end{figure}

\begin{sidewaysfigure}
  \centering
  \begin{subfigure}[t]{0.32\textwidth}
    \centering
    \includegraphics[width=1.\textwidth]{figures/grid_basepoints_SVD_rat05_monojet.pdf}
    \caption{$sV$ model, $\gX/\gq = 0.5$, \monojet channel.}
  \end{subfigure}
  \begin{subfigure}[t]{0.32\textwidth}
    \centering
    \includegraphics[width=1.\textwidth]{figures/grid_allpoints_SVD_rat05.pdf}
    \caption{$sV$ model, $\gX/\gq = 0.5$, mono-$Z$ channel.}
  \end{subfigure}
  \begin{subfigure}[t]{0.32\textwidth}
    \centering
    \includegraphics[width=1.\textwidth]{figures/grid_basepoints_SVD_rat05_monoWZ.pdf}
    \caption{$sV$ model, $\gX/\gq = 0.5$, mono-$W/Z$ channel.}
    \vspace{0.75cm}
  \end{subfigure}
  \begin{subfigure}[t]{0.32\textwidth}
    \centering
    \includegraphics[width=1.\textwidth]{figures/grid_basepoints_SAD_rat05_monojet.pdf}
    \caption{$sA$ model, $\gX/\gq = 0.5$, \monojet channel.}
  \end{subfigure}
  \begin{subfigure}[t]{0.32\textwidth}
    \centering
    \includegraphics[width=1.\textwidth]{figures/grid_allpoints_SAD_rat05.pdf}
    \caption{$sA$ model, $\gX/\gq = 0.5$, mono-$Z$ channel.}
  \end{subfigure}
  \begin{subfigure}[t]{0.32\textwidth}
    \centering
    \includegraphics[width=1.\textwidth]{figures/grid_basepoints_SAD_rat05_monoWZ.pdf}
    \caption{$sA$ model, $\gX/\gq = 0.5$, mono-$W/Z$ channel.}
  \end{subfigure}
  \caption{Upper limits on the coupling for the $s$-channel models in the \monojet (left), \monoZ (centre) and \monoWZ (right) channels, for $\gX / \gq$ = 0.5. The grey region represents the phase space where no meaningful limit was obtained. The hatched region represents a limit which leads to a width greater than $\Mmed / 2$, so the validity of the calculation begins to fail. The dotted region represents phase space where insufficient statistics were available.}
  \label{fig:results_sVsA_rat05}
\end{sidewaysfigure}

\begin{sidewaysfigure}
  \centering
  \begin{subfigure}[t]{0.32\textwidth}
    \centering
    \includegraphics[width=1.\textwidth]{figures/grid_basepoints_SVD_rat1_monojet.pdf}
    \caption{$sV$ model, $\gX/\gq = 1$, \monojet channel.}
  \end{subfigure}
  \begin{subfigure}[t]{0.32\textwidth}
    \centering
    \includegraphics[width=1.\textwidth]{figures/grid_allpoints_SVD_rat1.pdf}
    \caption{$sV$ model, $\gX/\gq = 1$, mono-$Z$ channel.}
  \end{subfigure}
  \begin{subfigure}[t]{0.32\textwidth}
    \centering
    \includegraphics[width=1.\textwidth]{figures/grid_basepoints_SVD_rat1_monoWZ.pdf}
    \caption{$sV$ model, $\gX/\gq = 1$, mono-$W/Z$ channel.}
    \vspace{0.75cm}
  \end{subfigure}
  \begin{subfigure}[t]{0.32\textwidth}
    \centering
    \includegraphics[width=1.\textwidth]{figures/grid_basepoints_SAD_rat1_monojet.pdf}
    \caption{$sA$ model, $\gX/\gq = 1$, \monojet channel.}
  \end{subfigure}
  \begin{subfigure}[t]{0.32\textwidth}
    \centering
    \includegraphics[width=1.\textwidth]{figures/grid_allpoints_SAD_rat1.pdf}
    \caption{$sA$ model, $\gX/\gq = 1$, mono-$Z$ channel.}
  \end{subfigure}
  \begin{subfigure}[t]{0.32\textwidth}
    \centering
    \includegraphics[width=1.\textwidth]{figures/grid_basepoints_SAD_rat1_monoWZ.pdf}
    \caption{$sA$ model, $\gX/\gq = 1$, mono-$W/Z$ channel.}
  \end{subfigure}
  \caption{Upper limits on the couplings for the $s$-channel models in the \monojet (left), \monoZ (centre) and \monoWZ (right) channels, for $\gX / \gq$ = 1. Refer to fig.~\ref{fig:results_sVsA_rat05} for details.}
  \label{fig:results_sVsA_rat1}
\end{sidewaysfigure}

\begin{sidewaysfigure}
  \centering
  \begin{subfigure}[t]{0.32\textwidth}
    \centering
    \includegraphics[width=1.\textwidth]{figures/grid_basepoints_SVD_rat2_monojet.pdf}
    \caption{$sV$ model, $\gX/\gq = 2$, \monojet channel.}
  \end{subfigure}
  \begin{subfigure}[t]{0.32\textwidth}
    \centering
    \includegraphics[width=1.\textwidth]{figures/grid_allpoints_SVD_rat2.pdf}
    \caption{$sV$ model, $\gX/\gq = 2$, mono-$Z$ channel.}
  \end{subfigure}
  \begin{subfigure}[t]{0.32\textwidth}
    \centering
    \includegraphics[width=1.\textwidth]{figures/grid_basepoints_SVD_rat2_monoWZ.pdf}
    \caption{$sV$ model, $\gX/\gq = 2$, mono-$W/Z$ channel.}
    \vspace{0.75cm}
  \end{subfigure}
  \begin{subfigure}[t]{0.32\textwidth}
    \centering
    \includegraphics[width=1.\textwidth]{figures/grid_basepoints_SAD_rat2_monojet.pdf}
    \caption{$sA$ model, $\gX/\gq = 2$, \monojet channel.}
  \end{subfigure}
  \begin{subfigure}[t]{0.32\textwidth}
    \centering
    \includegraphics[width=1.\textwidth]{figures/grid_allpoints_SAD_rat2.pdf}
    \caption{$sA$ model, $\gX/\gq = 2$, mono-$Z$ channel.}
  \end{subfigure}
  \begin{subfigure}[t]{0.32\textwidth}
    \centering
    \includegraphics[width=1.\textwidth]{figures/grid_basepoints_SAD_rat2_monoWZ.pdf}
    \caption{$sA$ model, $\gX/\gq = 2$, mono-$W/Z$ channel.}
  \end{subfigure}
  \caption{Upper limits on the coupling for the $s$-channel models in the \monojet (left), \monoZ (centre) and \monoWZ (right) channels, for $\gX / \gq$ = 2. Refer to fig.~\ref{fig:results_sVsA_rat05} for details.}
  \label{fig:results_sVsA_rat2}
\end{sidewaysfigure}

\begin{sidewaysfigure}
  \centering
  \begin{subfigure}[t]{0.32\textwidth}
    \centering
    \includegraphics[width=1.\textwidth]{figures/grid_basepoints_SVD_rat5_monojet.pdf}
    \caption{$sV$ model, $\gX/\gq = 5$, \monojet channel.}
  \end{subfigure}
  \begin{subfigure}[t]{0.32\textwidth}
    \centering
    \includegraphics[width=1.\textwidth]{figures/grid_allpoints_SVD_rat5.pdf}
    \caption{$sV$ model, $\gX/\gq = 5$, mono-$Z$ channel.}
  \end{subfigure}
  \begin{subfigure}[t]{0.32\textwidth}
    \centering
    \includegraphics[width=1.\textwidth]{figures/grid_basepoints_SVD_rat5_monoWZ.pdf}
    \caption{$sV$ model, $\gX/\gq = 5$, mono-$W/Z$ channel.}
    \vspace{0.75cm}
  \end{subfigure}
  \begin{subfigure}[t]{0.32\textwidth}
    \centering
    \includegraphics[width=1.\textwidth]{figures/grid_basepoints_SAD_rat5_monojet.pdf}
    \caption{$sA$ model, $\gX/\gq = 5$, \monojet channel.}
  \end{subfigure}
  \begin{subfigure}[t]{0.32\textwidth}
    \centering
    \includegraphics[width=1.\textwidth]{figures/grid_allpoints_SAD_rat5.pdf}
    \caption{$sA$ model, $\gX/\gq = 5$, mono-$Z$ channel.}
  \end{subfigure}
  \begin{subfigure}[t]{0.32\textwidth}
    \centering
    \includegraphics[width=1.\textwidth]{figures/grid_basepoints_SAD_rat5_monoWZ.pdf}
    \caption{$sA$ model, $\gX/\gq = 5$, mono-$W/Z$ channel.}
  \end{subfigure}
  \caption{Upper limits on the coupling for the $s$-channel models in the \monojet (left), \monoZ (centre) and \monoWZ (right) channels, for $\gX / \gq$ = 5. Refer to fig.~\ref{fig:results_sVsA_rat05} for details.}
  \label{fig:results_sVsA_rat5}
\end{sidewaysfigure}

\begin{figure}
  \centering
  \begin{subfigure}[t]{0.495\textwidth}
    \centering
    \includegraphics[width=1.\textwidth]{figures/grid_basepoints_SVD_rat02_monojet.pdf}
    \caption{$sV$ model, $\gX/\gq = 0.2$, \monojet channel.}
  \end{subfigure}
  \begin{subfigure}[t]{0.495\textwidth}
    \centering
    \includegraphics[width=1.\textwidth]{figures/grid_basepoints_SAD_rat02_monojet.pdf}
    \caption{$sA$ model, $\gX/\gq = 0.2$, \monojet channel.}
  \end{subfigure}
  \caption{Upper limits on the coupling for the $s$-channel models in the \monojet channel, for $\gX / \gq$ = 0.2. Refer to fig.~\ref{fig:results_sVsA_rat05} for details.}
  \label{fig:results_sVsA_rat02}
\end{figure}

\begin{figure}
  \centering
  \begin{subfigure}[t]{0.495\textwidth}
    \centering
    \includegraphics[width=1.\textwidth]{figures/grid_allpoints_TSD_rat1.pdf}
    \caption{$tS$ model, mono-$Z$ channel.}
  \end{subfigure}
  \begin{subfigure}[t]{0.495\textwidth}
    \centering
    \includegraphics[width=1.\textwidth]{figures/grid_basepoints_TSD_rat1_monoWZ.pdf}
    \caption{$tS$ model, mono-$W/Z$ channel.}
  \end{subfigure}
  \caption{Upper limits on the coupling $\gqX$ for the $t$-channel model in the \monoZ (left) and \monoWZ (right) channels. Refer to fig.~\ref{fig:results_sVsA_rat05} for details.}
  \label{fig:results_tS}
\end{figure}

\subsection{Comparison with Relic Density Constraints}
%%%%%%%%%%%%%%%%%%%%%%%%%%%%%%%%

%\comm{Copied from my paper with Karl, so I'll have to rewrite - Tom.}

In Figs.~\ref{} we show lines where the constraint on the coupling corresponds to the coupling strength that would reproduce the correct DM density if DM is a thermal relic of the early universe. For points diagonally above and to the left of the dashed line, the LHC constraints naively rule out the couplings leading to the correct relic density. Below and to the right of this line the relic density coupling is still allowed.

 In this scenario, the measured abundance is approximately related to the unknown self-annihilation cross-section via
%
\begin{equation}
  \Omega_{\rm DM}h^2\simeq \frac{2\times2.4\times 10^{-10}\,{\rm GeV}^{-2}}{\langle\sigma v\rangle_{\rm ann}}.
  \label{simplerelic}
\end{equation}
%
This is used with measurements of the DM abundance by Planck, $\Omega_{\rm DM}^{\rm obs}h^2=0.1199\pm0.0027$ \cite{Ade:2013zuv}, to find $\sigv_{\rm ann}\simeq 4.0\times 10^{-9}\,{\rm GeV}^{-2}$ for thermal relic DM.
%
This relation is only approximately accurate, and so we use the micrOMEGAs code \cite{Belanger:2014vza} to determine the coupling strength leading to the correct relic density for each model. We verified this technique against the semi-analytic technique outlined in e.g. ref.~\cite{Busoni:2014gta}.

If the DM mass lies at the electroweak scale, the thermal relic scenario provides a natural explanation for the observed DM density, and so the coupling strengths leading to the correct relic density are a natural  benchmark with which to compare constraints from other DM searches, indicating the scale at which we expect the couplings may lie. However the relic density couplings should by no means be treated as a constraint. If the DM was not produced thermally or if there is some unknown effect which modifies the evolution of the density with temperature, then these relations break down. Further, even if DM is a thermal relic, then the relationship no longer holds if there are other annihilation channels not taken into account, or if there are other beyond-SM particles contributing to the DM abundance.

%%%%%%%%%%%%%%%%%%%%%%%%%%%%%%%%
\subsection{Comparison with Direct Detection Constraints}
%%%%%%%%%%%%%%%%%%%%%%%%%%%%%%%%

In Figs.~\ref{} we also show the intercept line where constraints from  direct detection experiments are equally as strong as the LHC constraint. Below and to the right of the dotted line, direct detection constraints are stronger than the LHC constraint, while above and to the left, the LHC gives the stronger constraint. We use the toolset from Ref.~\cite{1307.5955} to convert the strongest available direct detection constraints, which are from the LUX 2013 dataset ~\cite{1309.3259}, onto constraints on our models.

Compared to direct detection, the LHC performs relatively better for the SAD model than for the SVD model. This is because the axial-vector coupling leads to a suppressed scattering rate in direct detection experiments while the LHC is relatively insensitive to the difference between the vector and axial-vector couplings. In the non-relativistic limit, the TSD model leads to a mix of both suppressed and unsuppressed operators.

The direct detection constraints assume that the DM candidate under consideration contributes 100\% of the local DM density, while the LHC constraints make no assumptions about either the local DM density or overall abundance. In this sense the LHC constraints remain useful even in the region where they are not as strong as those from direct detection.


%%%%%%%%%%%%%%%%%%%%%%%%%%%%%%%
\subsection{Discussion}
%%%%%%%%%%%%%%%%%%%%%%%%%%%%%%%

\begin{itemize}

\item Comparison to direct mediator searches: dijet gives strongest constraints on mediator especially for small r. Missing ET still good for large M but in this region EFT is fine

\item Comparison to non-grid searches, e.g. McCullough et al

\item Comparison to grid searches e.g. Zurek et al, Jacques and Nordstrom

\end{itemize}


\section{Conclusion} 
\label{sec:sec5}
% !TEX root = ../new_paper.tex
%\begin{flushleft}

In this paper we have examined a set of three simplified dark matter models, extracting constraints from ATLAS Run I missing energy searches featuring the associated production of a mono-jet, $Z(\rightarrow$ leptons), or $W/Z (\rightarrow$ hadrons). We explored a phase space where both the DM and mediator masses span $\mathcal{O}$(GeV) to $\mathcal{O}$(TeV), and considered ratios of $\gX / \gq$ of 0.2, 0.5, 1, 2 and 5 in the $s$-channel models. Where $\mX > \Mmed$ and perturbative unitarity isn't violated (in the $sA$ model), we applied a reweighting procedure to account for the \MG~treatment of the mediator as a Breit-Wigner propagator.  Rather than setting limits in the $\Mmed - \mDM$ plan for a fixed value of the coupling strength, we instead constrained the coupling strength as a function of both $\Mmed$ and $\mDM$ in a 3D plane. Whilst this approach necessitates the introduction of some approximations, it also allows for a thorough examination of the interplay between the DM production cross-section and the free parameters of the models.

As expected, the \monojet channel is found to yield the strongest limits on vector and axial-vector SM and DM couplings to a vector mediator exchanged in the $s$-channel. This channel is also found to perform well for small values of $\gX$. The limits obtained in the \monoZ channel, in comparison, are generally weaker by a factor of a few, while the \monoWZ results are weaker again. This is partly due to our conservative estimations of the systematic uncertainties and partly due to limited statistics resulting from a harder $\met$ selection cut. The width effects associated with the $t$-channel exchange of an SU(2) doublet scalar mediator are observed to vanish in both the \monoZ and \monoWZ channels, greatly simplifying the analysis and confirming these as straightforward and competitive channels for future collider DM detection.

Where the axial-vector model is not excluded by perturbative unitarity requirements, we find the coupling limits to be on par with those of the vector model within each analysis channel. Weaker limits are found for the $t$-channel model, a result of cross-section suppression not present in the $s$-channel models.

Finally, we compared our limits to constraints from relic density and direct detection; although each search has a different set of assumptions, this demonstrates the complementarity and impressive reach of simplified models as a tool for the interpretation of collider DM searches. We eagerly await the improved constraints expected from Run II of the LHC.

%For example, where $\mDM$ and $\Mmed$ are (near-)degenerate, \textcolor{magenta}{on-shell} and low ($\mathcal{O}$(1 - 20 GeV)), the strongest limits on $\sqrt{\gq\gX}$ are $<$0.4 (coming from the \monojet channel). Outside of this regime, the strongest valid limits range from $<$0.001 to 0.9 (again coming from the \monojet channel). For the $t$-channel model, we find that the best limits on $\gqX$ originate from the \monoZ channel and range from \textcolor{magenta}{something to something}.


%\iffalse
%%%%%%%%%%%%%%%%%%%%%%%%%%%%%%%%%
%\subsection{Comparison with Relic Density Constraints}
%%%%%%%%%%%%%%%%%%%%%%%%%%%%%%%%%
%
%%\comm{Copied from my paper with Karl, so I'll have to rewrite - Tom.}
%
%In Figs.~\ref{} we show lines where the constraint on the coupling corresponds to the coupling strength that would reproduce the correct DM density if DM is a thermal relic of the early universe. For points diagonally above and to the left of the dashed line, the LHC constraints naively rule out the couplings leading to the correct relic density. Below and to the right of this line the relic density coupling is still allowed.
%
% In this scenario, the measured abundance is approximately related to the unknown self-annihilation cross-section via
%%
%\begin{equation}
%  \Omega_{\rm DM}h^2\simeq \frac{2\times2.4\times 10^{-10}\,{\rm GeV}^{-2}}{\langle\sigma v\rangle_{\rm ann}}.
%  \label{simplerelic}
%\end{equation}
%%
%This is used with measurements of the DM abundance by Planck, $\Omega_{\rm DM}^{\rm obs}h^2=0.1199\pm0.0027$ \cite{Ade:2013zuv}, to find $\sigv_{\rm ann}\simeq 4.0\times 10^{-9}\,{\rm GeV}^{-2}$ for thermal relic DM.
%%
%This relation is only approximately accurate, and so we use the Micromegas code \cite{Belanger:2014vza} to determine the coupling strength leading to the correct relic density for each model. We verified this technique against the semi-analytic technique outlined in e.g. Ref.~\cite{Busoni:2014gta}.
%
%If the DM mass lies at the electroweak scale, the thermal relic scenario provides a natural explanation for the observed DM density, and so the coupling strengths leading to the correct relic density are a natural  benchmark with which to compare constraints from other DM searches, indicating the scale at which we expect the couplings may lie. However the relic density couplings should by no means be treated as a constraint. If the DM was not produced thermally or if there is some unknown effect which modifies the evolution of the density with temperature, then these relations break down. Further, even if DM is a thermal relic, then the relationship no longer holds if there are other annihilation channels not taken into account, or if there are other beyond-SM particles contributing to the DM abundance.
%
%%%%%%%%%%%%%%%%%%%%%%%%%%%%%%%%%
%\subsection{Comparison with Direct Detection Constraints}
%%%%%%%%%%%%%%%%%%%%%%%%%%%%%%%%%
%
%In Figs.~\ref{} we also show the intercept line where constraints from  direct detection experiments are equally as strong as the LHC constraint. Below and to the right of the dotted line, direct detection constraints are stronger than the LHC constraint, while above and to the left, the LHC gives the stronger constraint. We use the toolset from Ref.~\cite{1307.5955} to convert the strongest available direct detection constraints, which are from the LUX 2013 dataset ~\cite{1309.3259}, onto constraints on our models.
%
%Compared to direct detection, the LHC performs relatively better for the SAD model than for the SVD model. This is because the axial-vector coupling leads to a suppressed scattering rate in direct detection experiments while the LHC is relatively insensitive to the difference between the vector and axial-vector couplings. In the non-relativistic limit, the TSD model leads to a mix of both suppressed and unsuppressed operators.
%
%The direct detection constraints assume that the DM candidate under consideration contributes 100\% of the local DM density, while the LHC constraints make no assumptions about either the local DM density or overall abundance. In this sense the LHC constraints remain useful even in the region where they are not as strong as those from direct detection.
%
%
%%%%%%%%%%%%%%%%%%%%%%%%%%%%%%%%
%\subsection{Discussion}
%%%%%%%%%%%%%%%%%%%%%%%%%%%%%%%%
%
%\begin{itemize}
%
%\item Comparison to direct mediator searches: dijet gives strongest constraints on mediator especially for small r. Missing ET still good for large M but in this region EFT is fine
%
%\item Comparison to non-grid searches, e.g. McCullough et al
%
%\item Comparison to grid searches e.g. Zurek et al, Jacques and Nordstrom
%
%\end{itemize}
%
%\fi

%\end{flushleft}


\section{Acknowledgements} 
\label{sec:sec6}

\appendix
\section{Validation of Monte Carlo Generation and Event Selection }
\label{AppendixA}
\subsection{Monojet Channel}
\label{monojet_validation}
\begin{flushleft}
As mentioned in Section \ref{monojet_constraints}, the signal selection and Monte Carlo (MC) generation procedures for the monojet channel were validated via reproduction of the ATLAS limits on the suppression scale, M$_{*}$, for the s-channel vector simplified model. A comparison of SR7 limits for a representative sample of mediator masses with $m_{\chi} = $ 50 GeV, $\Gamma = M/8\pi$ and $\sqrt{g_{q}g_{\chi}} = 1$ is presented in Table \ref{M_star_limits_monojet}. In general, good agreement is observed between the ATLAS and reproduced limits, with a maximum discrepancy of $>$11\%. We note that a discrepancy of a few percent is expected and allowed for two reasons. Firstly, the Monte Carlo generation procedure employed in this analysis does not include a full simulation of the ATLAS detector. \comm{Instead, reconstruction effects are modelled somewhat crudely by smearing the leading jet $p_{T}$ distribution by a conservative factor of 5\%.} Secondly, the 95\% CL uncertainties associated with the ATLAS limits are estimated using the HistFitter package while a different approach is favoured for this analysis (see appendix \ref{AppendixB}). \comm{Given that the difference between the ATLAS and reproduced limits is consistently negative, we consider this approach acceptable.}

%In general, our approach  results in more conservative limits on M$_{*}$ but ultimately removes the hassle of a full histFitter analysis. Importantly, agreement between M$_{*}^{\mbox{\tiny ATLAS,95}}$ and the reproduced nominal\footnote{Nominal in this case explicitly refers to a quantity which does not yet include statistical or systematic uncertainties.} M$_{*}$ values (denoted M$_{*}^{\mbox{\tiny R,N}}$ in Table \ref{M_star_limits_monojet}) is still reasonably high - within \comm{something percent} - validating our signal selection and MC generation procedures.}

%Also note that while the signal samples used for validation of the monojet channel are generated for the processes $pp \rightarrow j\chi\bar{\chi}$ and $pp \rightarrow jj\chi\bar{\chi}$ where $j$ is a final state jet, the signal samples used to constrain the the simplified models discussed in Section \ref{SiM_models} also include the tree-level process $pp \rightarrow \chi \bar{\chi}$. \comm{This approach is taken as...}

%\begin{table}[!htbp]
%\centering
%\begin{tabular}{c|c|c|c|c|c}
% \hline
% \hline
% $M$ [TeV] & M$_{*}^{\mbox{\tiny ATLAS,95}}$ [GeV] & M$_{*}^{\mbox{\tiny R,N}}$ [GeV] & Difference [\%] & M$_{*}^{\mbox{\tiny R,95}}$ [GeV] & Difference [\%] \\ 
% \hline
%0.05 & 91 & 100.05 & +9.05 & 94.33 & +3.53 \\
%%0.1 & 217 & 322.70 & +32.75 & 280.57 & +22.66 \\
%0.3 & 1151 & 1288.15 & +10.65 & 1092.52 & $-$5.35 \\
%0.6 & 1868 & 2013.68 & +7.23 & 1668.27 & $-$11.97 \\
%1 & 2225 & 2363.06 & +5.84 & 1975.58 & $-$12.63 \\
%3 & 1349 & 1479.66 & +8.83 & 1274.73 & $-$5.83 \\
%%6 & 945 & 856.37 & $-$10.35 & 730.98 & $-$29.28 \\
%10 & 928 & 1000.93 & +7.29 & 842.03 & $-$10.21 \\
%30 & 914 & 989.49 & +7.54 & 838.34 & $-$9.03\\
% \hline
% \hline
%\end{tabular}
%\caption{Comparison of the ATLAS 95\% CL limits on M$_{*}$ (denoted M$_{*}^{\mbox{\tiny ATLAS,95}}$) with the reproduced nominal and reproduced 95\% CL limits on M$_{*}$ (denoted M$_{*}^{\mbox{\tiny R,N}}$ and M$_{*}^{\mbox{\tiny R,95}}$ respectively) for the s-channel vector mediator model with $m_{\chi} = $ 50 GeV, $\Gamma = M/8\pi$, $\sqrt{g_{q}g_{\chi}} = 1$ and QCUT = 80 GeV. Adapted from Ref. \cite{Aad:2015zva}.}
%\label{M_star_limits_monojet}
%\end{table}

\begin{table}[!htbp]
\centering
\begin{tabular}{c|c|c|c}
 \hline
 \hline
 $M$ [TeV] & M$_{*}^{\mbox{\tiny ATLAS,95}}$ [GeV] & M$_{*}^{\mbox{\tiny REPRO,95}}$ [GeV] & Difference [\%] \\ 
 \hline
 0.05 & 91 & 84.20 & $-$7.47 \\%$-$8.08 \\
%0.1 & 217 & 246.54 & +11.96 \\
0.3 & 1151 & 1088.87 & $-$5.40 \\ %$-$5.71 \\
0.6 & 1868 & 1697.19 & $-$9.14 \\% $-$10.06 \\
1 & 2225 & 1986.67 & $-$10.71 \\ %$-$12.00 \\
3 & 1349 & 1241.93 & $-$7.94 \\% $-$8.62 \\
%6 & 945 & 721.87 & $-$30.91 \\
10 & 928 & 844.33 & $-$9.02 \\% $-$9.91 \\
30 & 914 & 834.56 & $-$8.69 \\% $-$9.52\\

%0.05 & 91 & 94.33 & +3.53 \\
%%0.1 & 217 & 280.57 & +22.66 \\
%0.3 & 1151 & 1092.52 & $-$5.35 \\
%0.6 & 1868 & 1668.27 & $-$11.97 \\
%1 & 2225 & 1975.58 & $-$12.63 \\
%3 & 1349 & 1274.73 & $-$5.83 \\
%%6 & 945 & 730.98 & $-$29.28 \\
%10 & 928 & 842.03 & $-$10.21 \\
%30 & 914 & 838.34 & $-$9.03\\
 \hline
 \hline
\end{tabular}
\caption{Comparison of the 95\% confidence level upper limits on M$_{*}$ from this work (M$_{*}^{\mbox{\tiny REPRO,95}}$) and from the ATLAS monojet analysis (M$_{*}^{\mbox{\tiny ATLAS,95}}$). The values shown in columns two and three are   for the processes $pp \rightarrow j\chi\bar{\chi}$ and $pp \rightarrow jj\chi\bar{\chi}$ for the s-channel vector mediator model with $m_{\chi} = $ 50 GeV, $\Gamma = M/8\pi$, $\sqrt{g_{q}g_{\chi}} = 1$ and QCUT = 80 GeV + 300 GeV. The difference shown in column four corresponds to 100$\times$(M$_{*}^{\mbox{\tiny REPRO,95}}$ $-$ M$_{*}^{\mbox{\tiny ATLAS,95}}$)/M$_{*}^{\mbox{\tiny ATLAS,95}}$.
Adapted from Ref. \cite{Aad:2015zva}. \comm{Just a thought - would it make more sense to calculate these as (ATLAS - us)/ATLAS? Then we say ours are x\% smaller/more conservative than the official values. Also added bonus of reducing the percentages slightly. -Mia} Done! - Millie}
\label{M_star_limits_monojet}
\end{table}
\end{flushleft}

\subsection{Mono-Z Channel}
\label{monoZ_validation}
\begin{flushleft}

\bigskip

The ATLAS mono-$Z$ analysis result includes an upper limit on the coupling $g_{q \chi}$ for a $t$-channel simplified model that is very similar to the model tested here, and so is used for validating our procedure. The most significant differences are in the number of mediating particles --- the ATLAS model includes just two mediators ($up$- and $down$-type) compared to our six --- and in the nature of the DM particle, which is taken to be Majorana. This latter choice does not impact the kinematic behaviour, but does impact the cross section. Additionally, while we use a universal coupling $g_{q \chi}$ to all three quark generations, the analysis used a model which set $g_{t,b \chi}$ = 0.

\begin{table}
\begin{center}
\begin{tabular}{| c | c | c | c | c |}
\hline
\hline
$m_{\chi}$ & $M_{med}$ & $g_{q \chi}^{95\%\mathrm{CL}}$ & $g_{q \chi}^{95\%\mathrm{CL}}$ & Difference \T \\
$[$GeV$]$ & $[$GeV$]$ & (ATLAS) & (this work) & $[\%]$ \B \\
\hline
10 & 200 & 1.9 & 2.0 & -5.3 \T \\
 & 500 & 2.8 & 3.2 & -14.3 \\
 & 700 & 3.5 & 4.4 & -25.7 \\
 & 1000 & 4.5 & 5.2 & -15.6 \\
200 & 500 & 3.4 & 4.0 & -17.6 \T \\
 & 700 & 4.2 & 4.5 & -7.1 \\
 & 1000 & 5.2 & 5.3 & -1.9 \\
400 & 500 & 5.5 & 5.7 & -3.6 \T \\
 & 700 & 6.1 & 6.5 & -6.6 \\
 & 1000 & 7.2 & 7.4 & -2.8 \\
1000 & 1200 & 23.3 & 24.1 & -3.4 \T \B \\
\hline
\hline
\end{tabular}
\end{center}
\caption{Comparison of the upper limit on $g_{q \chi}$ from the ATLAS analysis (middle column) and this work.}
\label{tab:monoZvalidation}
\end{table} 

Table \ref{tab:monoZvalidation} shows the 95\% CL upper limits on $g_{q \chi}$ that we calculate using the same $t$-channel model and our own generation procedure, compared with the limits on this same variable taken from the ATLAS analysis. The difference as a percentage of the ATLAS limit is also shown in the table. We see reasonable agreement; most of the 11 points in parameter space are within 10\% of the ATLAS limits, and all are within 26\%. Additionally, our results are consistently more conservative, which is to be expected due to the less sophisticated nature of our generation procedure. Similarly to the mono-jet validation, the dominant effects are due to the use of $p_{\mathrm{T}}$ smearing applied to the leptons, rather than considering the full reconstruction effects, and the simple systematic treatment that was used with HistFitter.

\end{flushleft}

\iffalse

\subsection{Mono-W/Z Channel}
\label{monoWZ_validation}
\begin{flushleft}
\comm{Johanna, please put your validation results here.}
\end{flushleft}

\fi


\section{Limit Setting Strategy}
\label{AppendixB}
\begin{flushleft}
In this appendix we present a summation of the procedure employed to calculate the 95\% confidence level limits on M$_{*}$, $\sigma(pp \rightarrow X + \chi\bar{\chi})$ and $\sqrt{g_{q}g_{\chi}}$. Note that the following formulae are presented for the observed limits but also apply to the expected limits.
\end{flushleft}

\subsection{Nominal Values}
\begin{flushleft}
For each simplified model, the nominal value for the observed limit on the cross-section for the process $pp \rightarrow X + \chi\bar{\chi}$ is calculated using the formula:

\begin{equation}
\label{sigma_nom}
\sigma_{obs}^{lim}(pp \rightarrow X + \chi\bar{\chi}) = \frac{N_{obs}}{\mathcal{L}\times\mathcal{A}\times\epsilon}
\end{equation}

where $N_{obs}$ is the number of observed signal events in the channel and signal region of interest (a model-independent quantity), $\mathcal{L}$ is the integrated luminosity, $\mathcal{A}$ is the acceptance (the fraction of signal events passing the channel/SR-specific selection criteria) and $\epsilon$ is the efficiency of the ATLAS detector for selecting channel/SR-specific signal events. For both $X=0j +1j + 2j$ and $X=Z$, $\mathcal{L}=20.3$ fb$^{-1}$ and $\mathcal{A}\times\epsilon$ is regarded as a single variable.
\bigskip

The nominal value for the observed limit on the suppression scale is then calculated using the equation:

\begin{equation}
\label{M_*_nom}
\mbox{M}_{*, obs}^{lim} = \mbox{M}_{*}^{gen}\left[\frac{\sigma_{gen}}{\sigma_{obs}^{lim}(pp \rightarrow X + \chi\bar{\chi})}\right]^{1/4}
\end{equation}

where M$_{*}^{gen}$ is the theoretical suppression scale and $\sigma_{gen}$ is the theoretical cross-section.
\bigskip

Similarly, the nominal value for the limit on the observed coupling constants is calculated using the equation:

\begin{equation}
\label{coupling_nom}
(\sqrt{g_{q}g_{\chi}})_{obs}^{lim} = (\sqrt{g_{q}g_{\chi}})^{gen}\left[\frac{\sigma_{obs}^{lim}(pp \rightarrow X + \chi\bar{\chi})}{\sigma_{gen}}\right]^{1/4}
\end{equation}

where $(\sqrt{g_{q}g_{\chi}})^{gen}$ is the product of the theoretical coupling constants. Note that $(\sqrt{g_{q}g_{\chi}})^{gen}$ is always equal to 1 for the t-channel scalar mediator model. \comm{DOUBLE CHECK THIS!}
\end{flushleft}

\subsection{Uncertainty Estimation}
\begin{flushleft}
Our nominal limits on M$_{*}$, $\sigma(pp \rightarrow X + \chi\bar{\chi})$ and $\sqrt{g_{q}g_{\chi}}$ rely on both $\sigma_{gen}$ and $\mathcal{A}\times\epsilon$ and so are subject to systematic uncertainties which derive from our choice of MC generation procedure. For our MC samples, there are three key sources of systematic uncertainty: the factoriation and renormalisation scales, the strong coupling constant ($\alpha_{s}$) and the parton distribution function (PDF). The uncertainty associated with these parameters is estimated as follows.
\bigskip

%first for the theoretical cross-section, $\sigma_{gen}$ and then for the acceptance, $\mathcal{A}$, which is defined as:
%\begin{equation}
%\mathcal{A} = \frac{N_{truth}}{N_{total}}
%\end{equation}
%where $N_{truth}$ is the number of truth-level\footnote{In the vernacular, `truth-level' events/objects are independent of detector effects.} signal events passing the selection criteria of a specific channel and $N_{total}$ is the total number of truth-level signal events for that channel.
%\bigskip
%
%The uncertainty on 
%\end{flushleft}
%
%\subsection{Theoretical cross-section, $\sigma_{gen}$}
%\begin{flushleft}
Firstly, the factorisation and renormalisation default scales are varied simultaneously by a factor of 0.5 \fg{(`down')} and by a factor of 2 \fg{(`up')} \st{times the default scale}. The uncertainty on the theoretical cross-section, $\Delta \sigma_{gen}$, is then taken to be the average change in $\sigma_{gen}$ resulting from these up and down variations.
\bigskip

\draft{The systematic effects of the strong coupling constant and the PDF are difficult to separate and so are treated in tandem. We begin by assuming that the systematic uncertainty introduced by $\alpha_{s}$ at matrix-element level is negligible when compared to the PDF uncertainties. Similarly, we assume that the PDF effects are negligible when compared to the $\alpha_{s}$ systematic uncertainties at the parton-shower level. \comm{This approach is validated in Ref. (see Ref. \cite{CERN-THESIS-2015-038})(?)} While there is no way to change the value of $\alpha_{s}$ in \MG, there are specific `tunes' in
Pythia which can be used to perform this function. Hence we estimate the uncertainty on $\Delta \sigma_{gen}$ by changing the choice of PDF and tune simultaneously. Specifically, we change the nominal PDF (MSTW2008lo68) and tune (ATLAS UE AU2-MSTW2008LO) to the CTEQ6L1 PDF and ATLAS UE AU2-CTEQ6L1 tune respectively and then to the NNPDF2.1LO PDF and Monash tune respectively. The PDFs chosen represent the three main PDFs considered for LHC physics \cite{CooperSarkar:2011vp} while the associated tunes are the recommended defaults \comm{reference here}. The systematic uncertainty on $\sigma_{gen}$ is then calculated from the mean of the two variations (MSTW2008lo68 + ATLAS UE AU2-MSTW2008LO to CTEQ6L1 + ATLAS UE AU2-CTEQ6L1 and MSTW2008lo68 + ATLAS UE AU2-MSTW2008LO to NNPDF2.1LO + Monash). \question{Amelia - did you want to make any changes/further comments here?}}
\bigskip

For $\sigma_{gen}$, the uncertainty is then calculated using the formula:
\begin{equation}
\label{uncertainty_sigma_gen}
\left(\frac{\Delta \sigma_{gen}}{\sigma_{gen}}\right)^{2} = \left(\frac{\Delta \sigma_{gen}}{\sigma_{gen}}\right)_{\mbox{\footnotesize scale}}^{2} + \left(\frac{\Delta \sigma_{gen}}{\sigma_{gen}}\right)_{\mbox{\footnotesize PDF+tune}}^{2}
\end{equation}

where $\sigma_{gen}$ is the nominal theoretical cross-section, $(\Delta \sigma_{gen})_{\mbox{\footnotesize scale}}$ is the uncertainty on $\sigma_{gen}$ due to the factorisation and renormalisation scales and $(\Delta \sigma_{gen})_{\mbox{\footnotesize PDF+tune}}$ is the uncertainty on $\sigma_{gen}$ due to the PDF and strong coupling constant.
\bigskip

\textcolor{magenta}{Should we say something here about using only leading order predictions? For example, should we comment on the uncertainty associated with not including NLO corrections to the cross-section?} %It is important to note that a complete examination of the uncertainties associated
%with each simplified model and each mass and coupling combination was not necessary.

\comm{Good question, can Thomas comment? Tom+Karl paper seems to suggest the impact is negligible? - Amelia}
\bigskip

The uncertainty on the acceptance $\times$ efficiency is estimated using a similar approach but with two key differences. Firstly, the statistical uncertainty (taken to be the 95\% confidence interval on $\mathcal{A}\times\epsilon$ as calculated using the \comm{Wald approximation}) is subtracted from the nominal value of $\mathcal{A}\times\epsilon$. Equation \ref{uncertainty_sigma_gen} is then applied to this new variable (denoted $\mathcal{A}'$). Secondly, the matching scale (qcut) between \MG and \PYTHIA is included when estimating the uncertainty on $\mathcal{A}'$ for the monojet channel. Following the approach utilised by the ATLAS group \cite{CERN-THESIS-2015-038}, conservative matching scale uncertainties of 10\% for events with $E_{T}^{miss} <$ 350 GeV and 60\% for events with $E_{T}^{miss} >$ 350 GeV were used \comm{for the validation}.
 
%As in the case of the factorisation and renormalisation scales, the matching scale uncertainty is determined by varying the value of the qcut up by a factor of two (to a value of $m_{\chi}/2$) and down by a factor of two (to a value of $m_{\chi}/8$). The uncertainty on $\mathcal{A}'$ is then quantified as the average change in $\mathcal{A}'$ resulting from these up and down variations. 
\comm{The qcut uncertainty is ignored at the theoretical cross-section level because...}
\bigskip

Finally, the 95\% CL uncertainties on $\sigma_{obs}^{lim}$, M$_{*}$ and $f$ are given by the following equations:

\begin{equation}
\label{uncertainty_sigma_lim}
\frac{\Delta \sigma_{obs}^{lim}}{\sigma_{obs}^{lim}} = \sqrt{\left(\frac{\Delta \mathcal{A}'}{\mathcal{A}'}\right)^{2} + \left(\frac{\Delta \mathcal{L}}{\mathcal{L}}\right)^{2} + \left(\frac{\Delta N}{N}\right)^{2}}
\end{equation}

\begin{equation}
\label{uncertainty_M_star}
\frac{\Delta \mbox{M}_{*,obs}^{lim}}{\mbox{M}_{*,obs}^{lim}} = \frac{\Delta (\sqrt{g_{q}g_{\chi}})_{obs}^{lim}}{(\sqrt{g_{q}g_{\chi}})_{obs}^{lim}} = \left|\frac{1}{4}\right|\sqrt{\left(\frac{\Delta \sigma_{gen}}{\sigma_{gen}}\right)^{2} + \left(\frac{\Delta \sigma_{obs}^{lim}}{\sigma_{obs}^{lim}}\right)^{2}}
\end{equation}

\question{Should we have more of an explanation for why we use formulae \ref{sigma_nom} through \ref{uncertainty_M_star}?}
\end{flushleft}

\begin{thebibliography}{99}
\bibitem{Aad:1363019} ATLAS Collaboration, \emph{Search for new phenomena with the monojet and missing transverse momentum signature using the ATLAS detector in $\sqrt{s}$ = 7 TeV proton-proton collisions}, \emph{Phys. Lett. B} (2011), arXiv:1106.5327.

\bibitem{ATLAS-CONF-2012-147} ATLAS Collaboration. \emph{Search for New Phenomena in Monojet plus Missing Transverse Momentum Final States using 10 fb$^{1}$ of pp collisions at $\sqrt{s}$ = 8 TeV with the ATLAS detector at the LHC}, 2012, ATLAS-CONF-2012-147.

\bibitem{CMS-PAS-EXO-12-048} CMS Collaboration. \emph{Search for new physics in monojet events in pp collisions at $\sqrt{s}$ = 8 TeV}, 2013, CMS-PAS-EXO-12-048.

\bibitem{Abdallah:1472683} ATLAS Collaboration, \emph{Search for new phenomena with mono-jet plus missing transverse energy signature in pp collisions at $\sqrt{s}$=8 TeV with the ATLAS detector}, 2012, ATL-COM-PHYS-2012-1211.

\bibitem{Aad:2012ms} ATLAS Collaboration. \emph{Further search for supersymmetry at $\sqrt{s} =$ 7 TeV in final states with jets, missing transverse momentum and isolated leptons with the ATLAS detector}, \emph{Phys.Rev.} (2012), arXiv:1208.4688.

\bibitem{Aad:2012fqa} ATLAS Collaboration, \emph{Search for squarks and gluinos with the ATLAS detector in final states with jets and missing transverse momentum using 4.7 fb$^{-1}$ of $\sqrt{s}=7$ TeV proton-proton collision data}, \emph{Phys.Rev.} D 87, 012008 (2013), arXiv:1208.0949.

\bibitem{Aad:2014wea} ATLAS Collaboration, \emph{Search for squarks and gluinos with the ATLAS detector in final states with jets and missing transverse momentum using $\sqrt{s}=8$ TeV proton-proton collision data}, \emph{JHEP} 09 (2014) 146, arXiv:1405.7875.

\bibitem{SUSY_official_paper} ATLAS Collaboration, \emph{Search for pair-produced third-generation squarks decaying via charm quarks or in compressed supersymmetric scenarios in $pp$ collisions at $\sqrt{s}=8~$TeV with the ATLAS detector}, \emph{Phys.Rev.} D 90, 052008 (2014), arXiv:1407.0608.

\bibitem{DMCons2} J. Goodman et al., \emph{Constraints on Dark Matter from Colliders}, \emph{Phys.Rev.} D 82, 116010 (2010), arXiv:1008.1783.

\bibitem{Bai:2010hh} Y. Bai, P. J. Fox, and R. Harnik, \emph{The Tevatron at the Frontier of Dark Matter Direct Detection}, \emph{JHEP} (2010), arXiv:1005.3797.

\bibitem{Fox:2011fx} P. J. Fox, R. Harnik, J. Kopp, and Y. Tsai, \emph{LEP Shines Light on Dark Matter}, \emph{Phys.Rev.} (2011), arXiv:1103.0240.

\bibitem{Graesser:2011vj} M. L. Graesser, I. M. Shoemaker, and L. Vecchi, \emph{A Dark Force for Baryons}, 2011, arXiv:1107.2666.

\bibitem{An:2011ck} H. An and F. Gao, \emph{Fitting CoGeNT Modulation with an Inelastic, Isopin-Violating $Z'$ Model}, 2011, arXiv:1108.3943.

\bibitem{EFTDM} H. Dreiner et al., \emph{Contact Interactions Probe Effective Dark Matter Models at the LHC}, \emph{Europhys.Lett.} (2013), arXiv:1303.3348.

\bibitem{DMCons3} J. Goodman et al., \emph{Gamma Ray Line Constraints on Effective Theories of Dark Matter}, \emph{Nucl.Phys.} (2011), arXiv:1009.0008.

\bibitem{Buchmueller:2013dya} O.~Buchmueller, M.~J.~Dolan and C.~McCabe,
  \emph{Beyond Effective Field Theory for Dark Matter Searches at the LHC,}
  \emph{JHEP}   1401 (2014) 025
  %doi:10.1007/JHEP01(2014)025
  arXiv:1308.6799.

\bibitem{ValidEFT} G. Busoni et al., \emph{On the Validity of the Effective Field Theory for Dark Matter Searches at the LHC}, \emph{Phys.Lett.} (2014), arXiv:1307.2253.

\bibitem{ValidEFT_part2} G. Busoni et al., \emph{On the Validity of the Effective Field Theory for Dark Matter Searches at the LHC, Part II: Complete Analysis for the s-channel}, \emph{JCAP} 1406:060 (2014), arXiv:1402.1275.

\bibitem{ValidEFT_part3} G. Busoni et al., \emph{On the Validity of the Effective Field Theory for Dark Matter Searches at the LHC Part III: Analysis for the $t$-channel}, \emph{JCAP} 09 (2014) 022, arXiv:1405.3101.

\bibitem{DM_MET_LHC} J. Abdallah et al. \emph{Simplified Models for Dark Matter and Missing Energy Searches at the LHC}, (2014), arXiv:1409.2893.

\bibitem{DMOxfordReport} J. Abdallah et al. \emph{Simplified Models for Dark Matter Searches at the LHC}, \emph{Phys. Dark Uni.} 9-10 (2015) 8-23, arXiv:1506.03116.

\bibitem{DMForumReport} D. Abercrombie et al. \emph{Dark Matter Benchmark Models for Early LHC Run-2 Searches: Report of the ATLAS/CMS Dark Matter Forum}, arXiv:1507.00966.

\bibitem{Harris:2014hga} P. Harris, V. V. Khoze, M. Spannowsky and C. Williams, \emph{Constraining Dark Sectors at Colliders: Beyond the Effective Theory Approach}, \emph{Phys.Rev.} (2015), arXiv:1411.0535.

\bibitem{Buchmueller:2014yoa} Oliver Buchmueller, Matthew J. Dolan, Sarah A. Malik and Christopher McCabe, \emph{Characterising dark matter searches at colliders and direct detection experiments: Vector mediators}, 2014, arXiv:1407.8257.

\bibitem{DiFranzo:2013vra} A.~DiFranzo, K. I.~Nagao, A.~Rajaraman, and T. M. P.~Tait
\emph{Simplified Models for Dark Matter Interacting with Quarks,}
\emph{JHEP} 11 (2013) 014, arXiv:1308.2679.

\bibitem{Buckley:2014fba} M. R.~Buckley, D.~Feld, and D.~Goncalves,
\emph{Scalar Simplified Models for Dark Matter,}
\emph{Phys. \ Rev.\ D} 91 (2015) 015017,
arXiv:1410.6497.

%Z"

\bibitem{Buckley:2013jwa} M. R. Buckley, \emph{Using Effective Operators to Understand CoGeNT and CDMS-Si}, \emph{Phys.Rev.} D 88, 055028 (2013), arXiv:1308.4146.

\bibitem{MonoX} N. Zhou, D. Berge, and D. Whiteson, \emph{Mono-everything: combined limits on dark matter production at colliders from multiple final states}, \emph{Phys.Rev.} D 87, 095013 (2013), arXiv:1302.3619.

\bibitem{SiM_gluons} R. M. Godbole, G. Mendiratta and T. M. P. Tait, \emph{A Simplified Model for Dark Matter Interacting Primarily with Gluons}, \emph{JHEP} 08 (2015) 064, arXiv:1506.01408.

\bibitem{Heisig:2015ira} J.~Heisig, M.~Kr�mer, M.~Pellen and C.~Wiebusch,
\emph{Constraints on Majorana Dark Matter from the LHC and IceCube,}
  arXiv:1509.07867. %[hep-ph].

\bibitem{Blennow:2015gta} M.~Blennow, J.~Herrero-Garcia, T.~Schwetz and S.~Vogl,
\emph{Halo-independent tests of dark matter direct detection signals: local DM density, LHC, and thermal freeze-out,}
  \emph{JCAP} 1508 (2015) 08,  039
  %doi:10.1088/1475-7516/2015/08/039
  arXiv:1505.05710. %[hep-ph]].

\bibitem{Lebedev:2014bba} O.~Lebedev and Y.~Mambrini,
\emph{Axial dark matter: The case for an invisible $Z?$,}
  \emph{Phys.\ Lett.\ B} 734 (2014) 350
  %doi:10.1016/j.physletb.2014.05.025
  arXiv:1403.4837. %[hep-ph]].

\bibitem{Alves:2015pea} A.~Alves, A.~Berlin, S.~Profumo and F.~S.~Queiroz,
\emph{Dark Matter Complementarity and the Z$^\prime$ Portal,}
  \emph{Phys.\ Rev.\ D} 92 (2015) 8,  083004
  %doi:10.1103/PhysRevD.92.083004
  arXiv:1501.03490. %[hep-ph]].

\bibitem{Alves:2013tqa} A.~Alves, S.~Profumo and F.~S.~Queiroz,
\emph{The dark $Z^{'}$ portal: direct, indirect and collider searches,}
  \emph{JHEP} 1404 (2014) 063
  %doi:10.1007/JHEP04(2014)063
  arXiv:1312.5281. %[hep-ph]].

\bibitem{Alves:2015mua} A.~Alves, A.~Berlin, S.~Profumo and F.~S.~Queiroz,
\emph{Dirac-fermionic dark matter in U(1)$_{X}$ models,}
  \emph{JHEP} 1510 (2015) 076
  %doi:10.1007/JHEP10(2015)076
  arXiv:1506.06767. %[hep-ph]].

\bibitem{An:2012va} H.~An, X.~Ji and L.~T.~Wang,
   \emph{Light Dark Matter and $Z'$ Dark Force at Colliders,}
    \emph{JHEP} 1207 (2012) 182, arXiv:1202.2894.

\bibitem{An:2012ue} H.~An, R.~Huo and L.~T.~Wang,
\emph{Searching for Low Mass Dark Portal at the LHC,}
  \emph{Phys.\ Dark Univ.\ } 2 (2013) 50
  %doi:10.1016/j.dark.2013.03.002
  arXiv:1212.2221. %[hep-ph]].

\bibitem{Frandsen:2012rk} M.~T.~Frandsen, F.~Kahlhoefer, A.~Preston, S.~Sarkar and K.~Schmidt-Hoberg,
\emph{LHC and Tevatron Bounds on the Dark Matter Direct Detection Cross-Section for Vector Mediators,}
  \emph{JHEP} 1207 (2012) 123
  %doi:10.1007/JHEP07(2012)123
  arXiv:1204.3839. %[hep-ph]].

\bibitem{Arcadi:2013qia} G.~Arcadi, Y.~Mambrini, M.~H.~G.~Tytgat and B.~Zaldivar,
\emph{Invisible $Z^\prime$ and dark matter: LHC vs LUX constraints,}
  \emph{JHEP} 1403 (2014) 134
  %doi:10.1007/JHEP03(2014)134
  arXiv:1401.0221. %[hep-ph]].

\bibitem{Shoemaker:2011vi} I.~M.~Shoemaker and L.~Vecchi,
\emph{Unitarity and Monojet Bounds on Models for DAMA, CoGeNT, and CRESST-II,}
  \emph{Phys.\ Rev.\ D} 86 (2012) 015023
  %doi:10.1103/PhysRevD.86.015023
  arXiv:1112.5457. %[hep-ph]].

\bibitem{Frandsen:2011cg} M.~T.~Frandsen, F.~Kahlhoefer, S.~Sarkar and K.~Schmidt-Hoberg,
\emph{Direct detection of dark matter in models with a light Z',}
  \emph{JHEP} 1109 (2011) 128
  %doi:10.1007/JHEP09(2011)128
  arXiv:1107.2118. %[hep-ph]].

\bibitem{Gondolo:2011eq} P.~Gondolo, P.~Ko and Y.~Omura,
\emph{Light dark matter in leptophobic Z' models,}
  \emph{Phys.\ Rev.\ D} 85 (2012) 035022
  %doi:10.1103/PhysRevD.85.035022
  arXiv:1106.0885. %[hep-ph]].

\bibitem{Fairbairn:2014aqa} M.~Fairbairn and J.~Heal,
\emph{Complementarity of dark matter searches at resonance,}
  \emph{Phys.\ Rev.\ D} 90 (2014) 11,  115019
  %doi:10.1103/PhysRevD.90.115019
  arXiv:1406.3288. %[hep-ph]].

  %t-channel

\bibitem{NordstromSVD} T. Jacques and K. Nordstrom, \emph{Mapping monojet constraints onto Simplified Dark Matter Models}, \emph{JHEP} 06 (2015) 142, arXiv:1502.05721.

\bibitem{Bell:2015rdw} N.~F.~Bell, Y.~Cai and R.~K.~Leane,
  \emph{Mono-W Dark Matter Signals at the LHC: Simplified Model Analysis,}
  \emph{JCAP}  1601 (2016) 01,  051, arXiv:1512.00476.

\bibitem{Chala:2015ama} M.~Chala, F.~Kahlhoefer, M.~McCullough, G.~Nardini and K.~Schmidt-Hoberg,
\emph{Constraining Dark Sectors with Monojets and Dijets,}
  \emph{JHEP} 1507 (2015) 089
  %doi:10.1007/JHEP07(2015)089
  arXiv:1503.05916. %[hep-ph]].

\bibitem{Kahlhoefer:2015bea} F.~Kahlhoefer, K.~Schmidt-Hoberg, T.~Schwetz and S.~Vogl,
\emph{Implications of unitarity and gauge invariance for simplified dark matter models,}
  \emph{JHEP}  1602 (2016) 016, arXiv:1510.02110

\bibitem{SUSYDM} G. Jungman et al., \emph{Supersymmetric dark matter}, \emph{Phys.Rept.} (1996).

\bibitem{Bai:2013iqa} Y.~Bai and J.~Berger,
  \emph{Fermion Portal Dark Matter,}
  \emph{JHEP} 1311 (2013) 171
  %doi:10.1007/JHEP11(2013)171
  arXiv:1308.0612. %[hep-ph]].

\bibitem{An:2013xka} H.~An, L.~T.~Wang and H.~Zhang,
  \emph{Dark matter with $t$-channel mediator: a simple step beyond contact interaction,}
  \emph{Phys.\ Rev.\ D} 89 (2014) 11,  115014
  %doi:10.1103/PhysRevD.89.115014
  arXiv:1308.0592. %[hep-ph]].

\bibitem{Chang:2013oia} S.~Chang, R.~Edezhath, J.~Hutchinson and M.~Luty,
  \emph{Effective WIMPs,}
  \emph{Phys.\ Rev.\ D} 89 (2014) 1,  015011
  %doi:10.1103/PhysRevD.89.015011
  arXiv:1307.8120. %[hep-ph]].

\bibitem{Zurek:tchannel} M. Papucci, A. Vichi, and K. M. Zurek, \emph{Monojet versus rest of the world I: t-channel Models}, \emph{JHEP} (2014), arXiv:1402.2285.

\bibitem{Garny:2015wea} M.~Garny, A.~Ibarra and S.~Vogl,
  \emph{Signatures of Majorana dark matter with t-channel mediators,}
  \emph{Int.\ J.\ Mod.\ Phys.\ D} 24 (2015) 07,  1530019
  %doi:10.1142/S0218271815300190
  arXiv:1503.01500. %[hep-ph]].

\bibitem{Garny:2014waa} M.~Garny, A.~Ibarra, S.~Rydbeck and S.~Vogl,
  \emph{Majorana Dark Matter with a Coloured Mediator: Collider vs Direct and Indirect Searches,}
  \emph{JHEP} 1406 (2014) 169
  %doi:10.1007/JHEP06(2014)169
  arXiv:1403.4634. %[hep-ph]].

\bibitem{Bell:2011if} N.~F.~Bell, J.~B.~Dent, A.~J.~Galea, T.~D.~Jacques, L.~M.~Krauss and T.~J.~Weiler,
  \emph{W/Z Bremsstrahlung as the Dominant Annihilation Channel for Dark Matter, Revisited,}
  \emph{Phys.\ Lett.\ B} 706 (2011) 6
  %doi:10.1016/j.physletb.2011.10.057
  arXiv:1104.3823. %[hep-ph]].

\bibitem{monoWZ_run2} ATLAS Collaboration, \emph{Search for dark matter produced in association with a hadronically decaying vector boson in pp collisions at $\sqrt{s}=$ 13 TeV with the ATLAS detector at the LHC}, ATLAS-CONF-2015-080 (2015).

\bibitem{Aad:2015zva} ATLAS Collaboration. \emph{Search for new phenomena in final states with an energetic jet and large missing transverse momentum in pp collisions at $\sqrt{s}=8$ TeV with the ATLAS detector}, 2015, arXiv:1502.01518

\bibitem{Aad:2014monoZlep} ATLAS Collaboration. \emph{Search for dark matter in events with a $Z$ boson and missing transverse momentum in $pp$ collisions at $\sqrt{s}$ = 8 TeV with the ATLAS detector}, \emph{Phys.Rev.D \textbf{90}} (2014) 012004, arXiv:1404.0051.

\bibitem{Aad:2013monoWZ} ATLAS Collaboration. \emph{Search for dark matter in events with a hadronically decaying W or Z boson and missing transverse momentum in $pp$ collisions at $\sqrt{s} =$ 8 TeV with the ATLAS detector}, \emph{Phys.\ Rev.\ Lett.\  {\bf 112}} (2014) 041802, arXiv:1309.4017.

\bibitem{MG_aMCNLO2014} J. Alwall /emph{et al.}. \emph{The automated computation of tree-level and next-to-leading order differential cross sections, and their matching to parton shower simulations}, \emph{JHEP07} (2014) 079, arXiv:1405.0301.

\bibitem{MSTW} A. D. Martin, W. J. Stirling, R. S. Thorne, G. Watt, \emph{Parton distributions for the LHC}, \emph{Eur.Phys.J.C63}, (2009), 189-285, arXiv:0901.0002.

\bibitem{pythia8} T. Sjöstrand et al. \emph{An Introduction to PYTHIA 8.2}, \emph{Comput. Phys. Comm.} 191 (2015) 159-177, arXiv:1410.3012.

\bibitem{AUtune} ATLAS Collaboration. \emph{Summary of ATLAS Pythia8 tunes}, 2012, ATL-PHYS-PUB-2012-003.

\bibitem{FastJet} M. Cacciari, G. P. Salam and G. Soyez, \emph{FastJet user manual}, arXiv:1111.6097.

\bibitem{massdrop} ATLAS Collaboration, \emph{Performance of jet substructure techniques for large-R jets in proton-proton collisions at $\sqrt{s}$ = 7 TeV using the ATLAS detector}, \emph{JHEP} 09 (2013) 076, arXiv:1306.4945.

\bibitem{MLMscheme} M. L. Mangano et al. \emph{Matching matrix elements and shower evolution for top-quark production in hadronic collisions}, \emph{JHEP} 01 (2007), hep-ph/0611129.

\bibitem{HistFitter} M. Baak et al. \emph{HistFitter software framework for statistical data analysis}, \emph{Eur.Phys.J.C} (2015), arXiv:1410.1280.

\bibitem{Bell:gaugeInv} N. F. Bell et al, \emph{Dark matter at the LHC: EFTs and gauge invariance}, \emph{Phys. Rev. D} 92 (2015) 053008, arXiv:1503.07874.

\bibitem{Chatrchyan:2013qha} CMS Collaboration, \emph{
Search for narrow resonances using the dijet mass spectrum in pp collisions at $\sqrt{s}$ = 8TeV}, \emph{Phys.Rev.} (2013), arXiv:1302.4794.

\bibitem{Aad:2014aqa} G.~Aad {\it et al.} [ATLAS Collaboration],
 \emph{Search for new phenomena in the dijet mass distribution using $p-p$ collision data at $\sqrt{s}=8$ TeV with the ATLAS detector,}
  \emph{Phys.\ Rev.\ D} 91 (2015) 5,  052007, arXiv:1407.1376.

\bibitem{Aaltonen:2008dn} T.~Aaltonen {\it et al.} [CDF Collaboration],
  \emph{Search for new particles decaying into dijets in proton-antiproton collisions at s**(1/2) = 1.96-TeV,}
  \emph{Phys.\ Rev.\ D}  79 (2009) 112002,
  arXiv:0812.4036

\bibitem{Khachatryan:2015sja} V.~Khachatryan {\it et al.} [CMS Collaboration],
  \emph{Search for resonances and quantum black holes using dijet mass spectra in proton-proton collisions at $\sqrt{s} =$ 8 TeV,}
  \emph{Phys.\ Rev.\ D}   91 (2015) 5,  052009,
  arXiv:1501.04198

\bibitem{Ade:2013zuv} P.~A.~R.~Ade {\it et al.}  [Planck Collaboration],
\emph{Planck 2013 results. XVI. Cosmological parameters,}
  Astron.\ Astrophys.\  {\bf 571}, A16 (2014)
  [arXiv:1303.5076 [astro-ph.CO]].

\bibitem{Belanger:2014vza} G.~B�langer, F.~Boudjema, A.~Pukhov and A.~Semenov,
\emph{micrOMEGAs4.1: two dark matter candidates,}
  Comput.\ Phys.\ Commun.\  {\bf 192}, 322 (2015)
  %doi:10.1016/j.cpc.2015.03.003
  [arXiv:1407.6129 [hep-ph]].
  %%CITATION = doi:10.1016/j.cpc.2015.03.003;%%
  %53 citations counted in INSPIRE as of 15 f�vr. 2016
%Note sure if a better reference exists?

\bibitem{Busoni:2014gta} G. Busoni, A. De Simone, T. Jacques, E. Morgante and A. Riotto, \emph{Making the Most of the Relic Density for Dark Matter Searches at the LHC 14 TeV Run}, \emph{JCAP} 03 (2015) 022, arXiv:1410.7409.

\bibitem{DelNobile:2013sia} M.~Cirelli, E.~Del Nobile and P.~Panci,
  \emph{Tools for model-independent bounds in direct dark matter searches,}
  \emph{JCAP}1310 (2013) 019, arXiv:1307.5955.

\bibitem{Akerib:2013tjd} D.~S.~Akerib {\it et al.} [LUX Collaboration], \emph{First results from the LUX dark matter experiment at the Sanford Underground Research Facility,}
  \emph{Phys.\ Rev.\ Lett.} 112 (2014) 091303, arXiv:1310.8214.

\bibitem{CERN-THESIS-2015-038} S. Schramm, \emph{Searching for Dark Matter with the ATLAS Detector in Events with an Energetic Jet and Large Missing Transverse Momentum}, 2015, CERN-THESIS-2015-038.

\bibitem{MonoZ} N. Bell et al., \emph{Searching for Dark Matter at the LHC with a Mono-Z}, \emph{Phys.Rev.} D 86, 096011 (2012), arXiv:1209.0231.

\bibitem{ComppMSSM} M. Cahill-Rowley et al., \emph{Complementarity and Searches for Dark Matter in the pMSSM}, SLAC-PUB-15450 (2013), arXiv:1305.6921.

\bibitem{Kumar} J. Kumar and D. Marfatia. \emph{Matrix element analyses of dark matter scattering and annihilation}, \emph{Phys.Rev.} (2013), arXiv:1305.1611.

\bibitem{METSig} P. J. Fox et al., \emph{Missing Energy Signatures of Dark Matter at the LHC}, \emph{Phys.Rev.} (2012), arXiv:1109.4398.

\bibitem{Fox:2012ee} P. J. Fox, R. Harnik, R. Primulando, and C-T. Yu, \emph{Taking a Razor to Dark Matter Parameter Space at the LHC}, \emph{Phys.Rev.} (2012), arXiv:1203.1662.

\bibitem{Aad:2012hf} ATLAS Collaboration, \emph{Search for high-mass resonances decaying to dilepton final states in pp collisions at s**(1/2) = 7-TeV with the ATLAS detector}, \emph{JHEP} (2012), arXiv:1209.2535.

\bibitem{Kumar:2013iva} J. Kumar and D. Marfatia, \emph{Matrix element analyses of dark matter scattering and annihilation}, \emph{Phys.Rev.} (2013), arXiv:1305.1611.

\bibitem{Alves:2011wf} D. Alves et al., \emph{Simplified Models for LHC New Physics Searches}, \emph{J.Phys.} (2012), arXiv:1105.2838.

\bibitem{CMS:rwa} CMS Collaboration. \emph{Search for new physics in monojet events in pp collisions at $\sqrt{s} =$ 8 TeV}, 2013, CMS-PAS-EXO-12-048.

\bibitem{ATLAS:2012zim} ATLAS Collaboration. \emph{Search for New Phenomena in Monojet plus Missing Transverse Momentum Final States using 10 fb$^{1}$ of pp collisions at $\sqrt{s} =$ 8 TeV with the ATLAS detector at the LHC}, 2012, ATLAS-CONF-2012-147.

\bibitem{CooperSarkar:2011vp} A. Cooper-Sarkar. \emph{PDFs for the LHC}, 2011, arXiv:1107.5170.

\bibitem{ATLAS:2012ky} ATLAS Collaboration. \emph{Search for dark matter candidates and large extra dimensions in events with a jet and missing transverse momentum with the ATLAS detector}, 2013, CERN-PH-EP-2012-210, arXiV:1210.4491.

\bibitem{PJFox} P. J. Fox et al. \emph{Missing Energy Signatures of Dark Matter at the LHC}, \emph{Phys. Rev.}, 2012.

\bibitem{NBellDent} N. Bell, J. Dent, T. Jacques, and T. Weiler. \emph{W/Z Bremsstrahlung as the Dominant Annihilation Channel for Dark Matter}, \emph{Phys. Rev.}, 2011.

\end{thebibliography}
\end{document}
