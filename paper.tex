\documentclass[a4paper,11pt]{article}
\pdfoutput=1 % if your are submitting a pdflatex (i.e. if you have
             % images in pdf, png or jpg format)

\usepackage{jheppub} % for details on the use of the package, please
                     % see the JHEP-author-manual

\usepackage[T1]{fontenc} % if needed
%\usepackage[utf8]{inputenc}


\title{\boldmath Constraints on Simplified Dark Matter Models using Mono-X Collider Searches}


%% %simple case: 2 authors, same institution
%% \author{A. Uthor}
%% \author{and A. Nother Author}
%% \affiliation{Institution,\\Address, Country}

% more complex case: 4 authors, 3 institutions, 2 footnotes
\author[a,1]{Amelia J. Brennan,\note{Corresponding author.}}
\author[b]{Johanna Gramling,}
\author[b]{Thomas Jacques,}
\author[a]{and Millie F. McDonald}

% The "\note" macro will give a warning: "Ignoring empty anchor..."
% you can safely ignore it.

\affiliation[a]{The University of Melbourne, Parkville 3010, Australia}
\affiliation[b]{Universit\'{e} de Gen\`{e}ve, Quai E. Ansermet 24, 1211 Gen\`{e}ve 4, Switzerland}

% e-mail addresses: one for each author, in the same order as the authors
\emailAdd{a.brennan@student.unimelb.edu.au}
\emailAdd{johanna.gramling@cern.ch}
\emailAdd{thomas.jacques@unige.ch}
\emailAdd{m.mcdonald2@student.unimelb.edu.au}




\abstract{Abstract...}



\begin{document} 
\maketitle
\flushbottom

% Keep this section clean - create a new input tex file in sections/ if a new section is required, and add it here.
\section{Checking some latex commands}
\label{sec:intro}
% !TEX root = ../paper.tex

The first simplified model considered here comprises a Dirac fermion DM particle (labeled $\chi$), coupled to SM quarks via an $s$-channel vector mediator (labeled $\xi$). The couplings of the mediator to the quark pair and DM pair are denoted $g_q$ and $g_{\chi}$ respectively, and are assumed to be equal, that is, $g_q = g_{\chi}$. The width of the vector mediator is variable by hand, and in the following study three indicative widths will be used.



\section{Introduction} 
\label{sec:sec1}
%\input{sections/nextsection.tex}
\begin{flushleft}
Include:
\begin{enumerate}
\item Motivation for SiMs: limited validity of EFTs \& uneconomically broad parameter space of full UV-complete models/insensitivity of model-dependent results to the wider class of DM models $\rightarrow$ advantage of simplified models (relatively small parameter space, predictions accurate and complementary
across a broad energy range $\mathcal{O}$(GeV-TeV)).\footnote{Are we assuming that the reader understands the basics of a "simplified model"? Or should we have one or two sentences that summarise the differences between, say, an EFT and a SiM?}
\item Brief description of the aim of the analysis (e.g. something like "to apply constraints to $m_{\chi}$, $m_{M}$ and $g_{q} = g_{\chi} = g$ for four phenomenologically distinct simplified models, etc) along with a brief motivation for the choice of simplified models\footnote{Or should this part be in section \ref{sec:sec2}?}.
\item Summary of content of paper (eg. "This paper is organised as follows: section 2 is a compendium of the simplified models chosen for analysis..." etc)
\end{enumerate}
\end{flushleft}

\section{Models} 
\label{sec:sec2}
\begin{flushleft}
Include:
\begin{enumerate}
\item Brief motivation for choice of simplified models? (eg. something like "we consider the most straightforward UV-completions of the D1, D5 and D8 effective operators, corresponding to the s-channel scalar, vector and axial-vector models respectively. The D1, D5 and D8 operators form a nice starting point as they have been studied exhaustively in the past (see [ref]). This attention was motivated by the complementarity of these effective operators whereby collider limits
for the D1 and D5 (D8) operators were readily transformed into limits for spin-independent (spin-dependent) $\chi$-nucleon scattering and vice versa. With the exception of D9 and D11, the remaining effective operators induce elastic scattering which is suppressed by powers of the DM velocity or the momentum transferred [ref]. The D9 and D11 operators have no simple simplified model counterparts [ref].")
\item The interaction Lagrangians for our four SiMs along with an explanation for why we only assume coupling to SM quarks. 
\item The assumptions and the decay widths associated with our models (?).
\end{enumerate}
\end{flushleft}

\section{(Something like) Reinterpreting Mono-X Contraints} 
\label{sec:sec3}

\section{Results} 
\label{sec:sec4}

\section{Conclusion} 
\label{sec:sec5}

\section{Acknowledgements} 
\label{sec:sec6}

\begin{thebibliography}{99}

\bibitem{a}
Author, \emph{Title}, \emph{J. Abbrev.} {\bf vol} (year) pg.

\bibitem{b}
Author, \emph{Title},
arxiv:1234.5678.

\bibitem{c}
Author, \emph{Title},
Publisher (year).

% Please avoid comments such as "For a review'', "For some examples",
% "and references therein" or move them in the text. In general,
% please leave only references in the bibliography and move all
% accessory text in footnotes.

% Also, please have only one work for each \bibitem.

\end{thebibliography}
\end{document}
