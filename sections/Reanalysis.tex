The \monoX + $\met$ (abbreviated \monoX) signal is a popular collider signal in the search for new physics, particularly in the search for dark matter. Since WIMPs are not expected to interact with detector material, they appear as missing transverse momentum, $\vec{p}_{\mathrm{T}}^{\mathrm{miss}}$, when balanced against a visible object that is radiated from the initial or intermediate state.
%\st{It is predicted by an assorted collection of new physics models, including unparticle physics models }\cite{CMS-PAS-EXO-12-048}\st{ and the Arkani-Hamed, Dimopoulos, and Dvali (ADD) model of large extra spatial dimensions }\cite{ATLAS-CONF-2012-147}.
For the $s$-channel simplified models discussed in Section \ref{sec:sec2}, a SM particle, X, is emitted from one of a pair of intial-state partons (shown in Figure \ref{schannel_sig_phen}). The case where X is radiated from the mediator - a process known as virtual internal Bremsstrahlung - is only possible if the SM-dark matter interaction occurs via the $t$-channel (as shown in Figure \ref{tchannel_sig_phen}).
%\st{In both cases, the parton and Z boson are the SM particles most likely to be emitted \question{is Pr(W)$>$Pr(Z)?}}\cite{Kumar:2013iva}. \st{Hence, this paper will focus on thre detection channels: \monojet, \monoZ and mono-$W$.}
For all models, emission of a parton is the most likely scenario at the LHC owing to the strength of the strong coupling. Hence we focus on the \monojet channel as it is expected to provide the strongest limits. Emission of $Z$ and $W$ bosons or photons is also possible however, and may be chosen for study over jet processes to take advantage of the relative simplicity of leptons compared to jets. As such, we also include the mono-$Z(\rightarrow \ell^+ \ell^-)$ channel for comparison. Finally, we extend this work by including the hadronically-decaying \monoWZ channel for comparison.

%previously had a footnote: It is worth noting here that the mono-$W$ channel has previously been chosen for its apparent ability to distinguish a case where $u$- and $d$-type quarks couple with opposite sign to the new physics sector \cite{}\comm{(Tait)}. However, during the preparation of this paper this has been shown to be an unphysical scenario \cite{}\comm{(Bell, Leane)}


%, which may then enhance the limit in combination with monojet processes.}
%\bigskip

The procedure for recasting existing \monoX constraints as simplified model constraints is straightforward. Firstly, signal events are simulated as described in Section \ref{signal_generation}. The event selection criteria of the \monoX analysis of interest is then reproduced and applied to the simulated signal samples. Events surviving the selection criteria are counted to determine both the likelihood of a dark matter event occurring (referred to as the acceptance, $\mathcal{A}$) and the probability of detecting said event (refered to as the efficiency, $\epsilon$). These quantities are then used in combination with channel-specific model-independent limits on new physics events to limit the parameter phase space of a given model.
For a comprehensive description of the recasting procedure, see appendix \ref{Appendix_limitsetting}.
%\bigskip

In this paper, \monojet constraints are derived from a search for new phenomena conducted by the ATLAS Collaboration using $pp$ collisions at $\sqrt{s}=$ 8 TeV as described in Ref. \cite{Aad:2015zva}. Similarly, the leptonic \monoZ and hadronic \monoWZ constraints are derived from ATLAS dark matter searches originally optimised for the D1, D5 and D9 effective operators \cite{Aad:2014monoZlep, ATLASmonoWZ}. These analyses are described in further detail in Sections \ref{monojet_constraints}, \ref{monoZ_constraints} and \ref{monoWZ_constraints} respectively.

%\comm{We should put references to all the papers here! - Amelia}

% Note for figure: putting spaces between the \end{subfig} and the next begin{subfig} leads to the figures stacking rather than sitting side-by-side, don't want this!
% Also, have deliberately set TSD plots to width = 0.4, with horizontal space in the middle, to get the relative sizes of the figures looking correct.
\begin{figure}[t]
  \centering
  \begin{subfigure}[b]{0.45\textwidth}
    \centering
    \resizebox{\linewidth}{!}{
      \begin{tikzpicture}
        \draw[fermion] (-1.5,1.5)node[left]{$q$} --(-0.75,0.75);
        %\draw[dashed] (-0.75,0.75) -- (0,1.5)node[right]{$j$}; %\draw[dashed] (-0.75,0.75) --  (0,1.5)node[right]{X};
        \draw[gluon] (-0.75,0.75) -- (0,1.5)node[right]{$g$};
        \draw[fermion] (-0.75,0.75) -- (0,0);
        \draw[antifermion] (-1.5,-1.5)node[left]{$\bar{q}$} --(0,0);
        \draw[fill] (0,0) circle [radius=0.0]node[left]{$\gq\mbox{ }$};
        \draw[photon] (0,0) --node[above]{$\xi$} (2,0);
        \draw[fermion] (2,0) -- (3.5,1.5)node[right]{$\chi$};
        \draw[antifermion] (2,0) --(3.5,-1.5)node[right]{$\bar{\chi}$};
        \draw[fill] (2,0) circle [radius=0.0]node[right]{$\mbox{ }\gX$};
      \end{tikzpicture}
    }
    \caption{}
    %\label{Signal_phen_sa}
  \end{subfigure}
  \begin{subfigure}[b]{0.45\textwidth}
    \centering
    \resizebox{\linewidth}{!}{
      \begin{tikzpicture}
        \draw[fermion] (-1.5,1.5)node[left]{$q$} --(-0.75,0.75);
        \draw[photon] (-0.75,0.75) -- (0,1.5)node[right]{$W/Z$}; %\draw[dashed] (-0.75,0.75) -- (0,1.5)node[right]{X};
        \draw[fermion] (-0.75,0.75) -- (0,0);
        \draw[antifermion] (-1.5,-1.5)node[left]{$\bar{q}$} --(0,0);
        \draw[fill] (0,0) circle [radius=0.0]node[left]{$\gq\mbox{ }$};
        \draw[photon] (0,0) --node[above]{$\xi$} (2,0);
        \draw[fermion] (2,0) -- (3.5,1.5)node[right]{$\chi$};
        \draw[antifermion] (2,0) --(3.5,-1.5)node[right]{$\bar{\chi}$};
        \draw[fill] (2,0) circle [radius=0.0]node[right]{$\mbox{ }\gX$};
      \end{tikzpicture}
    }
    \caption{}
    %\label{Signal_phen_se}
  \end{subfigure}
  \begin{subfigure}[b]{0.4\textwidth}
    \centering
    \resizebox{\linewidth}{!}{
      \begin{tikzpicture}
        \draw[fermion] (-2,1.)node[left]{$q$} --(-1,1.);
        \draw[fermion] (-1,1.) --(0,1.);
        \draw[gluon] (-1,1.) --(0,2.)node[right]{$g$};
        \draw[antifermion] (-2,-1)node[left]{$\bar{q}$} --(0,-1);
        \draw[fill] (0,1.) circle [radius=0.0]node[above]{$\gqX$};
        \draw[dashed] (0,1.) -- node[left]{$\phi_{q}$}(0,-1);
        \draw[fermion] (0,1.) -- (2,1.)node[right]{$\chi$};
        \draw[antifermion] (0,-1) --(2,-1)node[right]{$\bar{\chi}$};
        \draw[fill] (0,-1) circle [radius=0.0]node[below]{$\gqX$};
      \end{tikzpicture}
    }
    \caption{}
    %\label{Signal_phen_tf}
  \end{subfigure}
  \hspace{1cm}
  \begin{subfigure}[b]{0.4\textwidth}
    \centering
    \resizebox{\linewidth}{!}{
      \begin{tikzpicture}
        \draw[fermion] (-2,1.)node[left]{$q$} --(0,1.);
        \draw[antifermion] (-2,-1)node[left]{$\bar{q}'$} --(0,-1);
        \draw[fill] (0,1.) circle [radius=0.0]node[above]{$\gqX$};
        \draw[dashed] (0,1.) -- node[left]{$\phi_{q}$}(0,0.25);
        \draw[photon] (0,0.) -- (1.5, 0.)node[right]{$W/Z$};
        \draw[dashed] (0,0.) -- node[left]{$\phi_{q'}$}(0,-1);
        \draw[fermion] (0,1.) -- (2,1.)node[right]{$\chi$};
        \draw[antifermion] (0,-1) --(2,-1)node[right]{$\bar{\chi}$};
        \draw[fill] (0,-1) circle [radius=0.0]node[below]{$\gqX$};
      \end{tikzpicture}
    }
    \caption{}
    %\label{Signal_phen_tg}
  \end{subfigure}
  \begin{subfigure}[b]{0.45\textwidth}
    \centering
    \begin{tikzpicture}
      \draw[fermion] (-1.5,1.5)node[left]{$q$} --(0,0);
      \draw[gluon] (-1.5,-1.5)node[left]{$g$} --(0,0);
      %\draw[fill] (0,0) circle [radius=0.05]node[left]{$g_{q'}\mbox{ }$};
      \draw[fill] (0,0) circle [radius=0.0]node[left]{$\mbox{ }$};
      \draw[fermion] (0,0) --node[above]{$q$} (2,0);
      \draw[photon] (2,0) -- node[above]{$\xi\mbox{\textcolor{white}{m}}$}(2.75,0.75);
      \draw[fermion] (2.75,0.75) -- (3.5,1.5)node[right]{$\chi$};
      \draw[antifermion] (2.75,0.75) --(3.5,0)node[right]{$\bar{\chi}$};
      \draw[fill] (2,0) circle [radius=0.0]node[below]{$\gq\mbox{ }$};
      \draw[fermion] (2,0) --(3.5,-1.5)node[right]{$q$};
      \draw[fill] (2.75,0.75) circle [radius=0.0]node[right]{$\mbox{\textcolor{white}{\tiny m}}\gX$};
    \end{tikzpicture}
    \caption{}
    \label{Signal_phen_sd}
  \end{subfigure}
  \begin{subfigure}[b]{0.45\textwidth}
    \centering
    \begin{tikzpicture}
      \draw[gluon] (-1.5,1.5)node[left]{$g$} --(0,0);
      \draw[gluon] (-1.5,-1.5)node[left]{$g$} --(0,0);
      %\draw[fill] (0,0) circle [radius=0.05]node[left]{$g_{q'}\mbox{ }$};
      \draw[fill] (0,0) circle [radius=0.0]node[left]{$\mbox{ }$};
      \draw[gluon] (0,0) -- (1,0);
      \draw[antifermion] (1,0) -- node[above]{$\bar{q}\mbox{\textcolor{white}{m}}$}(1.75,0.75);
      \draw[photon] (1.75,0.75) -- node[above]{$\xi\mbox{\textcolor{white}{m}}$}(2.5,1.5);
      \draw[fermion] (2.5, 1.5) -- (3.25, 2.25)node[right]{$\chi$};
      \draw[antifermion] (2.5, 1.5) -- (3.25, 0.75)node[right]{$\bar{\chi}$};
      \draw[antifermion] (1.75,0.75) --(2.5,0)node[right]{$\bar{q}$};
      %\draw[fill] (2,0) circle [radius=0.0]node[below]{$\gq\mbox{ }$};
      \draw[fermion] (1,0) --(2.5,-1.5)node[right]{$q$};
      \draw[fill] (1.75,0.75) circle [radius=0.0]node[right]{$\mbox{\textcolor{white}{\tiny m}}\gq$};
      \draw[fill] (2.5,1.5) circle [radius=0.0]node[right]{$\mbox{\textcolor{white}{\tiny m}}\gX$};
    \end{tikzpicture}
    \caption{}
    \label{Signal_phen_sd}
  \end{subfigure}
  \begin{subfigure}[b]{0.4\textwidth}
    \centering
    \resizebox{\linewidth}{!}{
      \begin{tikzpicture}
        \draw[fermion] (-2,1.)node[left]{$q$} --(0,1.);
        \draw[antifermion] (-2,-1)node[left]{$\bar{q}$} --(0,-1);
        \draw[dashed] (0,1.) -- (0,-1);
        \draw[fermion] (0,1.) -- (2,1.)node[right]{$\chi$};
        \draw[antifermion] (0,-1) --(2,-1)node[right]{$\bar{\chi}$};
      \end{tikzpicture}
    }
    \caption{}
    %\label{Signal_phen_tg}
  \end{subfigure}
  \begin{subfigure}[b]{0.4\textwidth}
    \centering
    \resizebox{\linewidth}{!}{
      \begin{tikzpicture}
        \draw[fermion] (-1.5,1.5)node[left]{$q$} --(0,0);
        \draw[antifermion] (-1.5,-1.5)node[left]{$\bar{q}$} --(0,0);
        \draw[dashed] (0,0) -- (2,0);
        \draw[fermion] (2,0) --(3.5,1.5)node[right]{$\chi$};
        \draw[antifermion] (2,0) --(3.5, -1.5)node[right]{$\bar{\chi}$};
      \end{tikzpicture}
    }
    \caption{}
    %\label{Signal_phen_tg}
  \end{subfigure}
  \caption{Representative dark matter pair-production processes with a gluon or a $W$ or $Z$ boson in the final state for the $s$-channel (a,b) and $t$-channel (c,d) models.}
  \label{allchannel_sig_phen}
\end{figure}

\subsection{Signal Simulation}
\label{signal_generation}
Signal samples for each channel and for each simplified model discussed in Section \ref{sec:sec2} were generated in the following manner. Firstly, leading order matrix elements for the process $pp \rightarrow \mathrm{X} + \chi\bar{\chi}$ (where X is either one or two jets\footnote{For the \monojet channel, jets are seeded by any parton excluding the (anti-)top quark.}, a $Z(\rightarrow \ell^+ \ell^-)$ boson or a $W/Z(\rightarrow$ jets) boson) were modelled using \MG$\_${\footnotesize A}MC$@$NLO v2.2.2 \cite{MG_aMCNLO2014} with the PDF MSTW2008lo68cl \cite{MSTW}. The default renormalisation and factorisation scales were also used and set to the sum of $\sqrt{m^{2} + p_{T}^{2}}$ for all particles in the final state. Showering and hadronisation were then performed by \PYTHIA  8.201 with the appropriate PDF and using the ATLAS UE Tune AU2-MSTW2008LO~\cite{AUtune}. The detector response was approximated by applying a gaussian smearing to the $p_{\mathrm{T}}$ of the leptons and jets. FastJet \ref{} was used to reconstruct small-radius jets (anti-kT algorithm with $R$ = 0.4) for the \monojet channel, and large-radius jets (Cambridge-Aachen algorithm with $R$ = 1.2) for the \monoWZ channel; the latter also uses a mass-drop filtering procedure as discussed at ref.~\ref{}, with $\mu$ = 0.67 and $y$ = 0.16.

\subsubsection{Parton Matching Scheme}
\label{matching_procedure}
%\comm{Make explicit the choice of $(0+1+2)j$ vs $(1+2)j$ here.}
%\comm{Also refer to the Papucci paper, and why we don't split the sample. Do we think this is still necessary?}
For the \monojet channel, matching of partons generated in \MG$\mbox{ }$ to jets generated in \PYTHIA is performed using the MLM scheme, with a single matching scale (known as the QCUT). The use of a single matching scale initially seems problematic as the choice of QCUT can influence somewhat the distributions of $p_{\mathrm{T}}$ and $\met$. In particular, it leads to increased uncertainty in the case where the mediator mass is significantly larger that the QCUT value, due to the resulting lack of statistics. The ATLAS \monojet analysis attempts to mitigate this effect with the creation of two subsamples, with different QCUT values, and merging these with a cut on the leading jet $p_{\mathrm{T}}$ to avoid double-counting. However, we found that use of a single QCUT value at 80 GeV was able to adequately reproduce the results of the ATLAS \monojet analysis for the masses of interest, while substantially reducing both the complexity and computational expense of the \monojet channel MC generation and systematic uncertainty estimation procedures (see section \ref{monojet_validation}).

\bigskip
We now move to a discussion of each of the \monoX channels separately.

\subsection{Monojet Constraints}
\label{monojet_constraints}
The ATLAS \monojet plus missing transverse energy search \cite{Aad:2015zva} was originally designed to set limits on three new physics scenarios\comm{,}
%\st{: the production of light grativinos in association with gluinos or scalar quarks in a gauge-mediated supersymmetric model, the production of graviton modes in the Arkani-Hamed, Dimopoulos, and Dvali model for large extra spatial dimensions, and}
the most relevant of which is the production of WIMP DM within the context of seven \comm{(?)} effective operators. The analysis also includes a brief study of a $Z'$ DM model which is analogous to our sV model.
%\bigskip

Signal selection is carried out based on at least one hard jet recoiling against missing energy. To ensure that the correct back-to-back jet + $\met$ topology is selected events are required to have a leading jet, $j_{1}$, with $p_{T} >$ 120 GeV and $|\eta| <$ 2.0 satisfying $p_{T}^{j_{1}}/\met >$ 0.5. Surviving events must then satisfy $|\Delta\phi(j,\metvec)|>1.0$, where $j$ is any jet with $p_{T} >$ 30 GeV and $|\eta| <$ 4.5. This criterion reduces the multijet background contribution where the large $\met$ originates mainly from jet energy mismeasurement. Note that there is no upper limit placed on the number of jets per event. The contribution from the dominant background processes, $W/Z+$jets (\footnote{Do I want to be more specific here? Eg. $W (\rightarrow \ell \nu)+$jets, $Z(\rightarrow \nu\bar{\nu})+$jets, $Z/\gamma^{*}(\rightarrow \ell^{+}\ell^{-})+$jets?}), is managed with a veto on events containing muons or electrons with $p_{T}>$ 7 GeV. A further veto is placed on events containing isolated tracks\footnote{A track is considered isolated when no additional track with $p_{T} >$ 3 GeV lies within a cone of radius 0.4 around it.} with $p_{T}>$ 10 GeV and $|\eta| <$ 2.5. This reduces the contribution from non-identified leptons ($e$, $\mu$ or $\tau$) in the final state. Lastly, nine separate signal regions are defined with increasing lower thresholds on $\met$, which range from 150 GeV to 700 GeV as shown in Table \ref{monojet_SRs}.
%\st{initially, events are required to have $\met>$ 150 GeV and at least one jet with $p_{T} >$ 30 GeV and $|\eta| <$ 4.5} (\comm{My logic here is that these cuts are for trigger, or pre-selection - they are overwritten by later cuts, right? So I think not worth including.}). we require $|\Delta\phi(\mathrm{j},\metvec)|>1.0$, where j is any jet with $p_{T} >$ 30 GeV and $|\eta| <$ 4.5.  have a leading jet (\comm{Is this the dominant background? If so, include `dominant', if not, this cut is maybe better understood as implementing back-to-backness. Is there an upper limit on $N_{jets}$?}). \comm{Note to self: include lepton veto criteria here.}
%\bigskip

The ATLAS \monojet analysis revealed no significant deviation of observed events from the expected SM backgrounds in the Run 1 8 TeV dataset. Subsequently, limits on new physics signatures were derived in terms of the visible cross-section, $\sigma\times\mathcal{A}\times\epsilon$, using the HistFitter package \cite{}. These model-independent limits are shown in Table \ref{monojet_SRs} and correspond to the 95\% confidence level.

\begin{table}[!htbp]
\centering
\begin{tabular}{c|c|c}
 \hline
 \hline
 Signal Region & $\met$ threshold [GeV] & $\sigma \times \mathcal{A} \times \epsilon$ [fb] \\ %$N_{obs}$ & $N_{exp}$ \\%& $N_{obs}/\mathcal{L}$ & $N_{exp}/\mathcal{L}$ \\
 \hline
 SR1 & 150 & 726 (935) \\ %14737.8 & 18980.5 \\%726 & 935 \\
 SR2 & 200 & 194 (271) \\ %3938.2 & 5501.3 \\%194 & 271 \\
 SR3 & 250 & 90 (106) \\ %1827 & 2151.8 \\%90 & 106 \\
 SR4 & 300 & 45 (51) \\ %913.5 & 1035.3 \\%45 & 51 \\
 SR5 & 350 & 21 (29) \\ %426.3 & 588.7 \\%21 & 29 \\
 SR6 & 400 & 12 (17) \\ %243.6 & 345.1 \\%12 & 17 \\
 SR7 & 500 & 7.2 (7.2) \\ %146.16 & 146.16 \\%7.2 & 7.2 \\
 SR8 & 600 & 3.8 (3.2) \\ %77.14 & 73.08 \\%3.8 & 3.6 \\
 SR9 & 700 & 3.4 (1.8) \\ %69.02 & 36.54 \\%3.4 & 1.8 \\
 \hline
 \hline
\end{tabular}
\caption{The ATLAS \monojet $\met$ signal regions and corresponding observed (expected) model-independent upper limits on $\sigma \times \mathcal{A} \times \epsilon$ at 95\% confidence level. Adapted from Ref. \cite{Aad:2015zva}.
%The ATLAS model-independent upper limits on the number of observed (expected) signal events, $N_{obs}$ ($N_{exp}$), at 95\% confidence level for the \monojet channel signal regions.
%\comm{Since this table is adapted from the \monojet paper, I think we should reproduce exactly what they show, ie the limit on $\sigma \times \mathcal{A} \times \epsilon$. Part of the reason is that if you multiply their values by the lumi, this is ignoring the uncertainty on that lumi that they've already included. More importantly, I don't think there is any advantage in showing the limits on $N$, since to obtain limits on $\sigma$ we divide immediately by $\mathcal{L}$ again anyway, plus the audience may be more familiar with the limit on $\sigma \times \mathcal{A} \times \epsilon$.}
}
\label{monojet_SRs}
\end{table}

The Monte Carlo (MC) generation and event selection procedures discussed above were validated for the \monojet channel via reproduction of ATLAS limits on the suppression scale, $\Mstar \equiv \Mmed / \sqrtgqgX$, for the $Z'$ model. The details of this process are contained in appendix \ref{monojet_validation}. Importantly, we observe agreement within $\sim$23\% for all samples.

%\st{Note that we only calculate limits on M$_{*}$ for this aspect of the analysis. Although it is customary to present constraints on dark matter models in the form of limits on M$_{*}$, we shall hereafter present constraints in the form of limits on the cross-section. This is done to better facilitate the comparison of collider and direct detection results, where $\sigma(pp \rightarrow X + \chi\bar{\chi})$ and $\sigma(N\chi \rightarrow N\chi)$ are related by a Fierz transformation in the simplified model framework} \cite{PJFox, NBellDent}.

%\draft{While the signal samples used by the ATLAS group (and for validation) correspond to the processes $pp \rightarrow j\chi\bar{\chi}$ and $pp \rightarrow jj\chi\bar{\chi}$ where $j$ is a final state jet, \monojet constraints on the simplified models studied in this paper were set using signal samples with zero, one and two jets in the final state.} \comm{This deviation from the ATLAS analysis is motivated by the observation that the limits on $\Mstar$ are improved with the addition of the process $pp \rightarrow \chi\bar{\chi}$.}
%However, while the ATLAS MC samples are generated for only two matching scale values (80 GeV and 300 GeV), we use a sliding value of $\mX/4$. \comm{This ensures that all jet-related kinematic distributions are smoothly connected across all dark matter masspoints.}

\subsection{Mono-$Z$ Constraints}
\label{monoZ_constraints}
The signature of the ATLAS mono-$Z(\rightarrow \ell^+ \ell^-)$ analysis \cite{Aad:2014monoZlep} is a pair of opposite-sign same-flavour leptons balanced against a large amount of missing transverse momentum. The analysis is designed to search for a set of EFT models of DM, where a $Z$ boson is radiated from an initial state quark. Leptons are in general much cleaner and simpler than jets, so this channel is included here to investigate whether the reduction in systematic uncertainties can provide easily-obtained results that are comparable to the more complicated \monojet channel.

The analysis also includes a short study of a $t$-channel simplified model similar to that discussed here. This model is used to validate our results in this channel; see the details in sec.~\ref{monoZ_validation}.

The selection is summarised as follows (see the paper for a full description). Electrons (muons) are required to have a $p_{\mathrm{T}}$ greater than 20 GeV, and $|\eta|$ less than 2.47 (2.5). Two opposite-sign, same-flavour leptons are selected, and required to have invariant mass and pseudorapidity such that $m_{\ell \ell} \in [76, 106]$ GeV and $|\eta^{\ell \ell}| < 2.5$. The reconstructed $Z$ boson should be approximately back-to-back and balanced against the $\met$, ensured with the selections $\Delta \phi (\metvec, p_{\mathrm{T}}^{\ell \ell}) > 2.5$ and $| p_{\mathrm{T}}^{\ell \ell} - \met | \, /  \, p_{\mathrm{T}}^{\ell \ell} < 0.5$. Jets are reconstructed with the anti-$k_t$ algorithm, with radius parameter 0.4; events containing a jet with $p_{\mathrm{T}}>$ 25 GeV and $|\eta|< $ 2.5 are vetoed. Events are also vetoed if they contain a third lepton with $p_{\mathrm{T}}>$ 7 GeV. The signal regions are defined by increasing lower $\met$ thresholds: $\met >$ 150, 250, 350, 450 GeV.

% Note: I haven't included any info on the overlap removal here.

The dominant background in this analysis is the irreducible $ZZ \rightarrow \ell^+ \ell^- \bar{\nu} \nu$ process, which has a softer $\met$ distribution that the DM signal. The background is estimated with MC simulation, and has a systematic uncertainty in the range 36-46$\%$ across the four signal regions.

A cut-and-count strategy is used, and the total numbers of expected and observed events, along with total uncertainties, are reported for each signal region. The published result unfortunately does not give upper limits on the number of new physics events, so we calculate these ourselves: we obtain upper limits on $N_{exp,obs}$ (see eq.~\ref{sigma_nom}) with a simple implementation of HistFitter that uses a frequentist calculator and a one-sided profile likelihood test statistic (the LHC default), giving the model-independent upper limits shown in table~\ref{tab:Nlim_monoZ}. Note that we use signal regions 1 and 2 only, as this simplified HistFitter approach was deemed inappropriate for the very low statistics of signal regions 3 and 4. These upper limits are also used for our validation procedure (see sec.~\ref{monoZ_validation}).

\begin{table}
\begin{center}
\begin{tabular}{ c  c  c }
\hline
\hline
& SR1 & SR2 \T \\
& ($E_{\mathrm{T}}^{\mathrm{miss}} > $ 150 GeV) & ($E_{\mathrm{T}}^{\mathrm{miss}} > $ 250 GeV) \B \\
\hline
$N_{\mathrm{sig}}^{\mathrm{exp}}$ & 34.7 & 6.8 \T \\
$N_{\mathrm{sig}}^{\mathrm{obs}}$ & 32.2 & 5.9 \B \\
\hline
\hline
\end{tabular}
\end{center}
\label{tab:Nlim_monoZ}
\caption{The expected and observed upper limits on the number of new physics events in the ATLAS \monoZ analysis, calculated with HistFitter using the results of \cite{Aad:2014monoZlep}.}
\end{table}

\subsection{Mono-$W/Z$ Constraints}
\label{monoWZ_constraints}

The analysis focusing on fully-hadronic decays of the bosons in the \monoWZ channel performed by ATLAS on the 8 TeV dataset was considered especially interesting, since constructive interference in the mono-$W(\rightarrow \ell \nu)$ channel was assumed to be possible for the vector EFT operator, leading to stronger limits than those from the \monojet searches in this case. More recently however, studies have revealed that such a scenario would violate unitarity and this interpretation is not emphasised in this work due to these concerns.

Nevertheless, this channel is an interesting addition, since it exploits the large branching fraction of hadronic boson decays. Also, the experimental techniques applied are significantly different; the selection is based on large-$R$ jets that are consistent with having come from an electroweak boson. In addition, large $\met$ is required, as for all \monoX searches.

The applied event selection is summarised in the following. Large-radius jets are reconstructed with the Cambridge-Aachen algorithm using radius parameter $R = 1.2$. A mass drop filter \ref{} is applied, where $\sqrt{y}>0.4$\footnote{$\sqrt{y} = \mathrm{min}(p_{\mathrm{T}_{j1}},p_{\mathrm{T}_{j2}})\Delta R / m_{jet}$, i.e. the momentum balance of the two leading subjets.} is required in order to suppress non-$W/Z$ processes.
Events with at least one large-radius jet with $p_{\mathrm{T}} >$ 250 GeV, $|\eta| <$ 1.2 and a mass within a window around the $W/Z$ mass, between 50 and 120 GeV, are selected.
To reduce the $t \bar{t}$ and QCD backgrounds, events containing small-$R$ (replace with `narrow' here?) jets (reconstructed with the anti-$k_T$ algorithm and radius parameter $R$ = 0.4) with $\Delta\phi(\mathrm{jet},\met)< 0.4$ or more than one jet ($p_{\mathrm{T}} >$ 40 GeV, $|\eta| <$ 4.5) with $\Delta R$(jet, large-$R$ jet)$>0.9$, are vetoed. Electrons, muons and photons are vetoed if their $p_{\mathrm{T}}$ is larger than 10 GeV and they are within $|\eta| <$ 2.47 (electrons), 2.5 (muons), 2.37 (photons). The original analysis considers two signal regions: $\met > 250$ GeV and $\met > 500$ GeV. We consider in the following only the signal region with $\met > 500$ GeV (\comm{should have a statement explaining this choice}).

The main background is due to $Z \rightarrow \bar{\nu} \nu$ events with additional jets from initial state radiation. Further important backgrounds from $W/Z$+jets events, in which the boson decays leptonically, enter the selection if the lepton(s) is missed due to being out of the acceptance range or failing ID requirements, or in the case that it is a hadronically-decaying $\tau$ lepton. All these backgrounds are estimated by the original analysis in dedicated control regions.

Whereas the ATLAS analysis uses a shape fit of the mass distribution of the large-radius jet, we simply regard the number of events in the signal region, since the data points of the $m_{\rm jet}$ distribution were not published. \comm{check!!!} From the published number of expected and observed events in the signal region and their uncertainties, we \comm{use HistFitter to} calculate the upper limit on the number of new physics events as described above for the \monoZ channel (see eq.~\ref{sigma_nom}). We obtain the following numbers:  $N_{exp} = 27.2$, $N_{obs} = 27.4$.
