\begin{flushleft}
The mono-X + $E_{T}^{\mathrm{miss}}$ signal is a popular signal in the search for new physics\fg{, particularly for dark matter, which is generally expected to be invisible to a detector, and so `appears' as missing transverse momentum when balanced against a visible object that is radiated from the initial or intermediate state}. \st{It is predicted by an assorted collection of new physics models, including unparticle physics models }\cite{CMS-PAS-EXO-12-048}\st{ and the Arkani-Hamed, Dimopoulos, and Dvali (ADD) model of large extra spatial dimensions }\cite{ATLAS-CONF-2012-147}. For the $s$-channel simplified models discussed in Section \ref{sec:sec2} the SM particle, X, originates from one of a pair of intial-state \st{quarks} \fg{partons} (shown in Figure \ref{Signal_phen_a} and \ref{Signal_phen_b}). The case where X is radiated from the mediator - a process known as virtual internal Bremsstrahlung - is only possible if the SM-dark matter interaction occurs via the $t$-channel (as shown in Figure \ref{Signal_phen_c}). \st{In both cases, the parton, W and Z bosons are the SM particles most likely to be emitted }\cite{Kumar:2013iva}\st{. Hence, this paper will focus on three detection channels: monojet, mono-Z and mono-W.} \fg{In all the $s$- and $t$-channel models we consider, radiation of a parton is the most likely scenario, due to the strength of the strong coupling, and so the mono-jet channel is expected to provide the strongest limits. However, emission of $Z$ and $W$ bosons \comm{(and photon?)} is also possible, often providing a cleaner signal, and so we also include results from the mono-$Z$ and mono-$W$ processes\footnote{\fg{It is worth noting here that the mono-$W$ channel has previously been chosen for its apparent ability to distinguish a case where $u$- and $d$-type quarks couple with opposite sign to the new physics sector \cite{}\comm{(Tait)}. However, during the preparation of this paper this has been shown to be an unphysical scenario \cite{}\comm{(Bell, Leane)}.}}, which may then enhance the limit in combination with the mono-jet.}

\begin{figure}[ht!]
\begin{subfigure}[b]{0.4\textwidth}
\centering
\begin{tikzpicture}
\draw[fermion] (-1.5,1.5)node[left]{$q$} --(-0.75,0.75);
\draw[dashed] (-0.75,0.75) -- (0,1.5)node[right]{X};
\draw[fermion] (-0.75,0.75) -- (0,0);
\draw[fermion] (-1.5,-1.5)node[left]{$\bar{q}'$} --(0,0);
\draw[fill] (0,0) circle [radius=0.05]node[left]{$g_{q'}\mbox{ }$};
\draw[photon] (0,0) --node[above]{$\xi$} (2,0);
\draw[fermion] (2,0) -- (3.5,1.5)node[right]{$\chi$};
\draw[fermion] (2,0) --(3.5,-1.5)node[right]{$\bar{\chi}$};
\draw[fill] (2,0) circle [radius=0.05]node[right]{$\mbox{ }g_{\chi}$};
\end{tikzpicture}
\caption{}
\label{Signal_phen_a}
\end{subfigure} \hspace{0.2cm}
\begin{subfigure}[b]{0.5\textwidth}
\centering
\begin{tikzpicture}
\draw[fermion] (-1.5,1.5)node[left]{$q$} --(-0.75,0.75);
\draw[dashed] (-0.75,0.75) -- (0,1.5)node[right]{X};
\draw[fermion] (-0.75,0.75) -- (0,0);
\draw[fermion] (-1.5,-1.5)node[left]{$\bar{q}'$} --(0,0);
\draw[fill] (0,0) circle [radius=0.05]node[left]{$g_{q'}\mbox{ }$};
\draw[dashed] (0,0) --node[above]{$\eta$} (2,0);
\draw[fermion] (2,0) -- (3.5,1.5)node[right]{$\chi$};
\draw[fermion] (2,0) --(3.5,-1.5)node[right]{$\bar{\chi}$};
\draw[fill] (2,0) circle [radius=0.05]node[right]{$\mbox{ }g_{\chi}$};
\end{tikzpicture}
\caption{}
\label{Signal_phen_b}
\end{subfigure}\vspace{0.2cm}
\centering
\begin{subfigure}[b]{0.5\textwidth}
\centering
\begin{tikzpicture}
\draw[fermion] (-2,1.5)node[left]{$q$} --(0,1.5);
\draw[fermion] (-2,-1)node[left]{$\bar{q}'$} --(0,-1);
\draw[fill] (0,1.5) circle [radius=0.05]node[above]{$g_{q}$};
\draw[photon] (0,1.5) -- node[left]{$\eta'_{q}$}(0,0.25);
\draw[dashed] (0,0.25) -- (2, 0.25)node[right]{X};
\draw[photon] (0,0.25) -- node[left]{$\eta'_{q'}$}(0,-1);
\draw[fermion] (0,1.5) -- (2,1.5)node[right]{$\chi$};
\draw[fermion] (0,-1) --(2,-1)node[right]{$\bar{\chi}$};
\draw[fill] (0,-1) circle [radius=0.05]node[below]{$g_{q'}$};
\end{tikzpicture}
\caption{}
\label{Signal_phen_c}
\end{subfigure}
\caption{Dark matter pair-production processes with a SM particle, X, in the final state for the $s$-channel (a) vector/axial-vector mediator and (b) scalar mediator simplified models and for (c) the $t$-channel scalar mediator model. \textcolor{red}{Do we need to include this?} \comm{Yep, I like the diagrams, however I think they need some changing - we don't have here the case of gluons in the initial state, and I think this is enough to warrant splitting into two figures. Also, can we change the arrows on the anti-particles? - Amelia}}
\end{figure}

\hspace{1cm}\draft{The monojet contraints are derived from a genertic search for new phenomena conducted by the ATLAS Collaboration using $pp$ collisions at $\sqrt{s}=$ 8 TeV (see Section \ref{monojet_constraints})}. Similarly, the mono-Z constraints are derived from a dark matter search originally optimised for the D1, D5 and D9 effective operators (see Section \ref{monoZ_constraints}). \textcolor{magenta}{Lastly, the mono-W constraints are derived from...} \comm{We should put references to all the papers here! And also include something like 'These analyses will be described in further detail in Section X.' - Amelia}
\bigskip

\textcolor{magenta}{Do we need something here like "The model-independent results for each of the above analyses are reintepreted as simplified model constraints following a simple methodology. Firstly..."? Or is it ok to assume that the reader understands the process employed in converting model-independent limits into limits on a specific model? Do we need to introduce them to $\sigma = N/\mathcal{L}\times\epsilon\times\mathcal{A}$?}

\comm{Hmm. I think that since the results we're using are all given in different ways (ie a limit on $\sigma^{\mathrm{vis}}$ for some, or a limit on $N_{NP}$, or even just the number of observed and predicted events (damn monoZ)), this would be better included in each channel-specific section instead. It may not be necessary to even include it though, I think it's reasonable to say 'we converted the limits to our own model' or whatever. - Amelia}

\end{flushleft}

\subsection{Signal Generation}
\label{signal_generation}
\begin{flushleft}
\comm{Here, I'd be ignoring the stuff specific to mono-jet (0,1,2 jets, and matching scheme), and include that in the mono-jet section. - Amelia}

For each channel, signal production was modelled with \MG5$\_$aMC 2.2.2 \comm{(references needed throughout this section!)}. Showering and hadronisation was then performed by \PYTHIA  8.201 using the \textcolor{magenta}{ATLAS UE Tune AU2-MSTW2008LO}. Using the MSTW2008lo68 PDF set (currently the benchmark for ATLAS analyses \cite{}. \question{Is this still true?}) and the default \MG factorization and renormalization scales\footnote{In \FNMG5 2.2.2 the default scale, $\mu$ is defined as the sum of $\sqrt{m_{\chi}^{2} + p_{T}^{2}}$ for all dark matter particles in the final state.}, samples were produced for a representative set of of dark matter and mediator masses, shown in Table \ref{Mass_coup_points}. \textcolor{magenta}{Something about the coupling strength, $f$, being set to a value of 1? Maybe leave out - we'll vary the value.} Note that rows one to three of Table \ref{Mass_coup_points} cover the case where the dark matter mass is small and the mediator has either a mass that is degenerate, low or sufficiently large so as to be in the EFT regime of validity\footnote{A recent study by Alves et al. found that EFT results do not apply to mediators with a mass less than 2.5 TeV at the LHC during Run I \cite{Alves:2011wf}.}. Rows four and five cover the case where the dark matter particle has a medium mass and the mediator is either near-degenerate or larger. Rows six and seven (eight and nine) cover the case where the dark matter particle is a few GeV (TeV) and the mediator is either near-degenerate or again sufficiently large so as to be in the EFT regime of validity. \comm{Comment on the masspoints being dictated by the narrow width approximation used by MG?}

\begin{table}[!htbp]
\centering
\begin{tabular}{l|l|l}
\hline
\hline
 \rule{0pt}{2.2ex}Column & $m_{\chi}$ [GeV] & $M$ [GeV]\\
\hline
 \rule{0pt}{2.2ex}1 & 10 & 10\\
 2 & 10 & 200\\
 3 & 10 & 25000\\
 4 & 100 & 200\\
 5 & 100 & 2000\\
 6 & 500 & 600\\
 7 & 500 & 25000\\
 8 & 1000 & 2000\\
 9 & 1000 & 25000\\
 \hline
 \hline
\end{tabular}
\caption{Mass points chosen for the analysis of simplified dark matter models. The masspoints are primarily representative of three regimes: (near-)degenerate ($M\approx m_{\chi}$), kinematically allowed ($M \geq 2m_{\chi}$), and EFT-like ($\sqrt{\hat{s}} << M$).}
\label{Mass_coup_points}
\end{table}
\end{flushleft}

\fg{We now move to a discussion of each of the mono-X channels separately.}

\subsection{Monojet Constraints}
\label{monojet_constraints}
\begin{flushleft}
For the monojet channel, signal selection is carried out as in Ref. \cite{Aad:2015zva}: initially, events are required to have $E_{T}^{miss}>$ 150 GeV and at least one jet with $p_{T} >$ 30 GeV and $|\eta| <$ 4.5. The background contribution from QCD jets is managed with the requirement that all jets passing this initial selection criteria also satsify $|\Delta\phi(jet,{E}_{T}^{miss})|>1.0$. In order to isolate monojet-like topologies, surviving events are then required to have a leading jet, $j_{1}$, with $p_{T} >$ 120 GeV and $|\eta| <$ 2.0 satisfying $p_{T}^{j_{1}}/E_{T}^{miss} >$ 0.5. Lastly, nine separate signal regions are defined with increasing lower thresholds on $E_{T}^{miss}$ from 150 GeV to 700 GeV as shown in Table \ref{monojet_SRs}.

\begin{table}[!htbp]
\centering
\begin{tabular}{c|c|c|c}
 \hline
 \hline
 Signal Region & $E_{T}^{miss}$ threshold [GeV] & $N_{obs}$ & $N_{exp}$ \\%& $N_{obs}/\mathcal{L}$ & $N_{exp}/\mathcal{L}$ \\
 \hline
 SR1 & 150 & 14737.8 & 18980.5 \\%726 & 935 \\
 SR2 & 200 & 3938.2 & 5501.3 \\%194 & 271 \\
 SR3 & 250 & 1827 & 2151.8 \\%90 & 106 \\
 SR4 & 300 & 913.5 & 1035.3 \\%45 & 51 \\
 SR5 & 350 & 426.3 & 588.7 \\%21 & 29 \\
 SR6 & 400 & 243.6 & 345.1 \\%12 & 17 \\
 SR7 & 500 & 146.16 & 146.16 \\%7.2 & 7.2 \\
 SR8 & 600 & 77.14 & 73.08 \\%3.8 & 3.6 \\
 SR9 & 700 & 69.02 & 36.54 \\%3.4 & 1.8 \\
 \hline
 \hline
\end{tabular}
\caption{The ATLAS model-independent upper limits on the number of observed (expected) signal events, $N_{obs}$ ($N_{exp}$), at 95\% confidence level for the monojet channel signal regions. Adapted from Ref. \cite{Aad:2015zva}.}
\label{monojet_SRs}
\end{table}

The signal selection and Monte Carlo (MC) generation procedures for the monojet channel were validated via reproduction of the ATLAS limits on the suppression scale, M$_{*}$, for the s-channel vector simplified model (refered to as the Z' model in the ATLAS monojet analysis). The results of this process are contained in appendix \ref{monojet_validation}.
\bigskip

In keeping with the ATLAS analysis, monojet signal samples were generated using the MLM scheme for matching between \MG and \PYTHIA. However, while the ATLAS analysis favours a single matching scale (QCUT) of 80 GeV, we use a sliding value of $m_{\chi}/4$. \comm{This ensures that all jet-related kinematic distributions are smoothly connected across all dark matter masspoints.} \draft{It is also important to note that while the signal samples used for validation correspond to the processes $pp \rightarrow j\chi\bar{\chi}$ and $pp \rightarrow jj\chi\bar{\chi}$ where $j$ is a final state jet, constraints on the simplified models studied in this paper are determined using signal samples with zero, one and two jets in the final state.} \comm{This deviation from the ATLAS analysis is motivated by the observation that the limits on $M_{*}$ are improved with the addition of the process $pp \rightarrow \chi\bar{\chi}$.}

%SUSY Selection
%For the monojet channel, signal selection is carried out as in Ref. \cite{SUSY_official_paper} wherein events are required to have a maximum of three good jets with $p_{T}>30\,\mbox{GeV}$, $|\eta|<2.8$ and $\Delta\phi(jet\,\slashed{E}_{T}^{miss})>0.4$. Three separate signal regions are then defined with increasing lower thresholds on the momentum of the leading jet, $p_{T}^{j1}$, and $\slashed{E}_{T}^{miss}$ as shown in Table \ref{monojet_SRs}.
%
%\begin{table}[!htbp]
%\centering
%\begin{tabular}{l|c|c|c}
% \hline
% \hline
% Signal Region: & M1 & M2 & M3\\
% \hline
% Minimum leading jet $p_{T}$ [GeV] & 280 & 340 & 450\\
% Minimum $E_{T}^{miss}$ [GeV] & 220 & 340 & 450\\
% \hline
% \hline
%\end{tabular}
%\caption{Signal region definitions and associated for the analysis of simplified models in the monojet channel. Adapted from Ref. \cite{SUSY_official_paper}.}
%\label{monojet_SRs}
%\end{table}

%The monojet channel is a popular channel in the search for new physics. It is predicted by an assorted collection of new physics models, including unparticle physics models \cite{CMS:raw} and the Arkani-Hamed, Dimopoulos, and Dvali (ADD) model of large extra spatial dimensions \cite{ATLAS:2012zim}. In addition, a monojet-like signal is predicted by R-parity conserving SUSY models for the process $pp\rightarrow j + \tilde{t}(\rightarrow c+\tilde{\chi}^{0})\bar{\tilde{t}}(\rightarrow \bar{c}+\tilde{\chi}^{0})$ where $\tilde{t}$ is the top squark (stop), $\tilde{\chi}^{0}$ is the lightest neutralino (assumed to be the lightest sparticle) and $j$ is a final-state jet. In this paper, we place limits on the simplified models discussed in Section \ref{models} using the results of a recent search for this process conducted by the ATLAS Supersymmetry group \cite{SUSY_official_paper}.
%\bigskip
%
%\textcolor{magenta}{Do I need to talk about why the SUSY analysis results are compatible with a simplified model search?} Eg. "The SUSY analysis assumes that the stop and LSP masses are nearly degenerate. Jets from the charm-quark fragmentation are then too soft to be detected or identified. In this case, the stops decay essentially invisibly and the final state is trivially comparable to that of the dark matter pair production process $pp\rightarrow j + \chi\bar{\chi}$. The two processes are of course topologically dissimilar but since the Supersymmetry analysis simply looks for an initial-state jet recoiling against missing energy it seemed possible that the results of the Supersymmetry analysis optimised for a $\tilde{t}$ search might be directly transferable to $\chi$. This was confirmed via a two-fold check of the distributions of key kinematic variables - namely $E_{T}^{miss}$, the jet multiplicity and the $p_{T}$ and $\eta$ of the leading jet - associated with both the dark matter and Supersymmetry signal processes..."

THIS IS WHERE I'M UP TO!

\end{flushleft}
%\subsection*{MC for monojet}
%\begin{flushleft}
%For the SUSY monojet-like analysis events are selected with the goal to gain sensitivity in the region of parameter space where $\tilde{t}$ and $\tilde{\chi}^{0}$ are nearly degenerate in mass. With this in mind, event selection is divided into two categories: preselection and signal selection. For the first catgeory, events are required to have a reconstructed primary vertex (interaction point) with at least five tracks attached to it. This vertex must align with the nominal (designed) interaction point and should there be more than one primary vertex, the vertex with the highest summed $p_{T}^{2}$ of the attached tracks is selected. This is the normal procedure for selecting hard-scatter interactions.
%\end{flushleft}
%\begin{flushleft}
%Next events are required to have $E_{T}^{miss}>150\,\mbox{GeV}$ in accordance with the performance of the $E_{T}^{miss}$ trigger, which is close to 100\% efficient for $E_{T}^{miss}>150\,\mbox{GeV}$ \cite{Aad:1363019}. In addition to this, events must have a minimum of one jet with $p_{T}>150\,\mbox{GeV}$ and $|\eta|<2.8$. This ensures that candidate jets originate from a hard-scattering process which has the correct jet + $E_{T}^{miss}$ topology. These requirements also omit much of the QCD multijet background.
%\end{flushleft}
%\begin{flushleft}
%In order to distinguish hard-scatter jets from fake jets and non-collision jets, 'jet quality' requirements are imposed. First an event is vetoed if it contains any jet with $p_{T}>20\,\mbox{GeV}$ and $|\eta|<4.5$ that presents an anomalous charged fraction defined by:
%\begin{equation}
%f_{ch} = \frac{\Sigma p_{T}^{track, jet}}{p_{T}^{jet}}
%\end{equation}
%where $\Sigma p_{T}^{track, jet}$ is the scalar sum of the transverse momenta
%of tracks associated with the primary vertex within a cone of radius $\Delta R = 0.4$ around the jet axis. Additionally, if any jet has an electromagnetic fraction or timing inconsistent with a proton-proton origin then the event is vetoed. These latter requirements ensure that noise produced by the calorimeters, which might potentially fake a jet, is omitted.
%\end{flushleft}
%\begin{flushleft}
%Finally, events are vetoed if they contain at least one identified electron with $p_{T}>20\,\mbox{GeV}$ or one muon with $p_{T}>10\,\mbox{GeV}$. These cuts remove non-signal $E_{T}^{miss}$ events with leptons in the signal region, for example W ($\rightarrow\Pe/\Pmu\Pneutrino$) + jets events and semileptonic top decays. 
%\end{flushleft}
%\begin{flushleft}
%For the signal selection, a maximum of three \textit{good} jets with $p_{T}>30\,\mbox{GeV}$ and $|\eta|<2.8$ are allowed\footnote{It is this particular cut that permits us to ignore the differences between the $\eta$ distributions of the SUSY and simplified models. Requiring a jet with $p_{T}>30\,\mbox{GeV}$ to fall within the range -2.8 $<$ $\eta$ $<$ 2.8 is designed to omit background jets whilst preserving a maximal number of signal jets. Looking at the top left plot in Figure \ref{SUSY_DM_kinematics} we see that, regardless of the difference in shape, a maximal number of signal jets sit within $|\eta|<2.8$ for both classes of models. Specifically, 93.73\% of signal jets are within $|\eta|<2.8$ for the SUSY model compared with $\sim$98\% for the simplified models.}. This cut comes from examination of the jet mulitplicity associated with the radiated parton in the stop pair-production process (see Figure \ref{SUSY_kinematics}), which peaks at about 1. This cut is also designed to preserve any potentially detectable jets resulting from the decay of the charm quark. To reduce the multijet background when $E_{T}^{miss}$ is constructed from jet(s) with mis-measured $p_{T}$, each good jet must be azimuthally separated from $E_{T}^{miss}$ by no less than 0.4. Note that for the stop pair-production process, the parton and the two stop-quarks are emitted back-to-back. When the parton fragments into two or more jets however, the jet with the higher $p_{T}$ is not expected to satisfy $\Delta\phi(jet,\,E_{T}^{miss})=\pi$. Since the parton is likely to be boosted, requiring the lead jet to satisfy $\Delta\phi(jet,\,E_{T}^{miss})>0.4$ is reasonable. Lastly, three separate signal regions are defined with increasing lower thresholds on the leading jet $p_{T}$ and $E_{T}^{miss}$. These regions - denoted M1, M2 and M3 - result from an optimization performed across the stop-neutralino mass plane with increasing $m_{\tilde{t}}$ and $m_{\tilde{\chi}^{0}}$ as discussed in Ref. \cite{Abdallah:1636856}. For convenience, the event selection criteria are summarised in Table \ref{Cutflow}.
%
%\begin{table}[!htbp]
%\centering
%\begin{tabular}{l|c|c|c}
% \hline
% \hline
% \multicolumn{4}{c}{Preselection}\\
% \hline
% \multicolumn{4}{l}{Primary vertex}\\
% \multicolumn{1}{l}{$E_{T}^{miss} > 150\,\mbox{GeV}$}\\
% \multicolumn{1}{l}{At least one jet with $p_{T}>150\,\mbox{GeV}$ and $|\eta|<2.8$}\\
% \multicolumn{1}{l}{Jet quality requirements}\\
% \multicolumn{1}{l}{Lepton veto}\\
% \hline
% \multicolumn{4}{c}{Monojet-like selection}\\
% \hline
% \multicolumn{1}{l}{At most three jets wth $p_{T}>30\,\mbox{GeV}$ and $|\eta|<2.8$}\\
% \multicolumn{1}{l}{$\Delta\phi(jet\,E_{T}^{miss})>0.4$}\\
% \hline
% Signal Region: & M1 & M2 & M3\\
% \hline
% Minimum leading jet $p_{T}$ [GeV] & 280 & 340 & 450\\
% Minimum $E_{T}^{miss}$ [GeV] & 220 & 340 & 450\\
% \hline
% \hline
%\end{tabular}
%\caption{Event selection criteria for the monojet-like analysis. Adapted from Ref. \cite{SUSY_official_paper}.}
%\label{Cutflow}
%\end{table}
%
%It is important to note that the control regions used in the SUSY monojet-like analysis also abide the preselection criteria discussed above but sans the lepton veto. The idea is that leptons are now used to identify and select background events in order to assess background contributions in the signal regions. Specifically, muon selection is used to constrain events from background processes involving a muon (W$\rightarrow \Pmu\Pneutrino$, Z$\rightarrow \Pmuon\APmuon$) or neutrino (Z$\rightarrow \Pneutrino\APneutrino$). In the same way, electrons are used to constrain background contributions from processes involving electrons or hadronic taus.
%\end{flushleft}

%Note that we are interested in the behaviour of only high pT jets so we
%remove jets with pT < 30 GeV. 

\subsection{Mono-Z Constraints}
\label{monoZ_constraints}
\begin{flushleft}
\comm{Note to Amelia: Here you should discuss the original intention of the mono-Z analysis. Also comment on any validation you did in order to confirm that you could use the results of the analysis. Lastly, discuss or list the cuts used in the analysis and the uncertainties associated with the results. This will inevitably include details of the background estimation and the detector performance.}
\end{flushleft}

\subsection{Mono-W Constraints}
\begin{flushleft}
\comm{Note to Johanna: Here you should discuss the original intention of the mono-W analysis. Also comment on any validation you did in order to confirm that you could use the results of the analysis. Lastly, discuss or list the cuts used in the analysis and the uncertainties associated with the results. This will inevitably include details of the background estimation and the detector performance.}
\end{flushleft}

\begin{flushleft}
\textcolor{magenta}{This section should include:}
\begin{enumerate}
\item \textcolor{magenta}{A description of the general process employed to reinterpret monojet, mono-W and mono-Z limits. This should cover MC signal generation, the cuts used in each mono-X channel and validation of the procedure(s) used to generate the results.}
\item \textcolor{magenta}{Monojet specifics: motivation for why the SUSY results were used instead of the monojet+MET results.}
\item \textcolor{magenta}{A description of the assessment of the systematics.}
\item \textcolor{magenta}{The limit setting strategy (?).}
\end{enumerate}
\end{flushleft}
