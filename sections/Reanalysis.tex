\begin{flushleft}
The mono-X + $E_{T}^{\mathrm{miss}}$ signal is a popular signal in the search for new physics, particularly in the search for dark matter. Since WIMPs are not expected to interact with detector material, they appear as missing transverse momentum, $\mathbf{p}_{T}^{miss}$, when balanced against a visible object that is radiated from the initial or intermediate state. 
%\st{It is predicted by an assorted collection of new physics models, including unparticle physics models }\cite{CMS-PAS-EXO-12-048}\st{ and the Arkani-Hamed, Dimopoulos, and Dvali (ADD) model of large extra spatial dimensions }\cite{ATLAS-CONF-2012-147}. 
For the $s$-channel simplified models discussed in Section \ref{sec:sec2}, a SM particle, X, is emitted from one of a pair of intial-state partons (shown in Figure \ref{schannel_sig_phen}). The case where X is radiated from the mediator - a process known as virtual internal Bremsstrahlung - is only possible if the SM-dark matter interaction occurs via the $t$-channel (as shown in Figure \ref{tchannel_sig_phen}). 
%\st{In both cases, the parton and Z boson are the SM particles most likely to be emitted \question{is Pr(W)$>$Pr(Z)?}}\cite{Kumar:2013iva}. \st{Hence, this paper will focus on thre detection channels: monojet, mono-Z and mono-W.}
\fg{For all models, emission of a parton is the most likely scenario owing to the strength of the strong coupling and so the mono-jet channel is expected to provide the strongest limits. However, emission of $Z$ and $W$ bosons is also possible, often providing a cleaner signal, and so we also include results from the mono-$Z$ \comm{and mono-$W$} processes\footnote{\fg{It is worth noting here that the mono-$W$ channel has previously been chosen for its apparent ability to distinguish a case where $u$- and $d$-type quarks couple with opposite sign to the new physics sector \cite{}\comm{(Tait)}. However, during the preparation of this paper this has been shown to be an unphysical scenario \cite{}\comm{(Bell, Leane)}.}}, which may then enhance the limit in combination with the mono-jet.}

\begin{figure}[ht!]
\begin{center}
\begin{subfigure}[b]{0.45\textwidth}
\centering
\begin{tikzpicture}
\draw[fermion] (-1.5,1.5)node[left]{$q$} --(-0.75,0.75);
\draw[dashed] (-0.75,0.75) -- (0,1.5)node[right]{$j$}; %\draw[dashed] (-0.75,0.75) -- (0,1.5)node[right]{X};
\draw[fermion] (-0.75,0.75) -- (0,0);
\draw[antifermion] (-1.5,-1.5)node[left]{$\bar{q}'$} --(0,0);
\draw[fill] (0,0) circle [radius=0.05]node[left]{$g_{q'}\mbox{ }$};
\draw[photon] (0,0) --node[above]{$\xi$} (2,0);
\draw[fermion] (2,0) -- (3.5,1.5)node[right]{$\chi$};
\draw[antifermion] (2,0) --(3.5,-1.5)node[right]{$\bar{\chi}$};
\draw[fill] (2,0) circle [radius=0.05]node[right]{$\mbox{ }g_{\chi}$};
\end{tikzpicture}
\caption{}
\label{Signal_phen_sa}
\end{subfigure} \hspace{0.1cm}
\begin{subfigure}[b]{0.45\textwidth}
\centering
\begin{tikzpicture}
\draw[fermion] (-1.5,1.5)node[left]{$q$} --(0,0);
\draw[antifermion] (-1.5,-1.5)node[left]{$\bar{q}$} --(0,0);
\draw[fill] (0,0) circle [radius=0.05]node[left]{$g_{q'}\mbox{ }$};
\draw[fermion] (0,0) --node[above]{$q$} (2,0);
\draw[dashed] (2,0) -- node[above]{$\xi\mbox{\textcolor{white}{m}}$}(2.75,0.75);
\draw[fermion] (2.75,0.75) -- (3.5,1.5)node[right]{$\chi$};
\draw[antifermion] (2.75,0.75) --(3.5,0)node[right]{$\bar{\chi}$};
\draw[fill] (2,0) circle [radius=0.05]node[right]{$\mbox{ }g_{q}$};
\draw[antifermion] (2,0) --(3.5,-1.5)node[right]{$q$};
\draw[fill] (2.75,0.75) circle [radius=0.05]node[right]{$\mbox{\textcolor{white}{\tiny m}}g_{\chi}$};
\end{tikzpicture}
\caption{}
\label{Signal_phen_sb}
\end{subfigure}
%\begin{subfigure}[b]{0.5\textwidth}
%\centering
%\begin{tikzpicture}
%\draw[fermion] (-1.5,1.5)node[left]{$q$} --(-0.75,0.75);
%\draw[dashed] (-0.75,0.75) -- (0,1.5)node[right]{X};
%\draw[fermion] (-0.75,0.75) -- (0,0);
%\draw[fermion] (-1.5,-1.5)node[left]{$\bar{q}'$} --(0,0);
%\draw[fill] (0,0) circle [radius=0.05]node[left]{$g_{q'}\mbox{ }$};
%\draw[dashed] (0,0) --node[above]{$\eta$} (2,0);
%\draw[fermion] (2,0) -- (3.5,1.5)node[right]{$\chi$};
%\draw[fermion] (2,0) --(3.5,-1.5)node[right]{$\bar{\chi}$};
%\draw[fill] (2,0) circle [radius=0.05]node[right]{$\mbox{ }g_{\chi}$};
%\end{tikzpicture}
%\caption{}
%\label{Signal_phen_b}
%\end{subfigure}\vspace{0.2cm}
%\centering
\begin{subfigure}[b]{0.45\textwidth}
\centering
\begin{tikzpicture}
\draw[gluon] (-1.5,1.5)node[left]{$g$} --(0,0); %-0.75,0.75);
\draw[dashed] (-0.75,0.75) -- (0,1.5)node[right]{$j$};
%\draw[gluon] (-0.75,0.75) -- (0,0);
\draw[gluon] (-1.5,-1.5)node[left]{$g$} --(0,0);
\draw[fill] (0,0) circle [radius=0.05]node[left]{$g_{q'}\mbox{ }$};
\draw[photon] (0,0) --node[above]{$\xi$} (2,0);
\draw[fermion] (2,0) -- (3.5,1.5)node[right]{$\chi$};
\draw[antifermion] (2,0) --(3.5,-1.5)node[right]{$\bar{\chi}$};
\draw[fill] (2,0) circle [radius=0.05]node[right]{$\mbox{ }g_{\chi}$};
\end{tikzpicture}
\caption{}
\label{Signal_phen_sc}
\end{subfigure} \hspace{0.1cm}
\begin{subfigure}[b]{0.45\textwidth}
\centering
\begin{tikzpicture}
\draw[fermion] (-1.5,1.5)node[left]{$q$} --(0,0);
\draw[gluon] (-1.5,-1.5)node[left]{$g$} --(0,0);
\draw[fill] (0,0) circle [radius=0.05]node[left]{$g_{q'}\mbox{ }$};
\draw[fermion] (0,0) --node[above]{$q$} (2,0);
\draw[dashed] (2,0) -- node[above]{$\xi\mbox{\textcolor{white}{m}}$}(2.75,0.75);
\draw[fermion] (2.75,0.75) -- (3.5,1.5)node[right]{$\chi$};
\draw[antifermion] (2.75,0.75) --(3.5,0)node[right]{$\bar{\chi}$};
\draw[fill] (2,0) circle [radius=0.05]node[right]{$\mbox{ }g_{q}$};
\draw[antifermion] (2,0) --(3.5,-1.5)node[right]{$q$};
\draw[fill] (2.75,0.75) circle [radius=0.05]node[right]{$\mbox{\textcolor{white}{\tiny m}}g_{\chi}$};
\end{tikzpicture}
\caption{}
\label{Signal_phen_sd}
\end{subfigure}
\begin{subfigure}[b]{0.55\textwidth}
\centering
\begin{tikzpicture}
\draw[fermion] (-1.5,1.5)node[left]{$q$} --(-0.75,0.75);
\draw[photon] (-0.75,0.75) -- (0,1.5)node[right]{$Z$}; %\draw[dashed] (-0.75,0.75) -- (0,1.5)node[right]{X};
\draw[fermion] (-0.75,0.75) -- (0,0);
\draw[antifermion] (-1.5,-1.5)node[left]{$\bar{q}'$} --(0,0);
\draw[fill] (0,0) circle [radius=0.05]node[left]{$g_{q'}\mbox{ }$};
\draw[photon] (0,0) --node[above]{$\xi$} (2,0);
\draw[fermion] (2,0) -- (3.5,1.5)node[right]{$\chi$};
\draw[antifermion] (2,0) --(3.5,-1.5)node[right]{$\bar{\chi}$};
\draw[fill] (2,0) circle [radius=0.05]node[right]{$\mbox{ }g_{\chi}$};
\end{tikzpicture}
\caption{}
\label{Signal_phen_se}
\end{subfigure}
\caption{Dark matter pair-production processes with a parton, $j$, or $Z$ boson in the final state for the $s$-channel vector/axial-vector mediator model. \textcolor{red}{Do we need to include this?} \comm{Yep, I like the diagrams, however I think they need some changing - we don't have here the case of gluons in the initial state, and I think this is enough to warrant splitting into two figures. Also, can we change the arrows on the anti-particles? - Amelia} \question{Do we want to instead have a page of Feynman diagrams where X is either a quark, gluon or Z so we can include loop diagrams, etc? - Millie}}
\label{schannel_sig_phen}
\end{center}
\end{figure}

\begin{figure}[ht!]
\begin{center}
\begin{subfigure}[b]{0.45\textwidth}
\centering
\begin{tikzpicture}
\draw[fermion] (-2,2.25)node[left]{$q$} --(-1,2.25);
\draw[dashed] (-1,2.25) -- (0,3)node[right]{$j$};
\draw[fermion] (-1,2.25) --(0,2.25);
\draw[antifermion] (-2,0.25)node[left]{$\bar{q}'$} --(0,0.25);
\draw[fill] (0,2.25) circle [radius=0.05]node[above]{$g_{q'}$};
\draw[photon] (0,2.25) -- node[left]{$\eta'_{q}$}(0,0.25);
\draw[antifermion] (0,0.25) -- (2, 0.25)node[right]{$\bar{\chi}$};
\draw[fermion] (0,2.25) -- (2,2.25)node[right]{$\chi$};
\draw[fill] (0,0.25) circle [radius=0.05]node[below]{$g_{q'}$};
\end{tikzpicture}
\caption{}
\label{Signal_phen_ta}
\end{subfigure} \hspace{0.1cm}
\begin{subfigure}[b]{0.45\textwidth}
\centering
\begin{tikzpicture}
\draw[gluon] (-2,2.25)node[left]{$g$} --(0,2.25);
%\draw[dashed] (-1,2.25) -- (0,3)node[right]{$j$};
\draw[fermion] (-2,0.25)node[left]{$q'$} --(0,0.25);
\draw[fill] (0,2.25) circle [radius=0.05]node[above]{$g_{q'}$};
\draw[photon] (0,2.25) -- node[left]{$\eta'_{q}$}(0,0.25);
\draw[fermion] (0,0.25) -- (2, 0.25)node[right]{$\chi$};
\draw[dashed] (0,2.25) -- node[above]{$\eta'_{q}$}(1,2.25);
\draw[fermion] (1,2.25) -- (2,3)node[right]{$q$};
\draw[fermion] (1,2.25) -- (2,1.5)node[right]{$\bar{\chi}$};
\draw[fill] (1,2.25) circle [radius=0.05]node[right]{$\textcolor{white}{\tiny m}g_{q'}$};
\draw[fill] (0,0.25) circle [radius=0.05]node[below]{$g_{q'}$};
\end{tikzpicture}
\caption{}
\label{Signal_phen_tb}
\end{subfigure}


\begin{subfigure}[b]{0.45\textwidth}
\centering
\begin{tikzpicture}
\draw[fermion] (-1.5,1.5)node[left]{$q$} --(0,0);
\draw[gluon] (-1.5,-1.5)node[left]{$g$} --(0,0);
\draw[fill] (0,0) circle [radius=0.05]node[left]{$g_{q}\mbox{\textcolor{white}{\tiny m}}$};
\draw[fermion] (0,0) --node[above]{$q$} (2,0);
\draw[dashed] (2,0) -- node[above]{$\xi\mbox{\textcolor{white}{m}}$}(2.75,0.75);
\draw[fermion] (2.75,0.75) -- (3.5,1.5)node[right]{$q$};
\draw[fermion] (2.75,0.75) --(3.5,0)node[right]{$\bar{\chi}$};
\draw[fill] (2,0) circle [radius=0.05]node[right]{$\mbox{ }g_{q'}$};
\draw[fermion] (2,0) --(3.5,-1.5)node[right]{$\chi$};
\draw[fill] (2.75,0.75) circle [radius=0.05]node[right]{$\mbox{\textcolor{white}{\tiny m}}g_{q'}$};
\end{tikzpicture}
\caption{}
\label{Signal_phen_tc}
\end{subfigure} \hspace{0.1cm}
\begin{subfigure}[b]{0.45\textwidth}
\centering
\begin{tikzpicture}
\draw[fermion] (-2,1.5)node[left]{$q$} --(0,1.5);
\draw[gluon] (-2,-1)node[left]{$g$} --(0,-1);
\draw[fill] (0,1.5) circle [radius=0.05]node[above]{$g_{q'}$};
\draw[photon] (0,1.5) -- node[left]{$\eta'_{q}$}(0,0.25);
\draw[antifermion] (0,0.25) -- (2, 0.25)node[right]{$\bar{\chi}$};
\draw[fill] (0,0.25) circle [radius=0.05]node[left]{$g_{q'}$};
\draw[fermion] (0,0.25) -- node[left]{$q$}(0,-1);
\draw[fermion] (0,1.5) -- (2,1.5)node[right]{$\chi$};
\draw[fermion] (0,-1) --(2,-1)node[right]{$q$};
\draw[fill] (0,-1) circle [radius=0.05]node[below]{$g_{q}$};
\end{tikzpicture}
\caption{}
\label{Signal_phen_td}
\end{subfigure}
\begin{subfigure}[b]{0.45\textwidth}
\centering
\begin{tikzpicture}
\draw[fermion] (-2,1.5)node[left]{$q$} --(0,1.5);
\draw[antifermion] (-2,-1)node[left]{$\bar{q}'$} --(0,-1);
\draw[fill] (0,1.5) circle [radius=0.05]node[above]{$g_{q'}$};
\draw[photon] (0,1.5) -- node[left]{$\eta'_{q}$}(0,0.25);
\draw[gluon] (0,0.25) -- (2, 0.25)node[right]{g};
\draw[photon] (0,0.25) -- node[left]{$\eta'_{q'}$}(0,-1);
\draw[fermion] (0,1.5) -- (2,1.5)node[right]{$\chi$};
\draw[antifermion] (0,-1) --(2,-1)node[right]{$\bar{\chi}$};
\draw[fill] (0,-1) circle [radius=0.05]node[below]{$g_{q'}$};
\end{tikzpicture}
\caption{}
\label{Signal_phen_te}
\end{subfigure}

\begin{subfigure}[b]{0.45\textwidth}
\centering
\begin{tikzpicture}
\draw[fermion] (-2,2.25)node[left]{$q$} --(-1,2.25);
\draw[photon] (-1,2.25) -- (0,3)node[right]{$Z$};
\draw[fermion] (-1,2.25) --(0,2.25);
\draw[antifermion] (-2,0.25)node[left]{$\bar{q}'$} --(0,0.25);
\draw[fill] (0,2.25) circle [radius=0.05]node[above]{$g_{q'}$};
\draw[photon] (0,2.25) -- node[left]{$\eta'_{q}$}(0,0.25);
\draw[antifermion] (0,0.25) -- (2, 0.25)node[right]{$\bar{\chi}$};
\draw[fermion] (0,2.25) -- (2,2.25)node[right]{$\chi$};
\draw[fill] (0,0.25) circle [radius=0.05]node[below]{$g_{q'}$};
\end{tikzpicture}
\caption{}
\label{Signal_phen_tf}
\end{subfigure}
\begin{subfigure}[b]{0.45\textwidth}
\centering
\begin{tikzpicture}
\draw[fermion] (-2,1.5)node[left]{$q$} --(0,1.5);
\draw[antifermion] (-2,-1)node[left]{$\bar{q}'$} --(0,-1);
\draw[fill] (0,1.5) circle [radius=0.05]node[above]{$g_{q'}$};
\draw[photon] (0,1.5) -- node[left]{$\eta'_{q}$}(0,0.25);
\draw[photon] (0,0.25) -- (2, 0.25)node[right]{Z};
\draw[photon] (0,0.25) -- node[left]{$\eta'_{q'}$}(0,-1);
\draw[fermion] (0,1.5) -- (2,1.5)node[right]{$\chi$};
\draw[antifermion] (0,-1) --(2,-1)node[right]{$\bar{\chi}$};
\draw[fill] (0,-1) circle [radius=0.05]node[below]{$g_{q'}$};
\end{tikzpicture}
\caption{}
\label{Signal_phen_tg}
\end{subfigure}
\caption{Dark matter pair-production processes with a quark, $q$, gluon, $g$, (collectively, a parton, $j$) or Z boson in the final state for the $t$-channel scalar mediator model.}
\label{tchannel_sig_phen}
\end{center}
\end{figure}

The monojet contraints are derived from a search for new phenomena conducted by the ATLAS Collaboration using $pp$ collisions at $\sqrt{s}=$ 8 TeV \cite{Aad:2015zva} (see Section \ref{monojet_constraints}). Similarly, the mono-Z constraints are derived from a dark matter search originally optimised for the D1, D5 and D9 effective operators (see Section \ref{monoZ_constraints}). \textcolor{magenta}{Lastly, the mono-W constraints are derived from...} \comm{We should put references to all the papers here! And also include something like 'These analyses will be described in further detail in Section X.' - Amelia}
\bigskip

\textcolor{magenta}{Do we need something here like "The model-independent results for each of the above analyses are reintepreted as simplified model constraints following a simple methodology. Firstly..."? Or is it ok to assume that the reader understands the process employed in converting model-independent limits into limits on a specific model? Do we need to introduce them to $\sigma = N/\mathcal{L}\times\epsilon\times\mathcal{A}$?}

\comm{Hmm. I think that since the results we're using are all given in different ways (ie a limit on $\sigma^{\mathrm{vis}}$ for some, or a limit on $N_{NP}$, or even just the number of observed and predicted events (damn monoZ)), this would be better included in each channel-specific section instead. It may not be necessary to even include it though, I think it's reasonable to say 'we converted the limits to our own model' or whatever. - Amelia}

\end{flushleft}

\subsection{Signal Simulation}
\label{signal_generation}
\begin{flushleft}
Signal samples for each channel and for each simplified model discussed in Section \ref{sec:sec2} were generated in the following manner. Firstly, leading order matrix elements for the process $pp \rightarrow X + \chi\bar{\chi}$ (where $X$ is either one or two jets\footnote{For the monojet channel, jets are seeded by any parton excluding the (anti-)top quark.} or a Z boson) were modelled using \MG$\_$aMC$@$NLO v2.2.2 with the default PDF (MSTW2008lo68). The default renormalisation and factorisation scales were also used and set to the sum of $\sqrt{m^{2} + p_{T}^{2}}$ for all particles in the final state. Showering and hadronisation was then performed by \PYTHIA  8.201 with the default tune (ATLAS UE Tune AU2-MSTW2008LO). For the monojet channel, matching between \MG$\mbox{ }$and \PYTHIA was done using the MLM scheme with matching scales of 80 GeV and 300 GeV. Samples made with the latter matching scale are included both to increase statistics in the high $E_{T}^{miss}$ region and to account for the shape bias introduced by the 80 GeV matching scale. To avoid double-counting, events from the 80 (300) GeV monojet samples are required to have a leading jet with $p_{T} <$ ($\geq$) 350 GeV.
\bigskip

For each channel and simplified model, signal samples were generated for a representative set of dark matter and mediator masses, shown in Table \ref{Mass_coup_points}. \textcolor{magenta}{Something about the coupling strength, $f$, being set to a value of 1? Maybe leave out - we'll vary the value.} The mediator masses were chosen \comm{to reflect} predominantly three regimes: (near-)degenerate ($M\approx m_{\chi}$), kinematically allowed ($M \geq 2m_{\chi}$), and EFT-like ($\sqrt{\hat{s}} << M$)\footnote{A recent study by Alves et al. found that EFT results do not apply to mediators with a mass less than 2.5 TeV at the LHC during Run I \cite{Alves:2011wf}.}. We also allow for the possibility of a light mediator/heavy WIMP scenario (M < $m_{\chi}$). \comm{In addition to covering a broad parameter space, the masspoints in Table \ref{Mass_coup_points} also satisfy the narrow width approximation employed by \MG.} 
%Rows one to three of Table \ref{Mass_coup_points} cover the case where the dark matter mass is small and the mediator mass is either degenerate, low or sufficiently large so as to be in the EFT regime of validity\footnote{A recent study by Alves et al. found that EFT results do not apply to mediators with a mass less than 2.5 TeV at the LHC during Run I \cite{Alves:2011wf}.}. Rows four and five cover the case where the dark matter particle has a medium mass and the mediator is either near-degenerate or larger. Rows six and seven (eight and nine) cover the case where the dark matter particle is a few GeV (TeV) and the mediator is either near-degenerate or again sufficiently large so as to be in the EFT regime of validity.

\begin{table}[!htbp]
\centering
\begin{tabular}{l|l}
\hline
\hline
 \rule{0pt}{2.2ex}$m_{\chi}$ [GeV] & $M$ [GeV]\\
\hline
 \rule{0pt}{2.2ex} 1 & 1, 2, 10, 20, 100, 200, 1000, 2000, 20000 \\
 10 & 1, 2, 10, 20, 100, 200, 1000, 2000, 20000 \\
 100 & 1, 2, 10, 20, 100, 200, 1000, 2000, 20000 \\
 1000 & 1, 2, 10, 20, 100, 200, 1000, 2000, 20000 \\
% 1 & 10 & 10\\
% 2 & 10 & 200\\
% 3 & 10 & 25000\\
% 4 & 100 & 200\\
% 5 & 100 & 2000\\
% 6 & 500 & 600\\
% 7 & 500 & 25000\\
% 8 & 1000 & 2000\\
% 9 & 1000 & 25000\\
 \hline
 \hline
\end{tabular}
\caption{Mass points chosen for the analysis of simplified dark matter models. The mediator masses are primarily representative of three regimes: (near-)degenerate ($M\approx m_{\chi}$), kinematically allowed ($M \geq 2m_{\chi}$), and EFT-like ($\sqrt{\hat{s}} << M$).}
\label{Mass_coup_points}
\end{table}

%signal production was modelled with \MG$\_$aMC 2.2.2 \comm{(references needed throughout this section!)}. Showering and hadronisation was then performed by \PYTHIA  8.201 using the default tune (ATLAS UE Tune AU2-MSTW2008LO). Using the MSTW2008lo68 PDF set (currently the benchmark for ATLAS analyses \cite{}. \question{Is this still true?}) and the default \MG factorization and renormalization scales\footnote{In \FNMG5 2.2.2 the default scale, $\mu$ is defined as the sum of $\sqrt{m_{\chi}^{2} + p_{T}^{2}}$ for all dark matter particles in the final state.}, samples were produced for a representative set of of dark matter and mediator masses, shown in Table \ref{Mass_coup_points}. \textcolor{magenta}{Something about the coupling strength, $f$, being set to a value of 1? Maybe leave out - we'll vary the value.} Note that rows one to three of Table \ref{Mass_coup_points} cover the case where the dark matter mass is small and the mediator has either a mass that is degenerate, low or sufficiently large so as to be in the EFT regime of validity\footnote{A recent study by Alves et al. found that EFT results do not apply to mediators with a mass less than 2.5 TeV at the LHC during Run I \cite{Alves:2011wf}.}. Rows four and five cover the case where the dark matter particle has a medium mass and the mediator is either near-degenerate or larger. Rows six and seven (eight and nine) cover the case where the dark matter particle is a few GeV (TeV) and the mediator is either near-degenerate or again sufficiently large so as to be in the EFT regime of validity. \comm{Comment on the masspoints being dictated by the narrow width approximation used by MG?}
\end{flushleft}

%\fg{We now move to a discussion of each of the mono-X channels separately.}

\subsection{Monojet Constraints}
\label{monojet_constraints}
\begin{flushleft}
For the monojet channel, signal selection is carried out as in Ref. \cite{Aad:2015zva}: initially, events are required to have $E_{T}^{miss}>$ 150 GeV and at least one jet with $p_{T} >$ 30 GeV and $|\eta| <$ 4.5. The background contribution from QCD jets is managed with the requirement that all jets passing this initial selection criteria also satsify $|\Delta\phi(jet,{E}_{T}^{miss})|>1.0$. In order to isolate monojet-like topologies, surviving events are then required to have a leading jet, $j_{1}$, with $p_{T} >$ 120 GeV and $|\eta| <$ 2.0 satisfying $p_{T}^{j_{1}}/E_{T}^{miss} >$ 0.5. Lastly, nine separate signal regions are defined with increasing lower thresholds on $E_{T}^{miss}$ from 150 GeV to 700 GeV as shown in Table \ref{monojet_SRs}.

\begin{table}[!htbp]
\centering
\begin{tabular}{c|c|c|c}
 \hline
 \hline
 Signal Region & $E_{T}^{miss}$ threshold [GeV] & $N_{obs}$ & $N_{exp}$ \\%& $N_{obs}/\mathcal{L}$ & $N_{exp}/\mathcal{L}$ \\
 \hline
 SR1 & 150 & 14737.8 & 18980.5 \\%726 & 935 \\
 SR2 & 200 & 3938.2 & 5501.3 \\%194 & 271 \\
 SR3 & 250 & 1827 & 2151.8 \\%90 & 106 \\
 SR4 & 300 & 913.5 & 1035.3 \\%45 & 51 \\
 SR5 & 350 & 426.3 & 588.7 \\%21 & 29 \\
 SR6 & 400 & 243.6 & 345.1 \\%12 & 17 \\
 SR7 & 500 & 146.16 & 146.16 \\%7.2 & 7.2 \\
 SR8 & 600 & 77.14 & 73.08 \\%3.8 & 3.6 \\
 SR9 & 700 & 69.02 & 36.54 \\%3.4 & 1.8 \\
 \hline
 \hline
\end{tabular}
\caption{The ATLAS model-independent upper limits on the number of observed (expected) signal events, $N_{obs}$ ($N_{exp}$), at 95\% confidence level for the monojet channel signal regions. Adapted from Ref. \cite{Aad:2015zva}.}
\label{monojet_SRs}
\end{table}

%Monojet signal samples for each of the simplified models discussed in Section \ref{sec:sec2} and for each of the masspoints listed in Table \ref{Mass_coup_points} were generated following a similar procedure to that employed in the ATLAS analysis. Firstly, leading order matrix elements for the processes $pp \rightarrow j\chi\bar{\chi}$ and $pp \rightarrow jj\chi\bar{\chi}$ (where $j$ is a final state jet\footnote{In this analysis a jet is seeded by any parton excluding the (anti-)top quark.}) were generated using \MG$\_$aMC$@$NLO v2.2.2 with the default PDF (MSTW2008lo68). The default renormalisation and factorisation scales, $\mu_{R,F}$, which correspond to \comm{the geometric average of $m_{\chi}^{2} + p_{T, \chi}^{2}$}, were also used. Next, showering and hadronisation of the matrix elements was done using \PYTHIA \comm{8.185} with the default tune (ATLAS UE AU2-MSTW2008LO). Matching between \MG$\mbox{ }$and \PYTHIA was performed using the MLM scheme with matching scales of 80 GeV and 300 GeV. Samples made with the latter matching scale are included both to increase statistics in the high $E_{T}^{miss}$ region and to account for the shape bias introduced by the 80 GeV matching scale. To avoid double-counting, events from the 80 (300) GeV samples are required to have a leading jet with $p_{T} <$ ($\geq$) 350 GeV.
%\bigskip

For the monoejet channel, the event selection and Monte Carlo (MC) generation procedures discussed above were validated via reproduction of ATLAS limits on the suppression scale, M$_{*}$, for the s-channel vector simplified model (refered to as the Z' model in the ATLAS monojet analysis). The results of this process are contained in appendix \ref{monojet_validation}.
\bigskip

%\draft{While the signal samples used by the ATLAS group (and for validation) correspond to the processes $pp \rightarrow j\chi\bar{\chi}$ and $pp \rightarrow jj\chi\bar{\chi}$ where $j$ is a final state jet, monojet constraints on the simplified models studied in this paper were set using signal samples with zero, one and two jets in the final state.} \comm{This deviation from the ATLAS analysis is motivated by the observation that the limits on $M_{*}$ are improved with the addition of the process $pp \rightarrow \chi\bar{\chi}$.}
%However, while the ATLAS MC samples are generated for only two matching scale values (80 GeV and 300 GeV), we use a sliding value of $m_{\chi}/4$. \comm{This ensures that all jet-related kinematic distributions are smoothly connected across all dark matter masspoints.}

THIS IS WHERE I'M UP TO!

\end{flushleft}

\subsection{Mono-Z Constraints}
\label{monoZ_constraints}
\begin{flushleft}
\comm{Note to Amelia: Here you should discuss the original intention of the mono-Z analysis. Also comment on any validation you did in order to confirm that you could use the results of the analysis. Lastly, discuss or list the cuts used in the analysis and the uncertainties associated with the results. This will inevitably include details of the background estimation and the detector performance.}
\end{flushleft}

\subsection{Mono-W Constraints}
\begin{flushleft}
\comm{Note to Johanna: Here you should discuss the original intention of the mono-W analysis. Also comment on any validation you did in order to confirm that you could use the results of the analysis. Lastly, discuss or list the cuts used in the analysis and the uncertainties associated with the results. This will inevitably include details of the background estimation and the detector performance.}
\end{flushleft}

\begin{flushleft}
\textcolor{magenta}{This section should include:}
\begin{enumerate}
\item \textcolor{magenta}{A description of the general process employed to reinterpret monojet, mono-W and mono-Z limits. This should cover MC signal generation, the cuts used in each mono-X channel and validation of the procedure(s) used to generate the results.}
\item \textcolor{magenta}{Monojet specifics: motivation for why the SUSY results were used instead of the monojet+MET results.}
\item \textcolor{magenta}{A description of the assessment of the systematics.}
\item \textcolor{magenta}{The limit setting strategy (?).}
\end{enumerate}
\end{flushleft}
