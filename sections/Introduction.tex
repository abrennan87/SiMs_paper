% !TEX root = ../new_paper.tex

Simplified models have emerged as a powerful tool for the interpretation of collider, direct and indirect detection signals of dark matter (DM). Previously,  searches for DM were conducted within the context of both Effective Field Theories (EFTs) \cite{Aad:1363019, ATLAS-CONF-2012-147, CMS-PAS-EXO-12-048, Buckley:2013jwa, Abdallah:1472683, MonoZ, MonoX} and full UV-complete theories such as Supersymmetry \cite{ComppMSSM, Aad:2012ms, Aad:2012fqa, Aad:2014wea, SUSY_official_paper}. The latter approach, though well-motivated, is typified by a broad parameter space and generally yields results which are insensitive to the wider class of DM models. EFT constraints, in comparison, are applicable to a broad range of models and rely on the specification of only a small set of parameters, namely the suppression scale, $\Mstar$, and the DM mass, $\mDM$ \cite{}.
%This prescription however is known to be problematic in high-energy scenarios \cite{}, where the underlying simplification becomes invalid.
%\bigskip
In the EFT framework, interactions between the dark and Standard Model (SM) sector are parametrised by a set of higher-dimensional effective operators, that arise when the mass of the mediating particle is assumed to be significantly larger than the momentum transferred in a given interaction. Where this is not the case, the EFT prescription can produce constraints which detour dramatically from those of the associated UV-complete model \cite{Bai:2010hh, DMCons2, Fox:2011fx, Graesser:2011vj, An:2011ck}. This is not so important in direct detection experiments where the momentum transferred in the scattering of DM particles with heavy nuclei is generally of the order of tens of keV \cite{EFTDM, DMCons3}, or in indirect searches where the annihilations of non-relativistic DM particles in the galactic halo occur with momentum transfers of order $\mDM$ \cite{}. However, for hadron collider searches - where the accessible center of mass energy of two colliding baryons may be sufficient to produce the mediator on-shell - the range of validity of the EFT prescription is significantly diminished. Indeed, recent works have shown the EFT approach to be unreliable in certain cases for the interpretation of data collected during the $\sqrt{\hat{s}} =$ 8 TeV Run I of the Large Hadron Collider (LHC) \cite{}. In light of this, simplified models have become the preferred tool for the interpretation of collider DM searches \cite{Harris:2014hga, Buchmueller:2014yoa}.
%\bigskip

In a nutshell, a simplified model (SiM) arises when the heavy mediator which was integrated out in the EFT framework is reintroduced. Like EFTs, SiMs admit the comparison of results obtained in the different avenues of dark matter study \cite{} and are defined by a relatively small set of parameters - namely $\mDM$, the mass of the mediator $\Mmed$, and the SM-mediator and DM-mediator coupling strengths, $\gq$ and $\gX$ (or $\gqX$ in the case of a single, SM-DM-mediator coupling). Unlike EFTs, constraints calculated within the context of a SiM are valid across a broad energy range.

%To date, very few analyses include a dedicated study of any simplified models of DM. This is generally because the focus in DM collider searches at both ATLAS and CMS has been on generic EFT models. The recent release of several reports and recommendations on simplified models for the LHC $[$DM forum report, other SiMs paper$]$ indicate that they are expected to play a more prominent role in Run II. The aim of this work, then, is to investigate a phenomenologically distinct set of simplified models likely to be included in Run II searches, and to constrain these using results already publicly available.
In this paper, we examine a phenomenologically distinct set of SiMs. In particular, we place constraints on the SiMs corresponding to the simplest UV-completions of the D5 (vector) and D8 (axial-vector) effective operators in the $s$-channel\footnote{The D5 and D8 operators form a nice starting point in the analysis of simplified models as they have been studied exhaustively in the past (see Ref. \cite{}). This attention is motivated by the fact that collider limits for the D5 (D8) operator can be readily transformed into limits on spin-independent (spin-dependent) DM-nucleon scattering and vice versa. With the exception of D1 (see sec. \ref{section}), and D9 and D11 (which have no simple simplified model counterparts \cite{}), the remaining effective operators induce elastic scattering which is suppressed by powers of the DM velocity or the momentum transferred \cite{Kumar}. Hence, these operators are not often considered in the literature.}. We also include a case in which a scalar mediator is exchanged in the $t$-channel, motivated by its analog of squark exchange in Supersymmetry. In the heavy mediator limit, this model is converted into a combination of the D5 and D8 EFTs via a Fierz transformation, yet remains kinematically distinct from its $s$-channel counterparts.

The models are constrained using public results from \monoX + missing transverse energy ($\met$) searches conducted by the ATLAS Collaboration. Specifically we focus on searches where $X$ is either a parton (manifesting in the detector as a narrow-radius jet), a leptonically-decaying $Z$ boson, or a hadronically-decaying $W$ or $Z$ boson (manifesting as a large-radius jet). The purpose of this work is to strengthen existing SiM limits \cite{} using the full 20.3 $fb^{-1}$ of Run I ATLAS data, and to explore an enhanced phase space with respect to the mediator and DM masses and the relative strength of the couplings to the visible and dark sectors.
%; while the \monojet channel is expected to be most sensitive, the inclusion of two additional channels is expected to yield enhanced limits in combination.
We choose to treat the mediator width as the minimal value naturally arising, as an alternative to many past analyses which handle this as a fixed value \cite{}. We extend the study by providing a cross-check and comparison of the performance of the three targeted collider detection channels, and compare against relic density and direct detection constraints.

The remainder of the paper is organised as follows. Section \ref{sec:sec2} contains a compendium of the SiMs chosen for analysis and the associated collider phenomenology. Section \ref{sec:sec3} outlines the techniques used to recast \monoX + $\met$ limits on the visible cross-section for any new physics process into constraints on SiMs, and specifically on the couplings $\gq$ and $\gX$. Lastly, our results are presented in Section \ref{sec:sec4} along with a discussion of the implications of this work. Appendices \ref{Appendix_limitsetting} and \ref{Appendix_validation} include  details of the limit setting and analysis validation procedures.
%\textcolor{magenta}{This section should include:}
%\begin{enumerate}
%\item \textcolor{magenta}{Motivation for SiMs: limited validity of EFTs \& uneconomically broad parameter space of full UV-complete models/insensitivity of model-dependent results to the wider class of DM models $\rightarrow$ advantage of simplified models (relatively small parameter space, predictions accurate and complementary across a broad energy range $\mathcal{O}$(GeV-TeV)).\footnote{\textcolor{red}{Are we assuming that the reader understands the basics of a "simplified model"? Or should we have one or two sentences that summarise the differences between, say, an EFT and a SiM?}}}
%\item \textcolor{magenta}{Brief description of the aim of the analysis (e.g. something like "to apply constraints to $\mX$, $\mMed$ and $\gq = \gX = f$ for four phenomenologically distinct simplified models, etc) along with a brief motivation for the choice of simplified models\footnote{\textcolor{red}{Or should this part be in section \ref{sec:sec2}?}}.}
%\item \textcolor{magenta}{Summary of content of paper (eg. "This paper is organised as follows: section 2 is a compendium of the simplified models chosen for analysis..." etc)}
%\end{enumerate}
