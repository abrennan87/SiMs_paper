Simplified models have emerged as a powerful tool for the interpretation of collider, direct and indirect detection signals of dark matter (DM). Previously,  searches for DM were conducted within the context of both Effective Field Theories (EFTs) \cite{Aad:1363019, ATLAS-CONF-2012-147, CMS-PAS-EXO-12-048, Buckley:2013jwa, Abdallah:1472683, MonoZ, MonoX} and full UV-complete theories like Supersymmetry \cite{ComppMSSM, Aad:2012ms, Aad:2012fqa, Aad:2014wea, SUSY_official_paper} and extra dimensions \cite{}. The latter approach, though well-motivated, is typified by a broad parameter space and generally yields results which are insensitive to the wider class of DM models. EFT constraints, in comparison, are applicable to a broad range of models and rely on the specification of only a small set of parameters, namely the suppression scale, $M_{\star}$ and the DM mass, $m_{DM}$ \cite{}. 
%This prescription however is known to be problematic in high-energy scenarios \cite{}, where the underlying simplification becomes invalid.
%\bigskip
In the EFT framework, interactions between the dark and Standard Model (SM) sector are parametrised by a set of higher-dimensional effective operators \cite{}. These operators arise when the mass of the mediating particle is assumed to be significantly larger than the momentum transferred in a given interaction \cite{}. Where this is not the case, the EFT prescription can produce constraints which detour dramatically from those of the associated UV-complete model \cite{Bai:2010hh, DMCons2, Fox:2011fx, Graesser:2011vj, An:2011ck}. This is not so important in direct detection experiments where the momentum transferred in the scattering of DM particles with heavy nuclei is generally of the order of tens of keV \cite{EFTDM, DMCons3}. Similarly, in indirect searches the annihilations of non-relativistic DM particles in the galactic halo occur with momentum transfers on the order of $m_{DM}$ \cite{}. However, for collider searches, where the accessible center of mass energy of two colliding baryons may be sufficient to produce the mediator on-shell, the range of validity of the EFT approach is significantly diminished \cite{}. Indeed, recent works (\cite{}) have shown the EFT approach to be unreliable in some cases for the $\sqrt{\hat{s}} =$ 8 TeV Run I of the Large Hadron Collider (LHC). Furthermore, the problem is expected to worsen in the upcoming 13/14 TeV Run II. So, to accurately probe this regime, we move to Simplified Models \cite{Buchmueller:2014yoa}.
%\bigskip

In a nutshell, a simplified model arises when the heavy mediator which was integrated out in the EFT framework is reintroduced. Like EFTs, simplified models admit the comparison of results obtained in the different avenues of dark matter study \cite{} and are defined by a relatively small set of parameters - namely $m_\chi$, the mass of the mediator, $M$, and the SM-mediator and DM-mediator coupling strengths, $g_{q}$ and $g_\chi$. Unlike EFTs, constraints calculated within the context of a simplified model are valid across a broad energy range ($\mathcal{O}$(GeV-TeV)).
%, removing the question of validity of the EFT approximation \cite{}. 
%\bigskip

To date, very few analyses include a dedicated study of any simplified models of DM. This is generally because the focus in DM collider searches at both ATLAS and CMS has been on generic EFT models. The recent release of several reports and recommendations on simplified models for the LHC $[$DM forum report, other SiMs paper$]$ indicate that they are expected to play a more prominent role in Run II. The aim of this work, then, is to investigate a phenomenologically distinct set of simplified models likely to be included in Run II searches, and to constrain these using results already publicly available. In particular, constraints are placed on the simplified models corresponding to the simplest UV-completions in the $s$-channel of the D5 (vector) and D8 (axial-vector) effective operators\footnote{The D5 and D8 operators form a nice starting point in the analysis of simplified models as they have been studied exhaustively in the past (see Ref. \cite{}). This attention is motivated by the fact that collider limits for the D5 (D8) operator can be readily transformed into limits on spin-independent (spin-dependent) DM-nucleon scattering and vice versa. With the exception of D1 (see sec. \ref{section}), and D9 and D11 (which have no simple simplified model counterparts \cite{}), the remaining effective operators induce elastic scattering which is suppressed by powers of the DM velocity or the momentum transferred \cite{Kumar}. Hence, these operators are largely ignored in the literature.}. We also include a case with mediator exchange in the $t$-channel, which approaches the vector EFT model in the heavy-mediator limit, but remains kinematically distinct from its $s$-channel counterpart.
%\st{These models are constrained using publically available and model-independent limits on the visible cross-section for new physics processes originating from mono-X\footnote{Where X is a W/Z boson or a parton.} + missing transverse energy ($E_{T}^{miss}$) searches conducted by the ATLAS Collaboration.} 
We constrain these models using public results from mono-X + missing transverse energy ($\met$) searches conducted by the ATLAS Collaboration. In particular, we focus on searches where X is either a parton (appearing as a jet) or a $Z$ boson (decaying to two leptons). The purpose of this approach is both to enhance and update existing simplified model limits \cite{} and to provide a cross-check of the channels' performance. Additionally, we further extend the study of simplified models by allowing the width of the mediator and the SM-/DM-mediator couplings to vary, which until now have been fixed quantities \cite{}. We also \comm{something} by including a comparison of collider limits with relic density constraints.
%\bigskip

%\textcolor{blue}{\st{The re-interpretation of theses results is a relatively simple task, requiring only the simulation of the signal associated with the new model for each point in parameter space (see Section 3).}} \comm{I wouldn't bother saying this at all - Amelia}
%\hspace{1cm}\textcolor{red}{Should we say something here about the specifics of our limits? E.g. Something like "Limits are applied to the cross-section for the process $pp \rightarrow X + \chi\bar{\chi}$ and to $g_{q} = g_{DM} = f$"?}
%\comm{Yep! Something like below? - Amelia}
%\fg{We calculate upper limits on the production cross section for each simplified model and mono-X channel, and rework these into upper limits on the coupling(s) for each set of $m_{\chi}$ and $M$.}
%%% I'm no longer sure this goes here, instead I've mentioned below ('in section blah') that we limit the couplings, and dropped the bit about the XSecs.

\comm{We should say something about how this differs from similar constraints on simplified models (different range of models, search channels, etc.) - Tom}

\comm{I agree - some points are: extended the set of models and channels studied (should point out that mono-jet will dominate, but that these could enhance the limits in combination), the mono-jet results that everyone uses has been updated (ie 10$fb^{-1} \rightarrow 20.3 fb^{-1}$), minimum width, study of varied couplings (both have been done in Tom+Karl paper of course), and inclusion of relic density constraints. Now just need to write the damn sentence. - Amelia}

\comm{Also make point here about using two channels.}

The remainder of the paper is organised as follows. Section \ref{sec:sec2} contains a compendium of the simplified models chosen for analysis. Section \ref{sec:sec3} outlines the technique used to convert mono-X + $\met$ limits on the visible cross-section for any new physics process into constraints on simplified models, specifically, the couplings $g_q$ and $g_{\chi}$. Lastly, the results are presented in Section \ref{sec:sec4} along with a discussion of the implications of this work.
%\textcolor{magenta}{This section should include:}
%\begin{enumerate}
%\item \textcolor{magenta}{Motivation for SiMs: limited validity of EFTs \& uneconomically broad parameter space of full UV-complete models/insensitivity of model-dependent results to the wider class of DM models $\rightarrow$ advantage of simplified models (relatively small parameter space, predictions accurate and complementary across a broad energy range $\mathcal{O}$(GeV-TeV)).\footnote{\textcolor{red}{Are we assuming that the reader understands the basics of a "simplified model"? Or should we have one or two sentences that summarise the differences between, say, an EFT and a SiM?}}}
%\item \textcolor{magenta}{Brief description of the aim of the analysis (e.g. something like "to apply constraints to $m_{\chi}$, $m_{M}$ and $g_{q} = g_{\chi} = f$ for four phenomenologically distinct simplified models, etc) along with a brief motivation for the choice of simplified models\footnote{\textcolor{red}{Or should this part be in section \ref{sec:sec2}?}}.}
%\item \textcolor{magenta}{Summary of content of paper (eg. "This paper is organised as follows: section 2 is a compendium of the simplified models chosen for analysis..." etc)}
%\end{enumerate}
