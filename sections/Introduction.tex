\begin{flushleft}
Simplified models of dark matter have emerged as a powerful tool for the interpretation of collider, direct and indirect detection signals of dark matter. Previous searches for dark matter were conducted within the context of both Effective Field Theories (EFTs) \cite{Aad:1363019, ATLAS-CONF-2012-147, CMS-PAS-EXO-12-048, Buckley:2013jwa, Abdallah:1472683, MonoZ, MonoX} and full UV-complete theories like Supersymmetry \cite{ComppMSSM, Aad:2012ms, Aad:2012fqa, Aad:2014wea, SUSY_official_paper} and extra dimensions \cite{}. The latter approach, though well-motivated, is typified by a broad parameter space and generally yields results which are insensitive to the wider class of dark matter models. EFT constraints, in comparison, are applicable to a broad range of models and rely on the specification of only a small set of parameters, namely the suppression scale, $\Lambda$ and the dark matter mass, $m_{DM}$ \cite{}.

\hspace{1cm}In the EFT framework, interactions between the dark and Standard Model (SM) sector are parametrised by a set of higher-dimensional effective operators \cite{}. These operators arise when the mass of the mediating particle is assumed to be significantly larger than the momentum transferred in a given interaction \cite{}. Where this is not the case, the EFT prescription can produce constraints which detour dramatically from those of the associated UV-complete model \cite{Bai:2010hh, DMCons2, Fox:2011fx, Graesser:2011vj, An:2011ck}. This is not so important in direct detection experiments where the momentum transferred in the scattering of dark matter particles with heavy nuclei is on the order of tens of keV \cite{EFTDM, DMCons3}. Similarly, in indirect searches the annihilations of non-relativistic dark matter particles in the galactic halo occur with momentum transfers on the order of $m_{DM}$ \cite{}. However, for collider searches, where the accessible center of mass energy of two colliding baryons may be sufficient to produce the mediator on-shell, the range of validity of the EFT approach is rapidly diminishing \cite{}. 
To accurately probe outside this range it is now a popular practise to use simplified Models \cite{Buchmueller:2014yoa}.

\hspace{1cm}In a nutshell, simplified models arise when the heavy mediator which was integrated out in the EFT framework is reintroduced. Like EFTs, simplified models admit the comparison of results obtained in the different dark matter detection avenues \cite{} and are defined by a relatively small set of parameters - namely $m_{DM}$, the mass of the mediator, $M$, and the SM-mediator and DM-mediator coupling strengths, $g_{q}$ and $g_{DM}$. Unlike EFTs, constraints calculated within the context of a simplified model are accurate across a broad energy range ($\mathcal{O}$(GeV-TeV)), removing the question of validity \cite{}. 

\hspace{1cm}The goal of this paper is not to conduct a full survey of all possible simplified models but rather to investigate a small handful of phenomenologically distinct cases. In particular, constraints are placed on the simplified models corresponding to the most straightforward UV-completions of the D1, D5 and D8 effective operators. %A t-channel resolution of the D1 operator is also included for completeness. 
These models are constrained using publically available and model-independent limits on the visible cross-section for new physics processes originating from mono-X\footnote{Where X is a W/Z boson or a parton.} + missing transverse energy ($E_{T}^{miss}$) searches conducted by the ATLAS Collaboration.  \textcolor{blue}{The re-interpretation of theses results is a relatively simple task, requiring only the simulation of the signal associated with the new model for each point in parameter space (see Section \ref{sec:sec3}).}

\hspace{1cm}\textcolor{red}{Should I say something here about the specifics of our limits? E.g. Something like "Limits are applied to the cross-section for the process $pp \rightarrow X + \chi\bar{\chi}$ and to $g_{q} = g_{DM} = f$"?}

\hspace{1cm}The remainder of the paper is organised as follows. Section \ref{sec:sec2} contains a compendium of the simplified models chosen for analysis. Section \ref{sec:sec3} outlines the technique used to convert mono-X + $E_{T}^{miss}$ limits on the visible cross-section for any new physics process into constrains on simplified models. \textcolor{blue}{Lastly, the results are presented in Section \ref{sec:sec4} along with a discussion of the implications of this research.}
\bigskip

\textcolor{magenta}{Include:}
\begin{enumerate}
\item \textcolor{magenta}{Motivation for SiMs: limited validity of EFTs \& uneconomically broad parameter space of full UV-complete models/insensitivity of model-dependent results to the wider class of DM models $\rightarrow$ advantage of simplified models (relatively small parameter space, predictions accurate and complementary across a broad energy range $\mathcal{O}$(GeV-TeV)).\footnote{\textcolor{red}{Are we assuming that the reader understands the basics of a "simplified model"? Or should we have one or two sentences that summarise the differences between, say, an EFT and a SiM?}}}
\item \textcolor{magenta}{Brief description of the aim of the analysis (e.g. something like "to apply constraints to $m_{\chi}$, $m_{M}$ and $g_{q} = g_{\chi} = f$ for four phenomenologically distinct simplified models, etc) along with a brief motivation for the choice of simplified models\footnote{\textcolor{red}{Or should this part be in section \ref{sec:sec2}?}}.}
\item \textcolor{magenta}{Summary of content of paper (eg. "This paper is organised as follows: section 2 is a compendium of the simplified models chosen for analysis..." etc)}
\end{enumerate}
\end{flushleft}