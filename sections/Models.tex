\begin{flushleft}
We begin with a short set of assumptions: that the DM particle, $\chi$, is a weakly interacting Dirac fermion, that it is a singlet under the SM, and that it is the lightest stable new particle. We also require minimal flavour violation (MVF) to hold wherever relevant. (\comm{Not sure if this is relevant anymore, if we've ditched the scalar model. - Mia}) Each model is built around a scenario whereby $\chi$ and SM quarks are coupled via a mediator. Coupling to SM leptons \cite{} or gluons \cite{} is beyond the scope of this paper, but these cases have been studied elsewhere. Resolving the contact interaction of an EFT at tree-level leads to two possibilities: the mediating particle is exchanged in the $s$-channel, in which case it is also a SM singlet, or it is exchanged in the $t$-channel, in which case it is necessarily charged and coloured. With these assumptions in mind, two $s$-channel models and one $t$-channel model were chosen for analysis. 

The $s$-channel models are characterised by vector (sV) or axial-vector (sA) couplings to both the dark and SM sectors. In the notation of Ref.~\cite{DMCons2}, these correspond to the D5 and D8 operators respectively in the EFT regime. These models are described by the following interaction Lagrangians:
\begin{equation}
\label{L_int_sV}
\mathcal{L}_{sV} = - \xi_{\mu}\left[ \sum\limits_{q} g_{q}\bar{q}\gamma^{\mu}q - g_{\chi}\bar{\chi}\gamma^{\mu}\chi\right],
\end{equation}
\begin{equation}
\label{L_int_sA}
\mathcal{L}_{sA} =  \xi_{\mu}\left[\sum\limits_{q} g_{q}\bar{q}\gamma^{\mu}\gamma_{5}q - g_{\chi}\bar{\chi}\gamma^{\mu}\gamma_{5}\chi\right],
\end{equation}
%\begin{equation}
%\label{L_int_sS}
%\mathcal{L}_{sS} = \sum\limits_{i=1}^{6} g_{q_i}\bar{q_i}q_i\eta + g_{\chi}\bar{\chi}\chi\eta,
%\end{equation}
\st{where the sum is over all quarks, $\xi$ is the (axial-)vector mediator and $\chi$ is the dark matter particle - a weakly interacting SM singlet Dirac fermion with mass $m_{\chi}$.}
%\question{The scalar model should really have a Yukawa coupling (ie $g_q \rightarrow g_q m_q / v$) if we want to be consistent with D1 and the ATLAS/CMS run2 simplified models. Is it too late to implement this? - Tom.} \comm{I agree, and this should be easy enough to implement I think. - Amelia} 
%\st{For simplicity, it is assumed that both $\xi$ and $\eta$ are SM singlet particles and couple only to pair-antipair combinations of $\chi$ and the SM quarks, with coupling strengths $g_{\chi}$ and $g_q$ respectively.} 
For the couplings $g_q$ and $g_{\chi}$ to remain within the perturbative regime, they are required to satisy $g_{q},g_{\chi} \leq 4\pi$, though stronger perturbativity requirements do exist \cite{ValidEFT}.
\bigskip

%\st{Finally, for convenience, the mediator is assumed to couple to }$\cancel{\chi}$\st{ with the same strength as to each of the six quarks, that is, }$\cancel{g_{q} = g_{\chi} \equiv f}$. \comm{Removed, since we want to include cases where this isn't true! - Amelia} 
%Note that the vector and axial-vector models have already been studied extensively \cite{Buchmueller:2014yoa, Chatrchyan:2013qha, Aad:2012hf, Harris:2014hga} and so serve as a good basis for the comparison of constraints. \comm{I need to add some citations for the scalar model here - Tom}

The last model considered in this paper, a $t$-channel scalar mediator model (which we refer to by the descriptor tS), juxtaposes nicely with the $s$-channel models. In the heavy-mediator limit, it is converted into a combination of the D5 and D8 EFT operators via Fierz transformation. In addition, the tS model is motivated by analogy with a common aspect of Supersymmetric models: neutralino DM interacting with the SM sector via $t$-channel exchange of a squark\footnote{Note that in the Supersymmetric scenario the DM particle is a Majorana fermion. Simplified models in which the DM particle is a Majorana fermion are not covered here (the exception being in the validation of the mono-$Z$ channel, see sec.\ref{monoZ_validation}) as they are kinematically identical to the corresponding Dirac cases, and only require multiplication of the cross-section by a simple factor in order to calculate limits. The exception to this rule is the $s$-channel vector mediator model, which vanishes if $\chi$ is a Majorana fermion \cite{METSig}.} \cite{SUSYDM}.
\bigskip

In this model, the mediator which we call $\phi$ necessarily has colour charge, and can couple to either the left or right-handed quarks as a SU(2) doublet or singlet respectively. Since the LHC is insensitive to the chirality of the quarks, for simplicity we assume that the mediator couples to the left-handed quarks only, that the masses and couplings of $\phi$ are equal across the three generations, and that the masses of the two components of $\phi$ are equal. The interaction Lagrangian for this model is then:
\begin{equation}
\label{L_int_tS}
\mathcal{L}_{int} = \sum_{Q} g_{q \chi} \bar{Q} P_R \phi \chi + {\rm h.c.},
\end{equation}
where the sum is over the three $Q_L$ doublets (\comm{does this make sense?}), $g_{q \chi}$ is the scalar coupling of the incoming quark, $\phi$ and $\chi$, and $P_R$ is the usual chiral projection operator. \question{Have you included the projection operator in the code? - Tom}

\iffalse
\hspace{1cm}\textcolor{blue}{\st{When constraining the above models using the collider production cross-section for the process }$\cancel{pp \rightarrow X + \bar{\chi}\chi}$\st{ it is also important to consider the impact of the mediator decay width, }$\cancel{\Gamma}$ \cite{}\st{. The reason being that }$\cancel{\sigma\left(pp \rightarrow X + \bar{\chi}\chi\right)}$\st{ scales as:}}
\begin{equation}
\label{cross_section}
\cancel{\frac{f^{4}E^{2}}{(q^{2} - M^{2})^{2} + M^2 \Gamma^{2}}}
\end{equation}
\st{where }$E$\st{ is the centre-of-mass energy of two colliding partons and }$q$\st{ is the four-momentum transferred by the mediator.} \comm{Careful with formulae from other papers, it's caused me enormous headaches in the past when they were wrong :) The formulae we quoted here from \cite{METSig} didn't even have the consistent dimensions! Somehow they dropped the wrong term when squaring the propagator! - Tom.} 

\fi

\subsection{Width effects}

An important factor when considering simplified models is to ensure the mediator width is treated appropriately, as it impacts both the cross-section calculation and, in some cases, the kinematic behaviour of the model. In previous analyses it has been customary to consider mediators of fixed width ranging from $\Gamma = M/8\pi$ to $\Gamma = M/3$ \cite{METSig, Fox:2012ee} \footnote{\textcolor{magenta}{Also add a reference for the recent monojet paper.}}. This approach is motivated by the observation that, where the mediator is exchanged in the $s$-channel and produced on-shell, $\sigma\left(pp \rightarrow X + \bar{\chi}\chi\right)$ is maximally enhanced when $\Gamma$ is small\footnote{For a more in depth discussion see Ref. \cite{METSig}.}. The smallest width, $\Gamma = M/8\pi$, corresponds to a mediator which couples only to one helicity and flavour of quark with $g_{q} = 1$ \cite{METSig}. In this paper, the mediator widths are expanded to include coupling to all kinematically accessible quark flavours. We also allow for flavour-blind coupling (when not restricted by MFV), so our definition of the minimum width must change to reflect this. Following $[$(other minimum width papers)$]$, the minimum width for each model is given by:
\begin{equation}
\label{gamma_sV}
\Gamma_{sV} = \frac{g_{\chi}^2 M}{12\pi}\left(1 + \frac{2m_{\chi}^{2}}{M^{2}}\right)\left(1 - \frac{4m_{\chi}^{2}}{M^{2}}\right)^{1/2} + \sum_{\substack{q}}\frac{g_q^2M}{4\pi}\left(1 + \frac{2m_{q}^{2}}{M^{2}}\right)\left(1 - \frac{4m_{q}^{2}}{M^{2}}\right)^{1/2}
\end{equation}
\begin{equation}
\label{gamma_sA}
\Gamma_{sA} = \frac{g_{\chi}^2 M}{12\pi}\left(1 - \frac{4m_{\chi}^{2}}{M^{2}}\right)^{3/2} + \sum_{\substack{q}}\frac{g_q^2 M}{4\pi}\left(1 - \frac{4m_{q}^{2}}{M^{2}}\right)^{3/2}
\end{equation}
%\begin{equation}
%\label{gamma_sS}
%\Gamma_{sS} = \frac{\fg{g_{\chi}^2} M}{8\pi}\left(1 - \frac{4m_{\chi}^{2}}{M^{2}}\right)^{3/2} + \sum_{\substack{q}}\frac{3\fg{g_q^2} M}{8\pi}\left(1 - \frac{4m_{q}^{2}}{M^{2}}\right)^{3/2}
%\end{equation}
\begin{equation}
\label{gamma_tS}
\Gamma_{tS} = \sum_{\substack{q}} \frac{g_{q\chi}^2M}{16\pi}\left(1 - \frac{m_{q}^{2}}{M^{2}} - \frac{m_{\chi}^{2}}{M^{2}}\right)\sqrt{\left(1 - \frac{m_{q}^{2}}{M^{2}} + \frac{m_{\chi}^{2}}{M^{2}}\right)^{2} - 4\frac{m_{\chi}^{2}}{M^{2}}}
\end{equation}
where $m_{\chi} \leq M/2$ is required for the $s$-channel models and $M \geq m_{\chi} + m_{q}$ is required for the $t$-channel model. Note also that the summation over $q$ is restricted to quarks with $m_{q} \leq M/2$ for all models. \comm{We need to extend this discussion; See e.g. my paper with Karl, 1502.05721; The width expression above is only valid for $\Gamma < M$, which doesn't always hold true for Run 1 data - Tom.}

\comm{I agree - I might try to implement something where we don't test a point if the minimum width is larger than the mediator mass, or at least reduce it by scaling down the coupling, so that we always avoid this issue. - Amelia}

%\textcolor{magenta}{This section should include:}
%\begin{enumerate}
%\item \textcolor{magenta}{Brief motivation for choice of simplified models? (eg. something like "we consider the most straightforward UV-completions of the D1, D5 and D8 effective operators, corresponding to the s-channel scalar, vector and axial-vector models respectively.}
%\item \textcolor{magenta}{The interaction Lagrangians for our four SiMs along with an explanation for why we only assume coupling to SM quarks.}
%\item \textcolor{magenta}{The assumptions and the decay widths associated with our models (?).}
%\item \textcolor{magenta}{Comments on the requirement that $\sqrt{g_{q}g_{\chi}} \leq 4\pi$ in order for the theory to remain perturbative? $\rightarrow$ comments on the choice of mass and coupling points used? Or does this belong in section \ref{sec:sec3}?}
%\end{enumerate}
\end{flushleft}

%Resolving the mediator leads to two possibilities: the mediating particle is exchanged in the s-channel, in which case it may be colour neutral, or it is exchanged in the t-channel in which case it is necessarily coloured \cite{}. 

%\cite{ValidEFT, BeyondEFT, CSUSY} t-channel \cite{Buchmueller:2014yoa, SiM}.
