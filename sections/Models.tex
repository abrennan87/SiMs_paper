\begin{flushleft}
Resolving the contact interaction of an EFT leads to two possibilities: the mediating particle is exchanged in the $s$-channel, in which case it may be colour neutral, or it is exchanged in the $t$-channel in which case it is necessarily coloured to conserved colour flow between the quarks. With comprehensiveness in mind, three $s$-channel models and a $t$-channel model were chosen for analysis. The $s$-channel models are characterised by vector (sV), axial-vector (sA) and scalar (sS) couplings to both the dark and SM sectors. In the notation of Ref.~\cite{DMCons2}, these correspond to the D5, D8 and D1 operators\footnote{The D1, D5 and D8 operators form a nice starting point in the analysis of simplified models as they have been studied exhaustively in the past (see Ref. \cite{}).  This attention is motivated by the fact that collider limits for the D1 and D5 (D8) operators can be readily transformed into limits on spin-independent (spin-dependent) dark matter-nucleon scattering and vice versa. With the exception of D9 and D11 (which have no simple - \textcolor{magenta}{read: tree-level} - simplified model counterparts \cite{}), the remaining effective operators induce elastic scattering which is suppressed by powers of the dark matter velocity or the momentum transferred \cite{Kumar}. Hence, these operators are largely ignored in the literature.}  respectively in the EFT regime ($\sqrt{\hat{s}} \ll M$). \comm{I've changed  AV to A, to avoid confusion with $\bar\chi \gamma^\mu (1-\gamma^5)\chi$ type models. Is that ok? - Tom.} These models are described by the following interaction Lagrangians:
\begin{equation}
\label{L_int_sV}
\mathcal{L}_{sV} = - \xi_{\mu}\left[ \sum\limits_{i=1}^{6} g_{q_i}\bar{q_i}\gamma^{\mu}q_i - g_{\chi}\bar{\chi}\gamma^{\mu}\chi\right],
\end{equation}
\begin{equation}
\label{L_int_sA}
\mathcal{L}_{sA} =  \xi_{\mu}\left[\sum\limits_{i=1}^{6} g_{q_i}\bar{q_i}\gamma^{\mu}\gamma_{5}q_i - g_{\chi}\bar{\chi}\gamma^{\mu}\gamma_{5}\chi\right],
\end{equation}
\begin{equation}
\label{L_int_sS}
\mathcal{L}_{sS} = \sum\limits_{i=1}^{6} g_{q_i}\bar{q_i}q_i\eta + g_{\chi}\bar{\chi}\chi\eta,
\end{equation}
where $i$ represents an index running over quark \st{family} flavour, $\xi$ is the (axial-)vector mediator and $\eta$ is the scalar mediator, and $\chi$ is the dark matter particle - a weakly interacting SM singlet Dirac fermion with mass $m_{\chi}$. \question{The scalar model should really have a Yukawa coupling (ie $g_q \rightarrow g_q m_q / v$) if we want to be consistent with D1 and the ATLAS/CMS run2 simplified models. Is it too late to implement this? - Tom.} For simplicity, it is assumed that both $\xi$ and $\eta$ are SM singlet particles and couple only to pair-antipair combinations of $\chi$ and the SM quarks, with coupling strengths $g_{\chi}$ and $g_q$ respectively. For the couplings to remain within the perturbative regime, $g_{q},g_{\chi}$ must be less than or equal to $4\pi$, \draft{though stronger perturbativity requirements can exist} \cite{ValidEFT}. \textcolor{blue}{Finally, for convenience, the mediator is assumed to couple to $\chi$ with the same strength as to each of the six quarks, that is, $g_{q} = g_{\chi} \equiv f$.} Note that the vector and axial-vector models have already been studied extensively \cite{Buchmueller:2014yoa, Chatrchyan:2013qha, Aad:2012hf, Harris:2014hga} and so serve as a good basis for the comparison of constraints. \comm{I need to add some citations for the scalar model here - Tom}

\hspace{1cm}The last model considered in this paper - the $t$-channel UV-completion of the D1 operator (denoted tS) - juxtaposes nicely with the $s$-channel models and is motivated separately by analogy with a common aspect of Supersymmetric models: neutralino dark matter interacting with the SM sector via $t$-channel exchange of a squark\footnote{Note that in the Supersymmetric scenario the dark matter particle is a Majorana fermion. Simplified models in which the dark matter particle is a Majorana fermion are not covered here as they are not expected to produce significantly different results, except in the case of the $s$-channel vector mediator model which vanishes if $\chi$ is a Majorana fermion \cite{METSig}.} \cite{SUSYDM}. 

In this model, the mediator which we call $\eta'$ must necessarily have colour charge, and can couple to the either the left or right-handed quarks as a SU(2) doublet or singlet respectively. Since the LHC is insensitive to the chirality of the quarks, for simplicity we assume that the mediator couples to the left-handed quarks only, that the masses and couplings of $\eta'$ are equal across the three generations, and that the masses of the two components $\eta'$ are equal. %
%
The interaction Lagrangian for this model is then
\begin{equation}
\label{L_int_tS}
\mathcal{L}_{int} = \sum_{i=1}^3 g_{Q_i} \bar{Q_i} P_R \eta' \chi + {\rm h.c.},
\end{equation}
where the quark doublets are represented by $Q = \{ (u,d), \, (c, s), \, (t, b) \}$, and $P_R$ is the usual chiral projection operator. \question{Have you included the projection operator in the code? - Tom}

\hspace{1cm}\textcolor{blue}{When constraining the above models using the collider production cross-section for the process $pp \rightarrow X + \bar{\chi}\chi$ it is also important to consider the impact of the mediator decay width, $\Gamma$ \cite{}. The reason being that $\sigma\left(pp \rightarrow X + \bar{\chi}\chi\right)$ scales as:}
\begin{equation}
\label{cross_section}
\frac{f^{4}E^{2}}{(q^{2} - M^{2})^{2} + M^2 \Gamma^{2}}
\end{equation}
where $E$ is the centre-of-mass energy of two colliding partons and $q$ is the four-momentum transferred by the mediator. \comm{Careful with formulae from other papers, it's caused me enormous headaches in the past when they were wrong :) The formulae we quoted here from \cite{METSig} didn't even have the consistent dimensions! Somehow they dropped the wrong term when squaring the propagator! - Tom.} In previous analyses it was customary to consider mediators of fixed width ranging from $\Gamma = M/8\pi$ to $\Gamma = M/3$ \cite{METSig, Fox:2012ee} \footnote{\textcolor{magenta}{Also add a reference for the recent monojet paper.}}. This approach is motivated by the observation that, where the mediator is exchanged in the $s$-channel and produced on-shell, $\sigma\left(pp \rightarrow X + \bar{\chi}\chi\right)$ is maximally enhanced when $\Gamma$ is small, as expected from equation \ref{cross_section}\footnote{For a more in depth discussion see Ref. \cite{METSig}.}. The smallest width, $\Gamma = M/8\pi$, corresponds to a mediator which couples  to only one helicity and flavour of quark with $g_{q} = 1$ \cite{METSig}. In this paper, the mediator widths are expanded to include coupling to all kinematically accessible quark flavours (in addition to the single flavour case). This move is based on the \emph{a priori} assumption that the mediators couple equally to each quark and so should be allowed to decay into quark pairs of any flavour. For the above models, the full decay widths are:
\begin{equation}
\label{gamma_sV}
\Gamma_{sV} = \frac{f^{2} M}{12\pi}\left(1 + \frac{2m_{\chi}^{2}}{M^{2}}\right)\left(1 - \frac{4m_{\chi}^{2}}{M^{2}}\right)^{1/2} + \sum_{\substack{q}}\frac{f^{2}M}{4\pi}\left(1 + \frac{2m_{q}^{2}}{M^{2}}\right)\left(1 - \frac{4m_{q}^{2}}{M^{2}}\right)^{1/2}
\end{equation}
\begin{equation}
\label{gamma_sA}
\Gamma_{sA} = \frac{f^{2} M}{12\pi}\left(1 - \frac{4m_{\chi}^{2}}{M^{2}}\right)^{3/2} + \sum_{\substack{q}}\frac{f^{2} M}{4\pi}\left(1 - \frac{4m_{q}^{2}}{M^{2}}\right)^{3/2}
\end{equation}
\begin{equation}
\label{gamma_sS}
\Gamma_{sS} = \frac{f^{2} M}{8\pi}\left(1 - \frac{4m_{\chi}^{2}}{M^{2}}\right)^{3/2} + \sum_{\substack{q}}\frac{3f^{2} M}{8\pi}\left(1 - \frac{4m_{q}^{2}}{M^{2}}\right)^{3/2}
\end{equation}
\begin{equation}
\label{gamma_tS}
\Gamma_{tS} = \sum_{\substack{q}} \frac{f^{2}M}{16\pi}\left(1 - \frac{m_{q}^{2}}{M^{2}} - \frac{m_{\chi}^{2}}{M^{2}}\right)\sqrt{\left(1 - \frac{m_{q}^{2}}{M^{2}} + \frac{m_{\chi}^{2}}{M^{2}}\right)^{2} - 4\frac{m_{\chi}^{2}}{M^{2}}}
\end{equation}
where $m_{\chi} \leq M/2$ is required for the $s$-channel models and $M \geq m_{\chi} + m_{q}$ is required for the $t$-channel model. Note also that the summation over $q$ is restricted to quarks with $m_{q} \leq M/2$ for all models. \comm{We need to extend this discussion; See e.g. my paper with Karl, 1502.05721; The width expression above is only valid for $\Gamma < M$, which doesn't always hold true for Run 1 data - Tom.}

%\textcolor{magenta}{This section should include:}
%\begin{enumerate}
%\item \textcolor{magenta}{Brief motivation for choice of simplified models? (eg. something like "we consider the most straightforward UV-completions of the D1, D5 and D8 effective operators, corresponding to the s-channel scalar, vector and axial-vector models respectively.}
%\item \textcolor{magenta}{The interaction Lagrangians for our four SiMs along with an explanation for why we only assume coupling to SM quarks.}
%\item \textcolor{magenta}{The assumptions and the decay widths associated with our models (?).}
%\item \textcolor{magenta}{Comments on the requirement that $\sqrt{g_{q}g_{\chi}} \leq 4\pi$ in order for the theory to remain perturbative? $\rightarrow$ comments on the choice of mass and coupling points used? Or does this belong in section \ref{sec:sec3}?}
%\end{enumerate}
\end{flushleft}

%Resolving the mediator leads to two possibilities: the mediating particle is exchanged in the s-channel, in which case it may be colour neutral, or it is exchanged in the t-channel in which case it is necessarily coloured \cite{}. 

%\cite{ValidEFT, BeyondEFT, CSUSY} t-channel \cite{Buchmueller:2014yoa, SiM}.
