\begin{flushleft}
\textcolor{magenta}{For each simplified model, the bulk of the universe's dark matter} is assumed to consist of a single species of particle: a weakly interacting SM singlet Dirac fermion, $\chi$, with mass $m_{\chi}$. For simplicity, it is also assumed that the mediator couples only to $\chi$ and the SM quarks, with coupling strengths $g_{\chi}$ and $g_q$ respectively. For the purposes of this paper, $g_q = g_{\chi}=f$ where $q = (u, d, c, s, b, t)$.

\hspace{1cm}The first simplified model - corresponding to the D5 effective operator - couples the dark and SM sectors via the $s$-channel exchange of a vector mediator (labelled $\xi$). The Lagrangian interaction term for this model is:
\begin{equation}
\label{L_int_sV}
\mathcal{L}_{int} = -\sum\limits_{i=1}^{3} f\xi_{\mu} \left[\bar{q}\gamma^{\mu}q - \bar{\chi}\gamma^{\mu}\chi\right]
\end{equation}
where $i$ represents an index running over quark family flavour. \textcolor{blue}{When applying constraints to the collider production cross-section for the process $pp \rightarrow X + \bar{\chi}\chi$, it is also important to consider the impact of the mediator decay width, $\Gamma$. The reason being that $\sigma\left(pp \rightarrow X + \bar{\chi}\chi\right)$ scales as:}
\begin{equation}
\label{cross_section}
\frac{f^{4}E^{2}}{(q^{2} - M^{2})^{2} + \Gamma^{2}/4}
\end{equation}
where $E$ is the centre-of-mass energy of two colliding partons and $q$ is the four-momentum transferred by the mediator. In previous analyses it was customary to consider mediators of fixed width ranging from $\Gamma = M/8\pi$ to $\Gamma = M/3$ \cite{METSig, Fox:2012ee} \footnote{\textcolor{magenta}{Also add a reference for the recent monojet paper.}}. This approach is motivated by the observation that, where the mediator is exchanged in the $s$-channel and produced on-shell, $\sigma\left(pp \rightarrow X + \bar{\chi}\chi\right)$ is maximally enhanced when $\Gamma$ is small, as expected from equation \ref{cross_section}\footnote{For a more in depth discussion see Ref. \cite{METSig}.}. The smallest width, $\Gamma = M/8\pi$, corresponds to a mediator which couples only to one flavour of quark and helicity with $g_{q}g_{\chi} = 1$ \cite{METSig}. In this paper, the mediator width is expanded to include coupling to all kinematically accessible quark flavours (in addition to the single flavour case). This move is based on a two-fold logic: firstly, there is no theoretical prejudice for the mediator to couple exclusively to the first generation of quarks. Secondly, signal events are generated with the presupposition that all quarks can annihilate to produce the mediator. It subsequently makes sense that the mediator should be allowed to decay into quark pairs of any flavour. For the above model, the full decay width is:
\begin{equation}
\label{gamma_sV}
\Gamma_{sV} = \frac{f^{2} M}{12\pi}\left(1 + \frac{2m_{\chi}^{2}}{M^{2}}\right)\left(1 - \frac{4m_{\chi}^{2}}{M^{2}}\right)^{1/2} + \sum_{\substack{q}}\frac{N_{c}f^{2}M}{12\pi}\left(1 + \frac{2m_{q}^{2}}{M^{2}}\right)\left(1 - \frac{4m_{q}^{2}}{M^{2}}\right)^{1/2}
\end{equation}
where $N_{C}$=3, $m_{\chi} \leq M/2$ is required and the summation over $q$ is restricted to quarks with $m_{q} \leq M/2$. Note that the $s$-channel vector mediator model has already been studied extensively \cite{Buchmueller:2014yoa, Chatrchyan:2013qha, Aad:2012hf, Harris:2014hga} and so serves as a good basis for the comparison of constraints. In addition to this model, a further three simplified models are considered here.

\hspace{1cm}The second model - corresponding to the D8 effective operator - is also well studied \cite{} and is characterised by the $s$-channel exchange of a heavy mediator with axial-vector couplings to both the dark and SM sectors. The Lagrangian interaction term for this model is:
\begin{equation}
\label{L_int_sAV}
\mathcal{L}_{int} = \sum\limits_{i=1}^{3} f\xi_{\mu} \left[\bar{q}\gamma^{\mu}\gamma_{5}q - \bar{\chi}\gamma^{\mu}\gamma_{5}\chi\right]
\end{equation}
and the decay width of the mediator is:
\begin{equation}
\label{gamma_sAV}
\Gamma_{sAV} = \frac{f^{2} M}{12\pi}\left(1 - \frac{4m_{\chi}^{2}}{M^{2}}\right)^{3/2} + \sum_{\substack{q}}\frac{N_{c}f^{2} M}{12\pi}\left(1 - \frac{4m_{q}^{2}}{M^{2}}\right)^{3/2}
\end{equation}
where again $m_{\chi} \leq M/2$ is required and the the summation over $q$ is restricted to quarks with $m_{q} \leq M/2$.

\hspace{1cm}The third model - corresponding to the D1 effective operator - is characterised by the $s$-channel exchange of a scalar mediator (labeled $\eta$). The Lagrangian interaction term for this model is:
\begin{equation}
\label{L_int_sS}
\mathcal{L}_{int} = \sum\limits_{i=1}^{3} f \left[\bar{q}q\eta + \bar{\chi}\chi\eta\right]
\end{equation}
and the width of the mediator is:
\begin{equation}
\label{gamma_sS}
\Gamma_{sS} = \frac{f^{2} M}{8\pi}\left(1 - \frac{4m_{\chi}^{2}}{M^{2}}\right)^{3/2} + \sum_{\substack{q}}\frac{N_{c}f^{2} M}{8\pi}\left(1 - \frac{4m_{q}^{2}}{M^{2}}\right)^{3/2}
\end{equation}
\textcolor{blue}{where again $m_{\chi} \leq M/2$ is required and the the summation over $q$ is restricted to quarks with $m_{q} \leq M/2$.}

\hspace{1cm}The last model chosen for analysis is characterised by the $t$-channel exchange of a scalar mediator, with interaction Lagrangian:
\begin{equation}
\label{L_int_tS}
\mathcal{L}_{int} = f \left[\bar{\chi}q\eta + \chi\chi\eta\right]
\end{equation}
and the width of the mediator is: 
\begin{equation}
\label{gamma_tS}
\Gamma_{tS} = \sum_{\substack{q}} \frac{f^{2}M}{16\pi}\left(1 - \frac{m_{q}^{2}}{M^{2}} - \frac{m_{\chi}^{2}}{M^{2}}\right)\sqrt{\left(1 - \frac{m_{q}^{2}}{M^{2}} + \frac{m_{\chi}^{2}}{M^{2}}\right)^{2} - 4\frac{m_{\chi}^{2}}{M^{2}}}
\end{equation}
where it is required that $M \geq m_{\chi} - m_{q}$. \textcolor{red}{Is this maths correct?}

The above model juxtaposes nicely with the $s$-channel models and is motivated separately by analogy with a common aspect of Supersymmetric models: neutralino dark matter interacting with the SM sector via $t$-channel exchange of a squark \cite{SUSYDM}. Note that in the Supersymmetric scenario the dark matter particle is instead a Majorana fermion. Simplified models in which the dark matter particle is a Majorana fermion are not considered here as they are not expected to produce significantly different results, except in the case of the $s$-channel vector mediator model which vanishes if $\chi$ is a Majorana fermion \cite{METSig}.
\bigskip

\textcolor{magenta}{Include:}
\begin{enumerate}
\item \textcolor{magenta}{Brief motivation for choice of simplified models? (eg. something like "we consider the most straightforward UV-completions of the D1, D5 and D8 effective operators, corresponding to the s-channel scalar, vector and axial-vector models respectively.}
\item \textcolor{magenta}{The interaction Lagrangians for our four SiMs along with an explanation for why we only assume coupling to SM quarks.}
\item \textcolor{magenta}{The assumptions and the decay widths associated with our models (?).}
\end{enumerate}
\end{flushleft}

%Resolving the mediator leads to two possibilities: the mediating particle is exchanged in the s-channel, in which case it may be colour neutral, or it is exchanged in the t-channel in which case it is necessarily coloured \cite{}. 

%\cite{ValidEFT, BeyondEFT, CSUSY} t-channel \cite{Buchmueller:2014yoa, SiM}.
