\begin{flushleft}
For each simplified model, the bulk of the universe's dark matter is assumed to consist of a single species of particle: a weakly interacting SM singlet Dirac fermion, $\chi$, with mass $m_{\chi}$. For simplicity, it is also assumed that the mediator couples only to $\chi$ and the SM quarks, with coupling strengths $g_{\chi}$ and $g_q$ respectively. For the purposes of this paper, $g_q = g_{\chi}=f$.

\hspace{1cm} The first simplified model - corresponding to the D1 effective operator - couples the dark and SM sectors via the $s$-channel exchange of a scalar mediator (labeled $\eta$). The Lagrangian interaction term for this model is:
\begin{equation}
\label{L_int_sS}
\mathcal{L}_{int} = \sum\limits_{i=1}^{3} f \left[\bar{q}q\eta + \bar{\chi}\chi\eta\right]
\end{equation}
where $i$ represents an index running over quark family flavour. \textcolor{magenta}{Put a discussion here of the need to include/advantage of including the full decay width of the mediator over the previous trend of using $M/3$ and $M/8\pi$.} Also, the decay width of the scalar mediator is given by:
\begin{equation}
\label{gamma_sS}
\Gamma_{sS} = \frac{f^{2} M}{8\pi}\left(1 - \frac{4m_{\chi}^{2}}{M^{2}}\right)^{3/2} + \sum_{\substack{q}}\frac{N_{c}f^{2} M}{8\pi}\left(1 - \frac{4m_{q}^{2}}{M^{2}}\right)^{3/2}
\end{equation}
where $N_{C}$=3 and $q = (u, d, c, s, b, t)$.
The second simplified model - corresponding to the D5 effective operator - is characterised by the s-channel exhange of a vector mediator (labelled $\xi$). The Lagrangian interaction term for this model is:
\begin{equation}
\label{L_int_sV}
\mathcal{L}_{int} = -\sum\limits_{i=1}^{3} f\xi_{\mu} \left[\bar{q}\gamma^{\mu}q - \bar{\chi}\gamma^{\mu}\chi\right]
\end{equation}
and the decay width of the vector mediator is:
\begin{equation}
\label{gamma_sV}
\Gamma_{sV} = \frac{f^{2} M}{12\pi}\left(1 + \frac{2m_{\chi}^{2}}{M^{2}}\right)\left(1 - \frac{4m_{\chi}^{2}}{M^{2}}\right)^{1/2} + \sum_{\substack{q}}\frac{N_{c}f^{2}M}{12\pi}\left(1 + \frac{2m_{q}^{2}}{M^{2}}\right)\left(1 - \frac{4m_{q}^{2}}{M^{2}}\right)^{1/2}
\end{equation}

The third simplified model - corresponding to the D8 effective operator - is characterised by the s-channel exchange of a heavy mediator with axial-vector couplings to both the dark and SM sectors. The Lagrangian interaction term for this model is:
\begin{equation}
\label{L_int_sAV}
\mathcal{L}_{int} = \sum\limits_{i=1}^{3} f\xi_{\mu} \left[\bar{q}\gamma^{\mu}\gamma_{5}q - \bar{\chi}\gamma^{\mu}\gamma_{5}\chi\right]
\end{equation}
and the decay width of the mediator for this model is:
\begin{equation}
\label{gamma_sAV}
\Gamma_{sAV} = \frac{f^{2} M}{12\pi}\left(1 - \frac{4m_{\chi}^{2}}{M^{2}}\right)^{3/2} + \sum_{\substack{q}}\frac{N_{c}f^{2} M}{12\pi}\left(1 - \frac{4m_{q}^{2}}{M^{2}}\right)^{3/2}
\end{equation}
The last model chosen for analysis is characterised by the $t$-channel exchange of a scalar mediator, with interaction Lagrangian:
\begin{equation}
\label{L_int_tS}
\mathcal{L}_{int} = f \left[\bar{\chi}q\eta + \chi\chi\eta\right]
\end{equation}
Note that, in accordance with current literature, it is now assumed that mediator couples only to the first generation of quarks \cite{t-channel}. Additionally, it is assumed that $m_{q}/M << 1$, which gives the following expression for the decay width per quark flavour: 
\begin{equation}
\label{gamma_tS}
\Gamma_{tS} = \frac{f M}{16\pi}\left(1 - \frac{m_{\chi}^{2}}{M^{2}}\right)^{2}
\end{equation}
This model juxtaposes nicely with the $s$-channel models and is motivated separately by analogy with a common aspect of Supersymmetric models: neutralino DM interacting with the SM sector via $t$-channel exchange of a squark \cite{SUSYDM}. Note that in the Supersymmetric scenario the DM particle is instead a Majorana fermion. Models in which the DM particle is a Majorana fermion are not considered here as they are not expected to produce significantly different results, except in the case of the $s$-channel vector mediator model which vanishes if $\chi$ is a Majorana fermion \cite{METSig}.
\bigskip

\textcolor{magenta}{Include:}
\begin{enumerate}
\item \textcolor{magenta}{Brief motivation for choice of simplified models? (eg. something like "we consider the most straightforward UV-completions of the D1, D5 and D8 effective operators, corresponding to the s-channel scalar, vector and axial-vector models respectively.}
\item \textcolor{magenta}{The interaction Lagrangians for our four SiMs along with an explanation for why we only assume coupling to SM quarks.}
\item \textcolor{magenta}{The assumptions and the decay widths associated with our models (?).}
\end{enumerate}
\end{flushleft}

%Resolving the mediator leads to two possibilities: the mediating particle is exchanged in the s-channel, in which case it may be colour neutral, or it is exchanged in the t-channel in which case it is necessarily coloured \cite{}. 

%\cite{ValidEFT, BeyondEFT, CSUSY} t-channel \cite{Buchmueller:2014yoa, SiM}.
