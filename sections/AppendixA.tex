\subsection{Monojet Channel}
\label{monojet_validation}
\begin{flushleft}
As mentioned in Section \ref{monojet_constraints}, the signal selection and Monte Carlo (MC) generation procedures for the monojet channel were validated via reproduction of the ATLAS limits on the suppression scale, M$_{*}$, for the s-channel vector simplified model. A comparison of SR7 limits for a representative sample of mediator masses with $m_{\chi} = $ 50 GeV, $\Gamma = M/8\pi$ and $\sqrt{g_{q}g_{\chi}} = 1$ is presented in Table \ref{M_star_limits_monojet}. In general, good agreement is observed between the ATLAS and reproduced limits, with a maximum discrepancy of $>$11\%. We note that a discrepancy of a few percent is expected and allowed for two reasons. Firstly, the Monte Carlo generation procedure employed in this analysis does not include a full simulation of the ATLAS detector. \comm{Instead, reconstruction effects are modelled somewhat crudely by smearing the leading jet $p_{T}$ distribution by a conservative factor of 5\%.} Secondly, the 95\% CL uncertainties associated with the ATLAS limits are estimated using the HistFitter package while a different approach is favoured for this analysis (see appendix \ref{AppendixB}). \comm{Given that the difference between the ATLAS and reproduced limits is consistently negative, we consider this approach acceptable.}

%In general, our approach  results in more conservative limits on M$_{*}$ but ultimately removes the hassle of a full histFitter analysis. Importantly, agreement between M$_{*}^{\mbox{\tiny ATLAS,95}}$ and the reproduced nominal\footnote{Nominal in this case explicitly refers to a quantity which does not yet include statistical or systematic uncertainties.} M$_{*}$ values (denoted M$_{*}^{\mbox{\tiny R,N}}$ in Table \ref{M_star_limits_monojet}) is still reasonably high - within \comm{something percent} - validating our signal selection and MC generation procedures.}

%Also note that while the signal samples used for validation of the monojet channel are generated for the processes $pp \rightarrow j\chi\bar{\chi}$ and $pp \rightarrow jj\chi\bar{\chi}$ where $j$ is a final state jet, the signal samples used to constrain the the simplified models discussed in Section \ref{SiM_models} also include the tree-level process $pp \rightarrow \chi \bar{\chi}$. \comm{This approach is taken as...}

%\begin{table}[!htbp]
%\centering
%\begin{tabular}{c|c|c|c|c|c}
% \hline
% \hline
% $M$ [TeV] & M$_{*}^{\mbox{\tiny ATLAS,95}}$ [GeV] & M$_{*}^{\mbox{\tiny R,N}}$ [GeV] & Difference [\%] & M$_{*}^{\mbox{\tiny R,95}}$ [GeV] & Difference [\%] \\ 
% \hline
%0.05 & 91 & 100.05 & +9.05 & 94.33 & +3.53 \\
%%0.1 & 217 & 322.70 & +32.75 & 280.57 & +22.66 \\
%0.3 & 1151 & 1288.15 & +10.65 & 1092.52 & $-$5.35 \\
%0.6 & 1868 & 2013.68 & +7.23 & 1668.27 & $-$11.97 \\
%1 & 2225 & 2363.06 & +5.84 & 1975.58 & $-$12.63 \\
%3 & 1349 & 1479.66 & +8.83 & 1274.73 & $-$5.83 \\
%%6 & 945 & 856.37 & $-$10.35 & 730.98 & $-$29.28 \\
%10 & 928 & 1000.93 & +7.29 & 842.03 & $-$10.21 \\
%30 & 914 & 989.49 & +7.54 & 838.34 & $-$9.03\\
% \hline
% \hline
%\end{tabular}
%\caption{Comparison of the ATLAS 95\% CL limits on M$_{*}$ (denoted M$_{*}^{\mbox{\tiny ATLAS,95}}$) with the reproduced nominal and reproduced 95\% CL limits on M$_{*}$ (denoted M$_{*}^{\mbox{\tiny R,N}}$ and M$_{*}^{\mbox{\tiny R,95}}$ respectively) for the s-channel vector mediator model with $m_{\chi} = $ 50 GeV, $\Gamma = M/8\pi$, $\sqrt{g_{q}g_{\chi}} = 1$ and QCUT = 80 GeV. Adapted from Ref. \cite{Aad:2015zva}.}
%\label{M_star_limits_monojet}
%\end{table}

\begin{table}[!htbp]
\centering
\begin{tabular}{c|c|c|c}
 \hline
 \hline
 $M$ [TeV] & M$_{*}^{\mbox{\tiny ATLAS,95}}$ [GeV] & M$_{*}^{\mbox{\tiny REPRO,95}}$ [GeV] & Difference [\%] \\ 
 \hline
 0.05 & 91 & 84.20 & $-$7.47 \\%$-$8.08 \\
%0.1 & 217 & 246.54 & +11.96 \\
0.3 & 1151 & 1088.87 & $-$5.40 \\ %$-$5.71 \\
0.6 & 1868 & 1697.19 & $-$9.14 \\% $-$10.06 \\
1 & 2225 & 1986.67 & $-$10.71 \\ %$-$12.00 \\
3 & 1349 & 1241.93 & $-$7.94 \\% $-$8.62 \\
%6 & 945 & 721.87 & $-$30.91 \\
10 & 928 & 844.33 & $-$9.02 \\% $-$9.91 \\
30 & 914 & 834.56 & $-$8.69 \\% $-$9.52\\

%0.05 & 91 & 94.33 & +3.53 \\
%%0.1 & 217 & 280.57 & +22.66 \\
%0.3 & 1151 & 1092.52 & $-$5.35 \\
%0.6 & 1868 & 1668.27 & $-$11.97 \\
%1 & 2225 & 1975.58 & $-$12.63 \\
%3 & 1349 & 1274.73 & $-$5.83 \\
%%6 & 945 & 730.98 & $-$29.28 \\
%10 & 928 & 842.03 & $-$10.21 \\
%30 & 914 & 838.34 & $-$9.03\\
 \hline
 \hline
\end{tabular}
\caption{Comparison of the 95\% confidence level upper limits on M$_{*}$ from this work (M$_{*}^{\mbox{\tiny REPRO,95}}$) and from the ATLAS monojet analysis (M$_{*}^{\mbox{\tiny ATLAS,95}}$). The values shown in columns two and three are   for the processes $pp \rightarrow j\chi\bar{\chi}$ and $pp \rightarrow jj\chi\bar{\chi}$ for the s-channel vector mediator model with $m_{\chi} = $ 50 GeV, $\Gamma = M/8\pi$, $\sqrt{g_{q}g_{\chi}} = 1$ and QCUT = 80 GeV + 300 GeV. The difference shown in column four corresponds to 100$\times$(M$_{*}^{\mbox{\tiny REPRO,95}}$ $-$ M$_{*}^{\mbox{\tiny ATLAS,95}}$)/M$_{*}^{\mbox{\tiny ATLAS,95}}$.
Adapted from Ref. \cite{Aad:2015zva}. \comm{Just a thought - would it make more sense to calculate these as (ATLAS - us)/ATLAS? Then we say ours are x\% smaller/more conservative than the official values. Also added bonus of reducing the percentages slightly. -Mia} Done! - Millie}
\label{M_star_limits_monojet}
\end{table}
\end{flushleft}

\subsection{Mono-Z Channel}
\label{monoZ_validation}
\begin{flushleft}

\bigskip

The ATLAS mono-$Z$ analysis result includes an upper limit on the coupling $g_{q \chi}$ for a $t$-channel simplified model that is very similar to the model tested here, and so is used for validating our procedure. The most significant differences are in the number of mediating particles --- the ATLAS model includes just two mediators ($up$- and $down$-type) compared to our six --- and in the nature of the DM particle, which is taken to be Majorana. This latter choice does not impact the kinematic behaviour, but does impact the cross section. Additionally, while we use a universal coupling $g_{q \chi}$ to all three quark generations, the analysis used a model which set $g_{t,b \chi}$ = 0.

\begin{table}
\begin{center}
\begin{tabular}{ c | c | c | c | c }
\hline
\hline
$m_{\chi}$ & $M_{med}$ & $g_{q \chi}^{95\%\mathrm{CL}}$ & $g_{q \chi}^{95\%\mathrm{CL}}$ & Difference \T \\
$[$GeV$]$ & $[$GeV$]$ & (ATLAS) & (this work) & $[\%]$ \B \\
\hline
10 & 200 & 1.9 & 2.0 & -5.3 \T \\
 & 500 & 2.8 & 3.2 & -14.3 \\
 & 700 & 3.5 & 4.4 & -25.7 \\
 & 1000 & 4.5 & 5.2 & -15.6 \\
200 & 500 & 3.4 & 4.0 & -17.6 \T \\
 & 700 & 4.2 & 4.5 & -7.1 \\
 & 1000 & 5.2 & 5.3 & -1.9 \\
400 & 500 & 5.5 & 5.7 & -3.6 \T \\
 & 700 & 6.1 & 6.5 & -6.6 \\
 & 1000 & 7.2 & 7.4 & -2.8 \\
1000 & 1200 & 23.3 & 24.1 & -3.4 \T \B \\
\hline
\hline
\end{tabular}
\end{center}
\caption{Comparison of the upper limit on $g_{q \chi}$ from the ATLAS analysis (middle column) and this work.}
\label{tab:monoZvalidation}
\end{table} 

Table \ref{tab:monoZvalidation} shows the 95\% CL upper limits on $g_{q \chi}$ that we calculate using the same $t$-channel model and our own generation procedure, compared with the limits on this same variable taken from the ATLAS analysis. The difference as a percentage of the ATLAS limit is also shown in the table. We see reasonable agreement; most of the 11 points in parameter space are within 10\% of the ATLAS limits, and all are within 26\%. Additionally, our results are consistently more conservative, which is to be expected due to the less sophisticated nature of our generation procedure. Similarly to the mono-jet validation, the dominant effects are due to the use of $p_{\mathrm{T}}$ smearing applied to the leptons, rather than considering the full reconstruction effects, and the simple systematic treatment that was used with HistFitter.

\end{flushleft}

\iffalse

\subsection{Mono-W/Z Channel}
\label{monoWZ_validation}
\begin{flushleft}
\comm{Johanna, please put your validation results here.}
\end{flushleft}

\fi
